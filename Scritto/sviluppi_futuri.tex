\documentclass[tesi.tex]{subfiles}
\begin{document}
\chapter{Considerazioni finali e sviluppi futuri}
L'applicazione presenta notevoli possibilit\`a di sviluppo.
Preliminarmente provvederei ad una pulizia generale del codice che
presenta diversi punti dove non \`e ottimizzato e che risente del
fatto che \`e stata sviluppata da pi\`u persone in momenti diversi non
sempre in modo coerente. Sono inoltre presenti molti frammenti di
codice morto e classi non usate. 

Spesso viene usato \emph{GATE} in punti dove non \`e molto appropriato, e le
\emph{pipe} non sono ottimizzate per l'esecuzione
differenziata. Inoltre non sembra molto appropriato l'utilizzo di
\emph{GATE} per il \emph{natural language processing} in lingua
italiana. Ho ricevuto notizie che per questo ultimo punto sono gi\`a
state valutate delle soluzioni che verranno applicate.

Un'altro punto su cui cercherei di intervenire \`e la struttura
dell'ontologia, questa non sfrutta a pieno le possibilit\`a offerte
dal web semantico, in particolar modo quelle offerte da \emph{OWL}. 

Per quanto riguarda le possibili estensioni dell'applicazione,
cercherei di risolvere uno dei maggiori problemi attualmente presenti:
le fasi di \emph{crawling} dipendono fortemente dalla struttura del
sito sul quale vengono lanciate. Questo \`e un problema complesso e
non di facile risoluzione implica che, ogni volta che
il sito sul quale viene lanciato il \emph{crawling} cambia struttura,
\`e necessario mettere mano in maniera pesante al codice.

Per mitigare questo problema propongo
l'adozione di una fase di preparazione delle pagine tramite
trasformazioni \emph{XSLT} come in figura \ref{fig:consFinXslt}.
\begin{figure}
  \begin{center}
    \begin{tikzpicture}[auto]
      \node [weblist, text width = 1.75cm] (pagine) {Pagine web};
      \node [app, text width = 1.75cm, right = 1cm of pagine] (xslt) {Trasformaz. XSLT};
      \node [file, text width = 1.75cm, right = 1cm of xslt] (doc) {Documento XML in forma nota};
      \node [app, text width = 1.75cm, right = 1cm of doc] (crawl) {Crawling};

      \draw [freccia] (pagine) to (xslt);
      \draw [freccia] (xslt) to (doc);
      \draw [freccia] (doc) to (crawl);
    \end{tikzpicture}
  \end{center}
  \caption{Proposta di pre-fase}
  \label{fig:consFinXslt}
\end{figure}

L'idea \`e di usare le potenzialit\`a delle trasformazioni \emph{XSLT}
per trasformare l'insieme delle pagine web desiderate in un documento, o insieme
di documenti, in formato \emph{XML} con una struttura ben definita a
priori, di fatto si tratterebbe di creare uno \emph{stylesheet} per le
pagine del sito. Quindi il \emph{crawler} verrebbe lanciato su questo documento
\emph{XML} e non pi\`u direttamente sulle pagine. I vantaggi sono che, in caso di modifica alla struttura del
sito, o anche nel caso si voglia applicare il crawling ad altri siti,
le modifiche necessarie per rendere nuovamente operativo il programma
si riducono alla modifica del solo schema di trasformazione
\emph{XSLT}, senza toccare minimamente il codice \emph{Java}
dell'applicazione.

In tal senso ho anche provveduto a effettuare degli esperimenti per
valutarne la fattibilit\`a, e questi hanno avuto risultati positivi.

Un altro punto a favore di questa idea \`e che la sua attuazione
potrebbe aprire altre vie per il futuro di questa applicazione, ad
esempio \`e possibile immaginare la creazione di un \emph{tool} che
aiuti nello sviluppo dello \emph{stylesheet} \emph{XSLT} applicato ad un certo sito,
rendendo l'operazione di aggiornamento della trasformazione pi\`u
semplice che non scrivendo da zero lo schema.

Per i dettagli sulle trasformazioni \emph{XSLT} vedere la sezione
\ref{sec:xslt}.
\end{document}
