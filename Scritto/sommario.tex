%*******************************************************
% Abstract
%*******************************************************
%\renewcommand{\abstractname}{Abstract}
%\pdfbookmark[1]{Abstract}{Abstract}
\begingroup
\let\clearpage\relax
\let\cleardoublepage\relax
\let\cleardoublepage\relax

\chapter*{Sommario}
Il presente testo \`e il risultato del lavoro di \emph{stage} svolto
presso il laboratorio \emph{DISIT} del Dipartimento di Sistemi e
Informatica dell'Universit\`a degli Studi di Firenze.

Il lavoro effettuato si inserisce nel progetto \emph{OSIM} che ha
l'obbiettivo di creare una base di conoscenza semantica riguardante il
personale universitario. In particolare \`e
stata sviluppata la parte dell'applicazione che riguarda il \emph{data
mining} delle informazioni riguardanti le persone, i corsi associati a
queste, e le competenze di ogni persona.

Il progetto fa uso delle
tecnologie \emph{RDF}, \emph{RDFS} e \emph{OWL} per l'organizzazione
della base di conoscenza semantica. Per quanto riguarda
l'applicazione, questa \`e sviluppata in \emph{Java} facendo uso delle
\emph{Servlet} e delle \emph{JSP} per fornire l'interfaccia \emph{web}
con l'utente.

Il testo \`e diviso in due parti: una parte consiste in una
sostanziosa lettura sullo stato dell'arte delle tecnologie usate,
dalle basi dell'\emph{XML} e delle tecnologie correlate sino ad una
visione eterogenea sulle tecnologie del web semantico;
l'altra parte concerne la descrizione del lavoro effettuato e un'analisi
dettagliata dell'applicazione nelle sue parti.

\endgroup			

\vfill
