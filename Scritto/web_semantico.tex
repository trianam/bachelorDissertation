\documentclass[tesi.tex]{subfiles}
\begin{document}
\chapter{Web semantico}
Il web semantico nasce dalla necessit\`a di rendere i contenuti in rete
disponibili non solo agli esseri umani, ma di cercare di renderli
accessibili anche alle macchine. 

Attualmente la maggior parte dell'informazione contenuta nel web \`e
concepita principalmente per essere fruita dagli
esseri umani, se io voglio sapere quali sono i corsi di un certo
professore devo andare sulla sua pagina web, cercare la sezione
insegnamenti ed elaborare il testo. Se volessi far fare lo stesso
lavoro ad una macchina dovrei prevedere tecniche di intelligenza
artificiale e di \emph{NLP} (Natural Language Processing). Il web
semantico cerca di fornire un'alternativa all'uso di questi mezzi
agendo direttamente all'origine, ossia organizzando il contenuto
informativo di una certa pagina in modo da rendere un programma in
grado di comprenderne il contenuto semantico senza effettuare alcun
processo sul testo. Un programma quindi potrebbe direttamente venire a
conoscenza della relazione che intercorre tra un certo professore e i
corsi da lui insegnati.

Il \emph{web semantico} \`e sostenuto dal
\emph{W3C}\footnote{L'indirizzo del gruppo specifico che si occupa di
  web semantico \`e: http://www.w3.org/2001/sw/.} ed \`e stato promosso
inizialmente da \emph{Tim Berners Lee}. 

\section{Visione d'insieme}\label{sec:visione-insieme}
Nascendo come evoluzione del web come attualmente lo conosciamo, il
web semantico riprende alcuni concetti ad esso associati, nel
dettaglio:
\begin{itemize}
\item ognuno pu\`o inserire qualsiasi tipo di contenuto ed
  informazione e pu\`o contribuire alla creazione della conoscenza
  per un certo argomento;
\item vige l'ipotesi del mondo aperto, ossia il fatto di non trovare
  alcuna sentenza su un dato argomento non implica la negazione di
  questa sentenza;
\item la stessa entit\`a pu\`o essere chiamata con nomi diversi, ad
  esempio a seconda di chi ne parla.
\end{itemize}
Nella progettazione del web semantico \`e stato deciso di tenere in
considerazione questi punti, cosicch\'e potesse essere
\emph{facilmente} integrato all'interno della struttura esistente.

Gli standard del web semantico sono organizzati a \emph{strati} con un
livello crescente di \emph{espressivit\`a}, nel senso che ogni strato
fornisce un linguaggio in grado di modellare concetti in domini
diversi e di complessit\`a crescente, inoltre ogni linguaggio basa la sua
esistenza su quello dello strato precedente. Questi strati sono:
\begin{description}
\item[RDF] \emph{Resource Description Framework}, \`e il \emph{layer} di base, questo permette di
  esprimere qualsiasi sentenza di base e fornisce un modello comune
  per farlo;
\item[RDFS] \emph{RDF Schema Language}, \`e un linguaggio che permette
  l'espressivit\`a tipica dei linguaggi ad oggetti, permettendo di
  organizzare \emph{classi}, \emph{sottoclassi} e \emph{propriet\`a};
\item[OWL] \emph{Web Ontology Language}, fornisce la potenza della
  logica (descrittiva) al web semantico aggiungendo capacit\`a di
  \emph{reasoning}.
\end{description}
Di seguito verranno analizzati dettagliatamente tutti questi livelli. \`E bene sapere
anche che, oltre a questi, esistono altri livelli, per esempio esiste
\emph{RDFS-Plus} che \`e un sottoinsieme di \emph{OWL} e quindi un
livello intermedio tra \emph{RDFS} e \emph{OWL}. Inoltre, essendo una
tecnologia in fase di sviluppo, esistono anche livelli previsti ma ancora non
esistenti. In figura \ref{fig:stackSemantico} si pu\`o vedere lo
stack cos\`i come progettato da \emph{Tim Berners Lee}, la linea
tratteggiata divide la parte realizzata (inferiore) da quella non
implementata (superiore).
\image{img/semantic-web-stack.eps}{Stack web semantico}{fig:stackSemantico}
Questi strati non sono da considerarsi livelli nettamente separati
gli uni dagli altri, \`e
possibile immaginarli come strutture che aumentano le funzionalit\`a di
un'unica entit\`a.
I tre strati inferiori appartengono alle tecnologie del web
preesistenti, non
strettamente collegate con il web semantico ma su cui esse si
appoggiano:
\begin{description}
\item[URI] \emph{Uniform Resource Identifier} come dice il nome
  permettono di identificare univocamente una risorsa nel web,
  successivamente verr� analizzato dettagliatamente il loro uso nel web semantico;
\item[Unicode] \`e la codifica dei caratteri;
\item[XML] \emph{eXtensible Markup Language} \`e un linguaggio di
  \emph{markup} che permette la creazione di documenti strutturati.
\end{description}
In realt\`a \emph{RDF} non si basa necessariamente su \emph{XML} e
neppure su \emph{Unicode} in
quanto esistono anche rappresentazioni alternative che verranno descritte in
seguito nel dettaglio.

Brevemente, le parti non ancora implementate sono:
\begin{description}
\item[Rules] permetteranno di aggiungere il supporto per le regole, in
  pratica permetteranno di ampliare la \emph{logica descrittiva} di
  \emph{OWL} rendendola (circa) una \emph{logica a clausole di Horn};
\item[Cryptography] aggiunger\`a il supporto alla crittografia per
  verificare le fonti;
\item[Trust] verificher\`a l'affidabilit\`a delle sentenze derivate
  verificando le fonti e basandosi sulla \emph{logica formale};
\item[User interface] sar\`a l'interfaccia utente che permetter\`a
  agli umani l'uso delle applicazioni web semantiche.
\end{description}

Per quanto riguarda \emph{SPARQL}, esso \`e un linguaggio di quering e
serve per poter interrogare i dati basati su \emph{RDF}.

\section{Resource Description Framework (RDF)}\label{sec:rdf}
L'\emph{RDF} indica il modo in cui i dati o meglio le \emph{risorse} e
le relazioni tra di esse
vengono rappresentate. Con risorsa si pu\`o intendere qualsiasi cosa
in quanto \emph{RDF} \`e stato pensato per avere un alto potere
descrittivo. I concetti base dell'\emph{RDF} sono gli \emph{URI} e le
\emph{triple}.
Degli uri e dei namespaces abbiamo gi\`a parlato nella sezione
\ref{sec:urienamespaces}, e abbiamo anche parlato della sintassi
\emph{qname}. Nell'\emph{RDF} (perlomeno nelle sue serializzazioni
\emph{XML/RDF} e \emph{N3}
che saranno analizzate successivamente nella sezione \ref{sec:serializzazioni})
si possono usare i \emph{namespaces} e la
sintassi \emph{qname} in modo assolutamente analogo a quello dell'\emph{XML}.

Sia l'\emph{RDF} che l'\emph{RDFS} che \emph{OWL} prevedono dei
\emph{namespaces}, questi sono rispettivamente:
\begin{itemize}
  \item \xml{http://www.w3.org/1999/02/22-rdf-syntax-ns\#};
  \item \xml{http://www.w3.org/2000/01/rdf-schema\#};
  \item \xml{http://www.w3.org/2002/07/owl\#}.
\end{itemize}
Quando in seguito verrano citati i prefissi \xml{rdf:}, \xml{rdfs:}
e \xml{owl:}, se non diversamente specificato, dovranno essere
considerati prefissi mappati ai sudetti \emph{namespaces}.

\subsection{Triple}
Cos\`i come gli \emph{URI} rappresentano le risorse, le \emph{triple}
implementano le relazioni che intercorrono tra di esse e le
propriet\`a che esse hanno. Queste triple in pratica rappresentano
delle sentenze,
concettualmente assumono la forma di tre valori ordinati che prendono in
prestito la terminologia dalla grammatica: \emph{soggetto},
\emph{predicato} e \emph{oggetto}. Il \emph{soggetto} si riferisce ad
una specifica risorsa quindi deve essere indicato con un \emph{URI},
il \emph{predicato} indica la propriet\`a che la tripla descrive e anche
questa deve essere indicata univocamente con un \emph{URI},
l'\emph{oggetto} \`e il valore della propriet\`a, questo pu\`o essere
sia un'altra risorsa (quindi un \emph{URI}) oppure un letterale.

Un modo per rappresentare e comprendere meglio un insieme di triple
\`e quello di disegnare un grafo orientato facendo in modo che i
soggetti e gli oggetti siano i nodi del grafo (distinguendo
gli oggetti che sono \emph{URI} da quelli letterali) e tanti archi
quante sono le triple, orientati da soggetto ad oggetto.
Ad esempio: volendo indicare delle sentenze riguardo ad un
professore, innanzitutto si deve indicare il soggetto ed i
predicati con degli uri, poi si pu\`o usare
\xml{http://www.unifi.it/rdf/namespace\#} o qualcosa di simile
come namespace per gli \emph{URI} e si indica in seguito con il
prefisso \xml{unifi}. Si crea cos\`\i la prima tripla dove
vengono usati i \emph{qnames} per chiarezza e compattezza, anche se questi
non potranno essere usati in tutte le
serializzazioni di \emph{RDF}; i letterali invece vanno racchiusi
tra virgolette.
\begin{triples}[Esempio di tripla][tab:esempioTripla]
  \triple{unifi:mario-rossi}{unifi:nome}{``Mario Rossi''}
\end{triples}
Ne risulta che la tripla indicata nella tabella \ref{tab:esempioTripla}
pu\`o essere rappresentata semplicemente con il grafo in figura
\ref{fig:esempioTripla}.
\begin{tgraph}[Esempio grafo tripla][fig:esempioTripla]
  \tgraphuri{n1}{0,0}{unifi:mario-rossi}
  \tgraphlet{n2}{1,0}{``Mario Rossi''}
  \tgrapharcl{n1}{n2}{unifi:nome}
\end{tgraph}

Viene elaborato di seguito un set pi\`u sostanzioso di quello
precedente. Se si vuole rappresentare delle informazioni su
del personale di un certo dipartimento, per esempio, si indica per
ognuno il nome, il ruolo e il
dipartimento di afferenza. Per quanto riguarda invece il dipartimento,
si indica il nome. Vengono definiti
i seguenti namespace per poter usare la forma compatta:
\xml{unifi="http://www.unifi.it/rdf/unifi\#"},
\xml{prof="http://www.unifi.it/rdf/prof\#"},
\xml{dip="http://www.unifi.it/rdf/dip\#"}.
Le \emph{propriet\`a} della persona sono rispettivamente:
\xml{unifi:ha-nome}, \xml{unifi:ha-ruolo} e
\xml{unifi:afferisce-a}. La propriet\`a del dipartimento, usando lo stesso \emph{namespace} di quelle della
persona, � \xml{unifi:ha-nome}. Come
\emph{URI} delle persone e dei dipartimenti vengono usati
\xml{prof:nome-cognome} e \xml{dip:nome}. Viene cos\`i costruita la
tabella \ref{tab:tripleProf},
\begin{triples}[Triple dei professori e dipartimenti][tab:tripleProf]
  \triple{prof:mario-rossi}{unifi:ha-nome}{``Mario Rossi''}
  \triple{prof:mario-rossi}{unifi:ha-ruolo}{``Professore''}
  \triple{prof:mario-rossi}{unifi:afferisce-a}{dip:sisinfo}
  \triple{prof:paolo-verdi}{unifi:ha-nome}{``Paolo Verdi''}
  \triple{prof:paolo-verdi}{unifi:ha-ruolo}{``Ricercatore''}
  \triple{prof:paolo-verdi}{unifi:afferisce-a}{dip:sisinfo}
  \triple{prof:franco-neri}{unifi:ha-nome}{``Franco Neri''}
  \triple{prof:franco-neri}{unifi:ha-ruolo}{``Professore''}
  \triple{prof:franco-neri}{unifi:afferisce-a}{dip:chimica}
  \triple{dip:sisinfo}{unifi:ha-nome}{``sistemi e informatica''}
  \triple{dip:chimica}{unifi:ha-nome}{``Chimica''}
\end{triples}
e si pu\`o notare che vi sono rappresentate le triple riguardanti tre persone e due
dipartimenti, per maggiore chiarezza viene mostrato il grafico corrispondente a
queste triple in figura \ref{fig:tripleProf}.
\begin{tgraph}[Grafo delle triple dei professori e dipartimenti][fig:tripleProf]
  \tgraphuri{mario-rossi}{0,0}{prof:mario-rossi}
  \tgraphuri{paolo-verdi}{0,1}{prof:paolo-verdi}
  \tgraphuri{franco-neri}{0,2}{prof:franco-neri}
  \tgraphuri{sisinfo}{0.75,0.5}{dip:sisinfo}
  \tgraphuri{chimica}{0.75,1.5}{dip:chimica}
  \tgraphlet{mario-rossi-nome}{-0.2,0.50}{``Mario Rossi''}
  \tgraphlet{mario-rossi-ruolo}{0.2,0.50}{``Professore''}
  \tgraphlet{paolo-verdi-nome}{-0.2,1.5}{``Paolo Verdi''}
  \tgraphlet{paolo-verdi-ruolo}{0.2,1.5}{``Ricercatore''}
  \tgraphlet{franco-neri-nome}{-0.2,2.5}{``Franco Neri''}
  \tgraphlet{franco-neri-ruolo}{0.2,2.5}{``Professore''}
  \tgraphlet{sisinfo-nome}{1,0.75}{``Sistemi e informatica''}
  \tgraphlet{chimica-nome}{1,1.75}{``Chimica''}
  \tgrapharcr{mario-rossi}{mario-rossi-nome}{unifi:ha-nome}
  \tgrapharcl{mario-rossi}{mario-rossi-ruolo}{unifi:ha-ruolo}
  \tgrapharcr{paolo-verdi}{paolo-verdi-nome}{unifi:ha-nome}
  \tgrapharcl{paolo-verdi}{paolo-verdi-ruolo}{unifi:ha-ruolo}
  \tgrapharcr{franco-neri}{franco-neri-nome}{unifi:ha-nome}
  \tgrapharcl{franco-neri}{franco-neri-ruolo}{unifi:ha-ruolo}
  \tgrapharcl{mario-rossi}{sisinfo}{unifi:afferisce-a}
  \tgrapharcr{paolo-verdi}{sisinfo}{unifi:afferisce-a}
  \tgrapharcr{franco-neri}{chimica}{unifi:afferisce-a}
  \tgrapharcl{sisinfo}{sisinfo-nome}{unifi:ha-nome}
  \tgrapharcl{chimica}{chimica-nome}{unifi:ha-nome}
\end{tgraph}

Una delle caratteristiche pi\`u utili dell'\emph{RDF} \`e la
possibilit\`a di \emph{combinare} dati da fonti diverse. Poich\'e
gli \emph{URI} identificano univocamente una
risorsa, si possono creare delle triple facendo riferimento a degli \emph{URI}
anche esterni al sistema. Ad esempio, se da qualche parte \`e
presente un \emph{RDF} che rappresenta l'elenco delle citt\`a
italiane, \`e possibile usare gli uri presenti per aggiungere delle
informazioni alle triple della figura \ref{fig:tripleProf} sulla
residenza dei professori o sulla sede dei dipartimenti. Tutto ci\`o
\`e conforme al principio enunciato nella sezione
\ref{sec:visione-insieme} secondo cui ognuno pu\`o aggiungere
informazione riguardante un concetto. Gli \emph{URI} garantiscono che
ci si riferisca effettivamente alle stesse cose in punti diversi
 e l'estrema flessibilit\`a dell'\emph{RDF} permette di
creare ed estendere nuovi concetti quando necessario.

\subsection{Blank nodes (bnodes)}\label{sec:bnodes}
I \emph{blank nodes}, anche detti \emph{bnodes}, servono ad indicare un
soggetto od un oggetto quando non si hanno informazioni su di esso, o
quando si desidera che tale entit\`a non abbia un \emph{URI}. 
Alcune serializzazioni permettono anche di usare i
\emph{bnodes} per il predicato, ma non verr\`a trattato il caso in
quanto non \`e di nessuna utilit\`a pratica.

Nella pratica i \emph{bnode} offrono la possibilit\`a di creare una o pi\`u
triple che hanno come \emph{soggetto} o \emph{oggetto} nessun uri,
nella sintassi si indica un \emph{bnode} con il prefisso
\xml{\_} seguito da un eventuale identificatore che indicher\`a univocamente un certo
\emph{bnode}. Concettualmente i \emph{bnodes} possono essere
interpretati come delle variabili quantificate esistenzialmente nella
logica dei predicati del primo ordine, le triple della tabella
\ref{tab:bnodes}
\begin{triples}[Esempio di \emph{bnode}][tab:bnodes]
  \triple{\_:X}{unifi:ha-nome}{``Mario Rossi''}
  \triple{\_:X}{unifi:ha-ruolo}{``Professore''}
  \triple{\_:X}{unifi:afferisce-a}{\_:Y}
  \triple{\_:Y}{unifi:ha-nome}{``sistemi e informatica''}
\end{triples}
sono estratte da quelle viste prima nella tabella \ref{tab:tripleProf}
a cui sono stati sostituiti gli uri delle \emph{entit\`a} che
rappresentavano il professore Mario Rossi e il dipartimento di
sistemi e informatica con dei \emph{bnodes}. Adesso il significato
che assumono \`e che ``esiste un \emph{qualcosa} che ha nome
Mario Rossi e ruolo Professore che afferisce a \emph{qualcos'altro} che
ha nome sistemi e informatica''.

I \emph{bnodes} sono molto usati in \emph{OWL} per la loro affinit\`a
con le variabili quantificate esistenzialmente.

\subsection{Serializzazioni}\label{sec:serializzazioni}
In questa sezione viene trattata la rappresentazione pratica dei dati
denominati nelle sezioni precedenti come triple, per poterli
scambiare e memorizzare.
Esistono tre tipi di serializzazioni: \emph{N-Triples}, \emph{Notation 3 RDF} e \emph{RDF/XML}.

\subsubsection{N-Triples}
 \`E la forma pi\`u semplice e consiste in un
    susseguirsi di triple delimitate da un carattere \ntriples{.}. Gli
    \emph{URI} si indicano racchiudendoli tra \ntriples{<...>}, i \emph{bnodes}
    esattamente come nella sezione \ref{sec:bnodes},
    ossia \ntriples{\_:X}, i letterali invece si racchiudono tra
    virgolette \ntriples{\"...\"}. Ad esempio, le triple della
    tabella \ref{tab:bnodes} sono serializzate in \emph{ntriples} come
    nel blocco seguente:
    \begin{ntriplesblock}
_:X <http://www.unifi.it/rdf/unifi#ha-nome> "Mario Rossi" .
_:X <http://www.unifi.it/rdf/unifi#ha-ruolo> "Professore" .
_:X <http://www.unifi.it/rdf/unifi#afferisce-a> _:Y .
_:Y <http://www.unifi.it/rdf/unifi#ha-nome> "sistemi e informatica" .
    \end{ntriplesblock}
    \`E da notare che la notazione \emph{n-triples} non permette l'uso
    dei \emph{qnames}.

 \subsubsection{Notation 3 RDF}
Anche abbreviata con \emph{N3}, \`e una
    serializzazione pi\`u compatta di \emph{n-triples}, ideata da
    \emph{Tim Berners Lee}. Estende la sintassi
    \emph{n-triples} aggiungendovi scorciatoie. Innanzitutto permette l'uso dei
    \emph{qnames} e per dichiararli si ricorre alla seguente sintassi:
    \begin{ntriples}
@prefix unifi:<http://www.unifi.it/rdf/unifi\#>
    \end{ntriples}
    poi, una volta dichiarati, possono essere usati nella scrittura
    delle triple con la solita sintassi:
    \begin{ntriplesblock}
_:X unifi:ha-nome "Mario Rossi" .
_:X unifi:ha-ruolo "Professore" .
_:X unifi:afferisce-a _:Y .
_:Y unifi:ha-nome "sistemi e informatica" .
    \end{ntriplesblock}
	
    Un'altra particolarit\`a \`e che \emph{N3} permette di raggruppare
    le triple che hanno lo stesso soggetto, separando le coppie di
    predicato e di oggetto con il carattere \ntriples{;} nel seguente modo:
    \begin{ntriplesblock}
_:X
      unifi:ha-nome "Mario Rossi" ;
      unifi:ha-ruolo "Professore" ;
      unifi:afferisce-a _:Y ;
_:Y unifi:ha-nome "sistemi e informatica" .
    \end{ntriplesblock}
	
    Inoltre \emph{N3} permette, allo stesso modo, di raggruppare triple che hanno il
    solito soggetto e il solito predicato, separando i soggetti con
    delle \ntriples{,} nel seguente modo:
    \begin{ntriplesblock}
_:X unifi:ha-figlio
      "Mario Rossi" ,
      "Paolo Bianchi" .
    \end{ntriplesblock}
	
    \emph{N3} mette a disposizione anche delle scorciatoie, per esempio
    \ntriples{rdf:type} pu\`o semplicemente essere indicata con
    \ntriples{a}.
	
 \subsubsection{RDF/XML}
 \`E la notazione pi\`u usata in quanto sfrutta il ben
    noto e supportato \emph{XML} per la rappresentazione delle
    triple. Per meglio comprendere la sintassi \`e utile partire dalla
    rappresentazione a grafo dell'\emph{RDF} come gi\`a visto prima,
    usiamo a tal proposito il grafo della figura \ref{fig:grafoxmlrdf}.
    \begin{tgraph}[Grafo di esempio per
        \emph{RDF/XML}][fig:grafoxmlrdf]
      \tgraphuri{mario}{0.5,0.8}{prof:mario-rossi}
      \tgraphuri{bnode}{0.25,0.4}{ }
      \tgraphlet{nome}{0.75,0.4}{``Mario Rossi''}
      \tgraphuri{tipo}{0,0}{unifi:articolo}
      \tgraphlet{titolo}{0.5,0}{``Una pubblicazione''}
      \tgrapharcl{mario}{bnode}{unifi:ha-pubblicazione}
      \tgrapharcr{mario}{nome}{unifi:ha-nome}
      \tgrapharcl{bnode}{tipo}{unifi:tipo}
      \tgrapharcr{bnode}{titolo}{unifi:titolo}
    \end{tgraph}
    Dove il nodo vuoto \`e un \emph{bnode}\footnote{Senza nessuna
      label in quanto non ci interessa.}, e i prefissi corrispondono
    a:
    \begin{tabular}{rcl}
      prof & = & ``http://www.unifi.it/rdf/prof\#''\\
      unifi & = & ``http://www.unifi.it/rdf/namespace\#''\\
    \end{tabular}

    Seguendo le \emph{syntax specification}
    dell'\emph{RDF/XML}\footnote{Visibili all'indirizzo
      \link{http://www.w3.org/TR/rdf-syntax-grammar/}.} vengono trascritti
    nell'\emph{XML} i nodi \emph{non letterali} del grafo come elementi
    \xml{rdf:Description} e gli archi come elementi con il loro nome; 
    convenzionalmente si indicano i primi come \emph{elementi nodo} e i secondi
    come \emph{elementi propriet\`a}. \`E necessario trascriverli facendo
    in modo che ogni \emph{elemento propriet\`a} sia figlio
    dell'\emph{elemento nodo} che rappresenta l'origine dell'arco e
    padre dell'\emph{elemento nodo} che rappresenta la
    destinazione dell'arco. 
	
	Se l'arco finisce in un \emph{letterale} allora
    l'\emph{elemento propriet\`a} che rappresenta l'arco non \`e padre
    di un elemento, ma contiene il testo del \emph{letterale}. Gli \emph{uri} dei
    nodi, quando sono presenti\footnote{Ossia quando non sono bnodes.}, sono
    indicati con un attributo di tipo \xml{rdf:about} dentro
    l'\emph{elemento nodo} che li rappresenta. 
	
	Gli elementi possono
    essere sia rappresentati sequenzialmente, svincolati gli uni dagli
    altri, che messi in cascata, raggruppandoli gli uni dentro agli
    altri. Poich\`e l'\emph{XML} necessita di una
    radice \`e previsto un elemento \xml{rdf:RDF} da usare a tal
    proposito. Tale elemento \`e opzionale nel caso in cui il documento
    contenga gi\`a una sola radice, comunque un uso di tale elemento
    pu\`o essere anche quello di indicare i \emph{namespaces} che
    servono per i \emph{prefissi} dei \emph{qnames} (ed anche il
    \emph{namespace} \xml{rdf}).
    
    Il grafo della figura \ref{fig:grafoxmlrdf}
    pu\`o essere serializzato nel seguente modo:
    \begin{xmlblock}
<?xml version="1.0"?>
<rdf:RDF
    xmlns:rdf=
      "http://www.w3.org/1999/02/22-rdf-syntax-ns#" 
    xmlns:unifi=
      "http://www.unifi.it/rdf/namespace#">
  <rdf:Description 
      rdf:about=
        "http://www.unifi.it/rdf/prof#mario-rossi">
    <unifi:ha-pubblicazione>
      <rdf:Description>
        <unifi:tipo>
          <rdf:Description 
            rdf:about=
              "http://www.unifi.it/rdf/namespace#articolo"
            />
        </unifi:tipo>
        <unifi:titolo>Una pubblicazione</unifi:titolo>
      </rdf:Description>
    </unifi:ha-pubblicazione>
    <unifi:ha-nome>Mario Rossi</unifi:ha-nome>
  </rdf:Description>
</rdf:RDF>
    \end{xmlblock}
    Si nota che non pu\`o essere usato il \emph{namespace} per
    risolvere gli uri degli \emph{elementi nodo}. Nell'esempio
    il \emph{bnode} non ha nessuna etichetta; nel caso in cui fosse
    necessario attribuirne una, si usa
    l'attributo \xml{rdf:nodeID} del \emph{nodo elemento} che
    rappresenta il \emph{bnode}.

    La sintassi \emph{RDF/XML} prevede numerose funzionalit\`a e
    scorciatoie, ne enunciamo alcune:
    \begin{itemize}
      \item un \emph{elemento nodo} che non ha nessun
        figlio pu\`o essere inglobato nell'\emph{elemento propriet\`a}
        padre tramite l'attributo \xml{rdf:resource};
      \item se un \emph{elemento propriet\`a} ha solo un figlio che
        \`e un \emph{elemento nodo letterale} allora entrambi possono
        essere inglobati nell'\emph{elemento nodo} padre, usando il
        nome dell'\emph{elemento propriet\`a} come nome di attributo e
        il \emph{letterale} come valore di tale attributo;
      \item i \emph{bnodes} senza id possono essere omessi usando
        l'attributo \xml{rdf:parseType="Resource"} che rende
        l'\emph{elemento propriet\`a} che era padre del \emph{bnode}
        anche un \emph{elemento nodo} che quindi prende i figli che
        erano del \emph{bnode}.
    \end{itemize}
    
    Usando tali scorciatoie \`e possibile scrivere il codice in modo pi\`u
    compatto:
    \begin{xmlblock}
<?xml version="1.0"?>
<rdf:RDF 
    xmlns:rdf=
      "http://www.w3.org/1999/02/22-rdf-syntax-ns#" 
    xmlns:unifi=
      "http://www.unifi.it/rdf/namespace#">
  <rdf:Description 
      rdf:about=
        "http://www.unifi.it/rdf/prof#mario-rossi"
      unifi:titolo="Una pubblicazione">
    <unifi:ha-pubblicazione rdf:parseType="Resource">
      <unifi:tipo 
        rdf:resource=
          "http://www.unifi.it/rdf/namespace#articolo"
        />
    </unifi:ha-pubblicazione>
    <unifi:ha-nome>Mario Rossi</unifi:ha-nome>
  </rdf:Description>
</rdf:RDF>
    \end{xmlblock}

    Per ulteriori informazioni sulla sintassi si rimanda al link:\\
    \link{http://www.w3.org/TR/rdf-syntax-grammar/}.

\subsection{Reificazione}\label{sec:reificazione}
La \emph{reificazione} nasce dalla impossibilit\`a di esprimere certi
concetti con il sistema di triple visto precedentemente, ad esempio,
\`e semplice modellizzare il concetto: <<Mario Rossi ha pubblicato
``Una pubblicazione''>>, ma asserire che <<Paolo Verdi ha fatto da
relatore a tale pubblicazione>> \`e pi\`u complesso. Il problema \`e
che sarebbe necessario poter fare delle \emph{asserzioni} sopra
un'altra \emph{asserzione}, la \emph{reificazione} implementa proprio
questa possibilit\`a. 

L'\emph{RDF} mette a disposizione un vocabolario
per implementare la \emph{reificazione}.
\begin{triples}[Tripla di esempio][tab:reifi1]
  \triple{prof:mario-rossi}{unifi:pubblica}{``Una Pubblicazione''}
\end{triples}
\begin{triples}[Esempio di reificazione][tab:reifi2]
  \triple{pubb:pubb123}{rdf:type}{rdf:Statement}
  \triple{pubb:pubb123}{rdf:subject}{prof:mario-rossi}
  \triple{pubb:pubb123}{rdf:predicate}{unifi:pubblica}
  \triple{pubb:pubb123}{rdf:object}{``Una Pubblicazione''}
\end{triples}
\begin{triples}[Esempio di asserzione su una asserzione][tab:reifi3]
  \triple{prof:paolo-verdi}{unifi:relatore}{pubb:pubb123}
\end{triples}
\`E possibile vedere nella tabella \ref{tab:reifi1} un esempio di una tripla
da \emph{reificare}, in \ref{tab:reifi2} la stessa tripla
\emph{reificata} con l'\emph{uri} \xml{pubb:pubb123}, e in
\ref{tab:reifi3} un esempio di come un'altra tripla si pu\`o riferire
a questa. 

Gli elementi del vocabolario che servono per
il procedimento sono:
\begin{description}
  \item[rdf:type e rdf:Statement] indicano che il soggetto
    \xml{pubb:pubb123} \`e di tipo \emph{Statement} quindi che
    rappresenta una tripla.
  \item[rdf:subject] indica il \emph{soggetto} della tripla
    reificata.
  \item[rdf:predicate] indica il \emph{predicato} della tripla
    reificata.
  \item[rdf:object] indica l'\emph{oggetto} della tripla
    reificata.
\end{description}

\subsection{Tipi di dato}\label{sec:tipididato}
Formalmente un \emph{tipo di dato} \`e composto da tre parti:
\begin{itemize}
  \item uno spazio lessicale, un set non vuoto di stringhe di caratteri.
  \item uno spazio di valori, un set non vuoto.
  \item un mapping dallo \emph{spazio lessicale} allo \emph{spazio di
    valori}.
\end{itemize}
\emph{RDF} supporta un \emph{subset} dei tipi di dato
definiti in \emph{XML Schema}\footnote{Vedere sezione
  \ref{sec:xmlschema}.}, inoltre fornisce un singolo tipo di dato:
\begin{xmlblock}
rdf:XMLLiteral
\end{xmlblock}
definito come\footnote{Vedere \link{http://www.w3.org/TR/rdf-concepts/\#section-XMLLiteral}.}:
\begin{description}
  \item[spazio lessicale:] l'insieme delle stringhe che sono XML ben
    bilanciato, in forma canonica esclusiva\footnote{Vedere
      \link{http://www.w3.org/TR/xml-exc-c14n/}.}, per il quale
    l'\emph{embedding} tra uno \emph{start tag} ed un \emph{end tag} porta
    alla formazione di un documento conforme con i \emph{namespace};
  \item[spazio di valori:] un set di entit\`a che \`e disgiunto dallo
    spazio lessicale, disgiunto da ogni spazio di valori di ogni tipo
    di dato dell'\emph{XML Schema}, disgiunto dal set delle stringhe
    delle stringhe dei caratteri \emph{unicode} e con una
    corrispondenza uno ad uno con lo spazio lessicale;
  \item[mapping:] un mapping uno ad uno tra i due spazi, iniettivo e
    suriettivo.
\end{description}

Per una definizione formale dei tipi di dato in \emph{RDF} e per
sapere qual \`e il subset preciso dei tipi di dato di \emph{XML
  Schema} si rimanda al link
\link{http://www.w3.org/TR/rdf-mt/\#DTYPEINTERP}.

\section{RDF Schema (RDFS)}
L'\emph{RDF Schema} \`e una estensione semantica dell'\emph{RDF}, esso
fornisce un vocabolario che permette di definire e descrivere classi,
propriet\`a e altre risorse, in modo simile a come fanno linguaggi ad
oggetti come il \emph{Java}. Viene usata la notazione
\xml{prefix:suffix} dove \xml{prefix} \`e uno dei prefissi gi\`a visti
nella sezione \ref{sec:rdf}:
\begin{itemize}
  \item \xml{rdf} = \xml{http://www.w3.org/1999/02/22-rdf-syntax-ns\#};
  \item \xml{rdfs} = \xml{http://www.w3.org/2000/01/rdf-schema\#}.
\end{itemize}
Viene usato anche il namespace \xml{rdf} nonostante si tratti di
\emph{RDFS} perch\'e, come gi\`a detto precedentemente, non
esiste una divisione netta tra i livelli che compongono il \emph{web
  semantico}, queste sono tecnologie nate per gradi e sviluppate l'una
sull'altra nel tempo. 

Ad esempio \emph{RDF} fornisce gi\`a il supporto
per i contenitori come \xml{rdf:bag}, nel sistema di classi di \emph{RDFS} quest'ultimo viene
formalizzato  rendendolo
sottoclasse di \xml{rdfs:Container}. I
concetti di classi e sottoclassi in \emph{RDF} verranno spiegati di seguito. La \emph{reificazione} vista prima nella sezione
\ref{sec:reificazione} viene in questa sezione formalizzata.

Buona parte delle informazioni contenute in questa sezione possono
essere reperite nella raccomandazione del \emph{W3C} dove \`e
descritto il vocabolario \emph{RDFS}:
\link{http://www.w3.org/TR/rdf-schema/}.

Il vocabolario principale dell'\emph{RDFS} pu\`o essere diviso tra:
\begin{itemize}
  \item classi;
  \item propriet\`a.
\end{itemize}
Le prime identificano dei gruppi di risorse, le seconde specificano
relazioni tra il soggetto e l'oggetto. Oltre a questi due vocabolari
esistono anche altri vocabolari, alcuni estendono e formalizzano
concetti appartenenti a \emph{RDF}.

\subsection{Classi}
Le risorse possono essere divise in \emph{classi}, i membri di tali
classi sono chiamate \emph{istanze}, le classi sono a sua volta delle
\emph{risorse}.

La propriet\`a \xml{rdf:type} \`e usata per indicare che una
risorsa \`e una istanza di una classe. L'\emph{insieme
delle istanze} di una certa classe \`e chiamato \emph{estensione
  della classe}. una classe ha ovviamente una sola estensione, per\`o
due classi diverse possono avere la stessa estensione, ossia, a
differenza di altri linguaggi ad oggetti come il Java, una risorsa
pu\`o essere istanza di pi\`u classi. Una classe pu\`o essere una
istanza di se stessa. Il gruppo delle risorse che sono classi \`e a
sua volta una classe, precisamente la classe \xml{rdfs:Class}.

Si possono definire delle \emph{sottoclassi} tramite la propriet\`a
\xml{rdfs:subClassOf}: se una classe ``B'' \`e sottoclasse di ``A''
allora ogni istanza di ``B'' \`e anche una istanza di ``A''. Il
termine \emph{superclasse} \`e usato come inverso di
\emph{sottoclasse}: se ``B'' \`e una sottoclasse di ``A'' allora ``A''
\`e una superclasse di ``B''.

Le classi principali sono:
\begin{description}
  \item[rdfs:Resource] \`e la classe di qualsiasi risorsa, tutte le
    altre classi sono sottoclassi di questa classe.

    \xml{rdfs:Resource} \`e una istanza di \xml{rdfs:Class};
  \item[rdfs:Class] indica le risorse che sono
    classi, ossia tutte le istanze di \xml{rdfs:Class} rappresentano delle
    classi.

    \xml{rdfs:Class} \`e una istanza di \xml{rdfs:Class} e
    sottoclasse di \xml{rdfs:Resource};
  \item[rdfs:Literal] \`e la classe dei valori letterali come stringhe
    e interi, possono essere \emph{tipati} o \emph{plain}. Quelli
    tipati sono istanze di una classe \emph{datatype}, non esiste la
    classe di quelli plain.
    
    \xml{rdfs:Literal} \`e una istanza di \xml{rdfs:Class} e
    sottoclasse di \xml{rdfs:Resource};
  \item[rdfs:Datatype] \`e la classe dei tipi di dato, ogni istanza di
    \xml{rdfs:Datatype} \emph{\`e una sottoclasse di}
    \xml{rdfs:Literal},

    \xml{rdfs:Datatype} \`e una istanza e una sottoclasse di
    \xml{rdfs:Class};
  \item[rdf:XMLLiteral] \`e la classe dei letterali nella
    serializzazione \emph{RDF/XML}.

    \xml{rdf:XMLLiteral} \`e una istanza di \xml{rdfs:Datatype} e una
    sottoclasse di \xml{rdfs:Literal};
  \item[rdf:Property] \`e la classe delle propriet\`a, tutte le
    propriet\`a sono istanze di questa.

    \xml{rdf:Property} \`e una istanza di \xml{rdfs:Class}.
\end{description}

\subsection{Propriet\`a}
Le \emph{propriet\`a} sono relazioni che legano una \emph{risorsa
  oggetto} ad una \emph{risorsa soggetto}, di fatto le propriet\`a
\emph{sono i predicati} delle triple viste nella sezione \ref{sec:rdf}.
 Ci\`o che implementa l'\emph{RDFS} \`e la possibilit\`a di
definire \emph{Sottopropriet\`a} tramite la propriet\`a
\xml{rdfs:subPropertyOf}, e di definire un \emph{dominio} e
un'\emph{immagine} alla propriet\`a tramite le propriet\`a
\xml{rdfs:domain} ed \xml{rdfs:range}.

Se una propriet\`a ``B'' \`e
\emph{sottopropriet\`a} di una propriet\`a ``A'', allora ogni coppia
di \emph{risorse} che viene posta in relazione da ``B'' \`e anche in
relazione tramite ``A''. Esiste anche il termine \emph{superpropriet\`a}
per riferirsi all'opposto di sottopropriet\`a: se ``B'' \`e
sottopropriet\`a di ``A'', allora ``A'' \`e superpropriet\`a di
``B''.

Per quanto riguarda il \emph{dominio} e l'\emph{immagine} di una
propriet\`a, essi
servono ad indicare tra quali istanze di classi \`e possibile
applicare tale propriet\`a. Il \emph{dominio} definisce la classe del
soggetto della propriet\`a, l'\emph{immagine} quella dell'oggetto. La
terminologia \`e uguale a quella matematica perch\'e
il loro significato \`e simile a quello delle funzioni: cos\`i
come una funzione mette in relazione l'insieme dei punti del dominio
con quelli del codominio, una propriet\`a mette in relazione l'insieme
delle risorse del dominio con quelle dell'immagine.

Le propriet\`a principali sono:
\begin{description}
  \item[rdfs:domain] definisce il dominio di una propriet\`a, la tripla
    \singletriple{P}{rdfs:domain}{C}
    indica che \xml{P} \`e una istanza di \xml{rdf:Property}, \xml{C}
    \`e una istanza di \xml{rdfs:Class} e le risorse denotate dai
    \emph{soggetti} delle triple il cui \emph{predicato} \`e \xml{P}
    sono istanze della classe \xml{C}. Se una propriet\`a ha pi\`u domin\^i definiti,
    allora i \emph{soggetti} delle triple il cui \emph{predicato} \`e
    \xml{P} sono istanze \emph{di tutte le classi} \xml{C} definite da
    tali domin\^i.

    \xml{rdfs:domain} \`e una istanza di \xml{rdf:Property}, il
    dominio di \xml{rdfs:domain} \`e \xml{rdf:Property}, l'immagine
    di \xml{rdfs:domain} \`e \xml{rdfs:Class};
  \item[rdfs:range] definisce l'immagine di una propriet\`a, la tripla
    \singletriple{P}{rdfs:range}{C}
    indica che \xml{P} \`e una istanza di \xml{rdf:Property}, \xml{C}
    \`e una istanza di \xml{rdfs:Class} e le risorse denotate dagli
    \emph{oggetti} delle triple il cui \emph{predicato} \`e \xml{P} sono istanze
    della classe \xml{C}. Se una propriet\`a ha pi\`u immagini definite,
    allora gli \emph{oggetti} delle triple il cui \emph{predicato} \`e
    \xml{P} sono istanze \emph{di tutte le classi} \xml{C} definite da
    tali immagini.

    \xml{rdfs:range} \`e una istanza di \xml{rdf:Property}, il
    dominio di \xml{rdfs:range} \`e \xml{rdf:Property}, l'immagine
    di \xml{rdfs:range} \`e \xml{rdfs:Class};
  \item[rdf:type] indica che una risorsa \`e una istanza di una certa
    classe. La tripla
    \singletriple{R}{rdf:type}{C}
    indica che \xml{C} \`e una istanza di \xml{rdfs:Class} e che
    \xml{R} \`e una istanza di \xml{C}.

    Il dominio di \xml{rdf:type} \`e \xml{rdfs:Resource}, l'immagine
    \`e \xml{rdfs:Class};
  \item[rdfs:subPropertyOf] indica che una propriet\`a \`e
    sottopropriet\`a di un'altra. La tripla
    \singletriple{P1}{rdfs:subPropertyOf}{P2}
    indica che \xml{P1} e \xml{P2} sono istanze di \xml{rdf:Property}
    e che se
    \singletriple{S}{P1}{O}
    allora vale anche
    \singletriple{S}{P2}{O}

    \xml{rdfs:subPropertyOf} \`e una istanza di \emph{rdf:Property},
    il dominio e l'immagine di \xml{rdfs:subPropertyOf} sono
    \xml{rdf:Property};
  \item[rdfs:subClassOf] \`e usata per indicare che una classe \`e
    sottoclasse di un'altra. La tripla
    \singletriple{C1}{rdfs:subClassOf}{C2}
    indica che \xml{C1} e \xml{C2} sono istanze di \xml{rdfs:Class} e
    che se
    \singletriple{I}{rdf:type}{C1}
    allora
    \singletriple{I}{rdf:type}{C2}

    \xml{rdfs:subClassOf} \`e una istanza di \xml{rdf:Property}, il
    dominio e l'immagine di \xml{rdfs:subClassOf} sono
    \xml{rdfs:Class};
  \item[rdfs:label] pu\`o essere usato per fornire una forma leggibile
    di un nome
    di una certa risorsa. La tripla
    \singletriple{R}{rdfs:label}{L}
    indica che \xml{R} \`e una istanza di \xml{rdfs:Resource} e che
    \xml{L} \`e una istanza di \xml{rdfs:Literal} che rappresenta una
    etichetta leggibile per \xml{R}.

    \xml{rdfs:label} \`e una istanza di \xml{rdf:Property}, il dominio
    \`e \xml{rdfs:Resource}, l'immagine \`e \xml{rdfs:Literal};
  \item[rdfs:comment] pu\`o essere usato per fornire una descrizione e
    documentazione
    in forma leggibile di una certa risorsa. La tripla
    \singletriple{R}{rdfs:comment}{L}
    indica che \xml{R} \`e una istanza di \xml{rdfs:Resource} e che
    \xml{L} \`e una istanza di \xml{rdfs:Literal} che rappresenta una
    descrizione leggibile per \xml{R}.

    \xml{rdfs:comment} \`e una istanza di \xml{rdf:Property}, il
    dominio \`e \xml{rdfs:Resource}, l'immagine \`e \xml{rdfs:Literal}.
\end{description}

\subsection{Altri vocabolari}
In questa sezione vengono analizzate altre classi e propriet\`a come i
contenitori e formalizzeremo nella gerarchia dell'\emph{RDFS} concetti
gi\`a visti in precedenza come la reificazione.

\subsubsection{Contenitori}
I contenitori sono risorse che rappresentano collezioni, esistono tre
classi di contenitori: \xml{rdf:Bag}, \xml{rdf:Seq},
\xml{rdf:Alt}. Sintatticamente e nel modo d'uso non ci sono
differenze tra le tre classi, la differenza sta nel significato
semantico che gli utenti danno a tali strutture. Per convenzione
\xml{rdf:Bag} \`e usata come lista non ordinata,
\xml{rdf:Seq} come lista ordinata nella quale l'ordine \`e
significativo e, infine, \xml{rdf:Alt} indica che il processo tipico
presuppone una scelta di un elemento tra i suoi membri.

Le risorse del vocabolario sono:
\begin{description}
  \item[rdfs:Container] \`e la super classe di tutti i contenitori,
    \xml{rdf:Bag}, \xml{rdf:Seq} e \xml{rdf:Alt};
  \item[rdfs:member] \`e la super propriet\`a delle propriet\`a di
    appartenenza di un contenitore.

    \xml{rdfs:member} \`e una istanza di \xml{rdf:Property}. Il
    dominio e l'immagine di \xml{rdfs:member} sono
    \xml{rdfs:Resource};
  \item[rdfs:ContainerMembershipProperty] \`e la classe delle
    propriet\`a di appartenenza di un contenitore.

    \xml{rdfs:ContainerMembershipProperty} \`e una istanza di
    \xml{rdfs:Class} e una sottoclasse di \xml{rdf:Property};
  \item[rdf:\_nnn], dove \xml{nnn} \`e un intero \emph{maggiore di
    zero} senza zeri in testa, \`e la propriet\`a che indica
    l'appartenenza di una risorsa ad un contenitore. La tripla
    \singletriple{C}{rdf:\_nnn}{O}
    indica che la risorsa \xml{O} \`e membro del contenitore \xml{C}.

    \xml{rdf:\_nnn} \`e una istanza di
    \xml{rdfs:ContainerMembershipProperty} e una sottopropriet\`a di
    \xml{rdfs:member}. Notare che, poich\'e il dominio di
    \xml{rdfs:member} \`e \xml{rdfs:Resource} e non
    \xml{rdfs:Container}, allora \xml{rdf:\_nnn} pu\`o essere
    applicata anche a risorse che non sono contenitori;
  \item[rdf:Bag] \`e la classe dei contenitori non ordinati.
    L'ordinamento numerico, dato dal numero \xml{nnn} delle propriet\`a
    di appartenenza \xml{rdf:\_nnn} applicate ad esso, \`e da
    considerarsi privo di significato.

    \xml{rdf:Bag} \`e una sottoclasse di \xml{rdf:Container};
  \item[rdf:Seq] \`e la classe dei contenitori ordinati,
    l'ordinamento numerico, dato dal numero \xml{nnn} delle propriet\`a
    di appartenenza \xml{rdf:\_nnn} applicate ad esso, \`e da
    considerarsi significativo.

    \xml{rdf:Seq} \`e una sottoclasse di \xml{rdf:Container};
  \item[rdf:Alt] \`e la classe dei contenitori che prevedono una
    selezione tra i suoi membri. Il primo membro, con valore numerico
    \xml{nnn} minore delle propriet\`a
    di appartenenza \xml{rdf:\_nnn}, \`e da
    considerarsi come il valore selezionato di default.

    \xml{rdf:Alt} \`e una sottoclasse di \xml{rdf:Container}.
\end{description}

\subsubsection{Collezioni}
Le collezioni sono liste di elementi, anche vuote,
rappresentate in modo simile a come vengono implementate le liste nei
linguaggi funzionali. Nella figura \ref{fig:collection} \`e mostrato
un uso tipico di queste liste.
\begin{tgraph}[Esempio di utilizzo delle Collection][fig:collection]
\tgraphuri{base}{0,1}{dip:sisinfo}
\tgraphuri{primo}{0.25,0.75}{}
\tgraphuri{primof}{1,0.75}{prof:mario-rossi}
\tgraphuri{secondo}{0.25,0.5}{}
\tgraphuri{secondof}{1,0.5}{prof:franco-neri}
\tgraphuri{terzo}{0.25,0.25}{}
\tgraphuri{terzof}{1,0.25}{prof:paolo-verdi}
\tgraphuri{nil}{0.25,0}{rdf:nil}
\tgrapharcl{base}{primo}{unifi:professori}
\tgrapharcl{primo}{primof}{rdf:first}
\tgrapharcl{primo}{secondo}{rdf:rest}
\tgrapharcl{secondo}{secondof}{rdf:first}
\tgrapharcl{secondo}{terzo}{rdf:rest}
\tgrapharcl{terzo}{terzof}{rdf:first}
\tgrapharcl{terzo}{nil}{rdf:rest}
\end{tgraph}

Ogni \emph{bnode} \`e istanza di \xml{rdf:List} cos\`i come
\xml{rdf:nil} che rappresenta una lista vuota. \xml{rdf:firs} indica
il primo elemento della lista in questione, \xml{rdf:rest} il resto
della lista senza il primo elemento.

Nel dettaglio il vocabolario \`e composto da:
\begin{description}
  \item[rdf:List] \`e la classe delle liste, le sue istanze
    rappresentano liste.

    \xml{rdf:List} \`e una istanza di \xml{rdfs:Class};
  \item[rdf:first] indica il primo elemento di una lista. La tripla:
    \singletriple{L}{rdf:first}{O}
    indica che \xml{L} \`e una istanza di \xml{rdf:List}, \xml{O} una
    istanza di \xml{rdfs:Resource} e che \xml{O} \`e il primo
    elemento della lista \xml{L}.

    \xml{rdf:first} \`e una istanza di \xml{rdf:Property}, il suo
    dominio \`e \xml{rdf:List} e la sua immagine \xml{rdfs:Resource};
  \item[rdf:rest] indica gli elementi di una lista escluso il
    primo. La tripla:
    \singletriple{L}{rdf:rest}{O}
    indica che \xml{L} \`e una istanza di \xml{rdf:List}, \xml{O} una
    istanza di \xml{rdfs:List} e che \xml{O} \`e una lista uguale ad
    \xml{L} togliendo il primo elemento.

    \xml{rdf:rest} \`e una istanza di \xml{rdf:Property}, il suo
    dominio \`e \xml{rdf:List} e la sua immagine \xml{rdfs:List};
  \item[rdf:nil] indica una lista vuota.

    \xml{rdf:nil} \`e una istanza di \xml{rdf:List}.
\end{description}

\subsubsection{Reificazione}
Nella sezione \ref{sec:reificazione} \`e gi\`a stato introdotto il concetto
di reificazione e la sua implementazione. \`E gia state descritte la classe
\xml{rdf:Statement} e le propriet\`a \xml{rdf:subject},
\xml{rdf:predicate} e \xml{rdf:object}; ma non erano ancora noti i
concetti di \emph{classe} e \emph{propriet\`a}. 

Viene descritto come formalizzare tali classe e propriet\`a in
\emph{RDFS}:
\begin{description}
  \item[rdf:Statement] \`e la classe delle sentenze, le sue istanze
    rappresentano delle triple, ognuna con un soggetto indicato da
    \xml{rdf:subject}, un predicato indicato da \xml{rdf:predicate} e
    un oggetto indicato da \xml{rdf:object}.

    \xml{rdf:Statement} \`e una istanza di \xml{rdfs:Class};
  \item[rdf:subject] \`e la propriet\`a che indica il soggetto di una
    tripla (una istanza di \xml{rdf:Sentence}). La tripla:
    \singletriple{S}{rdf:subject}{R}
    indica che \xml{S} \`e una istanza di \xml{rdf:Sentence}, \xml{R}
    di \xml{rdfs:Resource} e che \xml{R} \`e il soggetto di \xml{S}.

    \xml{rdf:subject} \`e una istanza di \xml{xml:Property}, il suo
    dominio \`e \xml{rdf:Statement}, l'immagine \xml{rdf:Resource};
  \item[rdf:predicate] \`e la propriet\`a che indica il predicato di una
    tripla (una istanza di \xml{rdf:Sentence}). La tripla:
    \singletriple{S}{rdf:predicate}{R}
    indica che \xml{S} \`e una istanza di \xml{rdf:Sentence}, \xml{R}
    di \xml{rdfs:Resource} e che \xml{R} \`e il predicato di \xml{S}.

    \xml{rdf:predicate} \`e una istanza di \xml{xml:Property}, il suo
    dominio \`e \xml{rdf:Statement}, l'immagine \xml{rdf:Resource};
  \item[rdf:object] \`e la propriet\`a che indica l'oggetto di una
    tripla (una istanza di \xml{rdf:Sentence}). La tripla:
    \singletriple{S}{rdf:object}{R}
    indica che \xml{S} \`e una istanza di \xml{rdf:Sentence}, \xml{R}
    di \xml{rdfs:Resource} e che \xml{R} \`e l'oggetto di \xml{S}.

    \xml{rdf:object} \`e una istanza di \xml{xml:Property}, il suo
    dominio \`e \xml{rdf:Statement}, l'immagine \xml{rdf:Resource}.
\end{description}

\subsubsection{Propriet\`a di utility}
Le seguenti tre propriet\`a sono definite per fornire pi\`u
informazioni riguardo una risorsa:
\begin{description}
  \item[rdfs:seeAlso] serve per indicare una risorsa che fornisce
    ulteriori informazioni su un'altra risorsa. La tripla:
    \singletriple{S}{rdfs:seeAlso}{O}
    indica che \xml{S} e \xml{O} sono istanze di \xml{rdfs:Resource}
    e che \xml{O} pu\`o fornire informazioni addizionali su \xml{S}.

    \xml{rdfs:seeAlso} \`e una istanza di \xml{rdf:Property}, il suo
    dominio \`e \xml{rdfs:Resource}, la sua immagine \xml{rdfs:Resource};
  \item[rdfs:isDefinedBy] serve per indicare una risorsa che definisce
    un'altra risorsa. La tripla:
    \singletriple{S}{rdfs:isDefinedBy}{O}
    indica che \xml{S} e \xml{O} sono istanze di \xml{rdfs:Resource}
    e che \xml{O} definisce \xml{S}.

    \xml{rdfs:isDefinedBy} \`e una istanza di \xml{rdf:Property}, il suo
    dominio \`e \xml{rdfs:Resource}, la sua immagine \xml{rdfs:Resource};
  \item[rdf:value] non ha un significato formale, ma di solito viene
    usata per fornire il valore principale di un dato strutturato. La
    tripla: 
    \singletriple{S}{rdf:value}{O}
    indica che \xml{S} e \xml{O} sono istanze di \xml{rdfs:Resource}
    e che \xml{O} \`e il valore principale di \xml{S}.

    \xml{rdf:value} \`e una istanza di \xml{rdf:Property}, il suo
    dominio \`e \xml{rdfs:Resource}, la sua immagine \xml{rdfs:Resource}.
\end{description}

\section{Web Ontology Language (OWL)}
\emph{Web Ontology Language} estende \emph{RDFS} fornendo un
vocabolario addizionale per descrivere altre propriet\`a e classi. Le
strutture descritte con \emph{OWL} prendono il nome di
\emph{ontologie}. \emph{OWL} fornisce un suo \emph{namespace}:
\link{http://www.w3.org/2002/07/owl\#}, e solitamente viene usato il
prefisso \xml{owl}.

Le ontologie includono descrizioni di classi, di propriet\`a e le loro
istanze. La principale potenzialit\`a nel descrivere ontologie
rispetto a schemi \emph{RDFS} sta nel fatto che le ontologie sono
strutture logiche dalle quali \`e possibile estrarre conoscenza. Il
mezzo con cui viene estratta questa conoscenza da un ontologia \`e il
\emph{reasoner} che \`e un programma che fornisce supporto
generico applicabile ad ogni ontologia \emph{OWL}.

 \emph{OWL} \`e
composto da tre sottolinguaggi con livelli di espressivit\`a logica
crescenti:
\begin{enumerate}
\item \emph{OWL Lite} fornisce supporto alle necessit\`a di una
  classificazione gerarchica con vincoli semplici;
\item \emph{OWL DL} fornisce la massima espressivit\`a senza perdita
  di computabilit\`a e decidibilit\`a dei \emph{reasoning};
\item \emph{OWL Full} fornisce la massima espressivit\`a senza
  garanzie computazionali.
\end{enumerate}
\emph{OWL Full} implementa una logica del primo ordine, mentre gli
altri due implementano delle logiche descrittive pi\`u restrittive.

Una analisi avanzata di \emph{OWL} va oltre gli scopi di questo testo,
per uno studio approfondito si rimanda a \cite{owl}.

\section{Simple Knowledge Organization System (SKOS)}\label{sec:skos}
\emph{SKOS} \`e un vocabolario \emph{RDF} che permette la descrizione
di semplici sistemi di organizzazione della conoscenza come tesauri,
tassonomie e schemi di classificazione. Rappresenta un anello tra i
linguaggi formali, come \emph{OWL}, e le attuali implementazioni
informali delle applicazioni web.

L'elemento fondamentale del vocabolario \`e il \emph{concept},
\emph{SKOS} mette a disposizione la classe \xml{skos:Concept} per
indicare che una certa risorsa \`e un \emph{concept}. Con la tripla
\singletriple{ex:esempioConcept}{rdf:type}{skos:Concept}
si indica che \xml{ex:esemioConcept} \`e un \emph{concept}.

\xml{skos:prefLabel}, \xml{skos:altLabel}, \xml{skos:hiddenLabel}
servono per indicare l'espressione in linguaggio naturale con cui ci
si riferisce ad un \emph{concept}, sono tutte sottopropriet\`a di
\xml{rdfs:label}. \xml{skos:prefLabel} indica l'etichetta principale,
mentre con \xml{skos:altLabel} \`e possibile definire sinonimi. Con
\xml{skos:hiddenLabel} \`e possibile indicare etichette che non devono
essere visibili.

\subsection{Relazioni semantiche}
Con \emph{SKOS} \`e possibile definire le seguenti relazioni
semantiche tra \emph{concept}:
\begin{itemize}
\item \xml{skos:broader} e \xml{skos:narrower} definiscono relazioni
  gerarchiche nelle quali un \emph{concept} \`e una specificazione di
  un altro, oppure una sua parte. \xml{skos:broader} e
  \xml{skos:narrower} sono una l'inversa dell'altra;
\item \xml{skos:relate} indica che vi \`e una associazione non
  gerarchica tra due \emph{concept}.
\end{itemize}
La figura \ref{fig:esempioNarrower} indica il modo di utilizzo di
\xml{skos:narrower} e \xml{skos:broader}, nell'esempio
\xml{animali:gatto} \`e una specializzazione di
\xml{animali:mammifero}.
\begin{tgraph}[Esempio di utilizzo di \xml{skos:narrower} e \xml{skos:broader}.][fig:esempioNarrower]
  \tgraphuri{n1}{0,0}{animali:mammifero}
  \tgraphuri{n2}{1,0}{animali:gatto}
  \tgrapharcl{n1}{n2}{skos:narrower}
  \tgrapharcl{n2}{n1}{skos:broader}
\end{tgraph}

\subsection{Note}
\emph{SKOS} fornisce la propriet\`a \xml{skos:note} e le sue
specializzazioni per indicare descrizioni e documentazioni testuali di un
\emph{concept}. \xml{skos:note} \`e per una documentazione generica,
le sue specializzazioni sono:
\begin{itemize}
\item \xml{skos:scopeNote} per indicazioni su l'uso di un concept e
  sui suoi limiti;
\item \xml{skos:definition} per una spiegazione competa del
  significato di un \emph{concept};
\item \xml{skos:example} per fornire esempi sull'uso di un
  \emph{concept};
\item \xml{skos:historyNote} per descrivere cambi sostanziali nel
  significato e nella forma di un \emph{concept};
\item \xml{skos:editorialNote} per fornire informazioni indirizzate
  all'amministrazione, come indicazioni sul lavoro ancora da fare;
\item \xml{skos:changeNote} per documentare i cambiamenti di un
  \emph{concept} a scopo amministrativo.
\end{itemize}

\subsection{\emph{Concept} schema}
I \emph{concept} possono essere usati come entit\`a a se stanti,
oppure possono essere organizzati in vocabolari chiamati schemi, \emph{SKOS} fornisce
la possibilit\`a di rappresentare questi schemi con la classe
\xml{skos:ConceptScheme}. Per indicare che un \emph{concept}
appartiene ad un certo schema si usa la propriet\`a
\xml{skos:inScheme}. Per indicare i punti di accesso ad una gerarchia
di uno schema definita con \xml{skos:narrower} e \xml{skos:broader},
si usa la propriet\`a di un \xml{skos:ConceptScheme}:
\xml{skos:hasTopConcept}.

In pratica con \xml{skos:narrower} e \xml{skos:broader} si definiscono
alberi di \emph{concept}, le radici di questi alberi possono essere
associate ad uno schema con \xml{skos:hasTopConcept}.

\subsection{Mappare \emph{concept} schema}
\`E possibile creare relazioni tra schemi diversi usando le
propriet\`a dei \emph{concept}:
\begin{itemize}
\item \xml{skos:exactMatch};
\item \xml{skos:closeMatch}.
\end{itemize}
Entrambe possono essere usate per indicare che due \emph{concept}
appartenenti a schemi diversi hanno lo stesso significato. La
differenza tra i due \`e che \xml{skos:exactMatch} \`e transitivo,
\xml{skos:closeMatch} no.

Due \emph{concept} di due schemi diversi possono essere anche mappati
con le propriet\`a parallele a quelle per le relazioni semantiche:
\begin{itemize}
\item \xml{skos:broadMatch};
\item \xml{skos:narrowMatch};
\item \xml{skos:relatedMatch}.
\end{itemize}

\subsection{Collezioni di \emph{concept}}
I \emph{concept} possono essere organizzati in collezioni che possono
essere ordinate o no. Queste collezioni vengono indicate con le
classi:
\begin{itemize}
\item \xml{skos:Collection} indica una collezione non ordinata, i
  \emph{concept} membri vengono indicati con la propriet\`a
  \xml{skos:member};
\item \xml{skos:OrderedCollection} indica una collezione non ordinata, i
  \emph{concept} membri vengono indicati con la propriet\`a
  \xml{skos:memberList};
\end{itemize}

\section{\emph{Friend Of A Friend} (\emph{FOAF})}
\emph{FOAF} consiste in un vocabolario per definire una rete sociale
tramite \emph{RDF}, \emph{RDFS} e \emph{OWL}. Con \emph{FOAF} \`e possibile descrivere persone e
relazioni tra di esse. \emph{FOAF} usa il \emph{namespace}
\link{http://xmlns.com/foaf/0.1/} a cui solitamente viene associato il
prefisso \xml{foaf:}.

In questo testo vengono analizzate solo le classi e le propriet\`a
stabili delle specifiche, per un elenco completo riferirsi a
\cite{foaf}.

\emph{FOAF} fornisce le seguenti classi:
\begin{itemize}
\item \xml{foaf:Agent} \`e la superclasse di tutti gli agenti che
  possono essere gruppi, persone o organizzazioni;
\item \xml{foaf:Group} rappresenta una collezione di agenti, ed \`e a
  sua volta un agente, i suoi membri sono definiti con la propriet\`a
  \xml{foaf:member}, \`e una sottoclasse di \xml{foaf:Agent};
\item \xml{foaf:Organization} rappresenta un agente corrispondente ad
  una organizzazione, impresa, etc\dots, \`e una sottoclasse di \xml{foaf:Agent};
\item \xml{foaf:Person} rappresenta una persona, \`e una sottoclasse
  di \xml{foaf:Agent};
\item \xml{foaf:Document} rappresenta un documento che pu\`o essere
  sia elettronico che fisico.
\end{itemize}

\emph{FOAF} fornisce le seguenti propriet\`a:
\begin{itemize}
\item \xml{foaf:homepage} indica la pagina web di qualcosa, una certa
  pagina pu\`o essere associata sono ad una risorsa, il dominio \`e
  \xml{owl:Thing}, l'immagine \xml{foaf:Document};
\item \xml{foaf:isPrimaryTopicOf} mette in relazione qualcosa con un
  documento che tratta principalmente di questa, \`e l'inversa di
  \xml{foaf:primaryTopic}, il dominio \`e
  \xml{owl:Thing}, l'immagine \xml{foaf:Document};
\item \xml{foaf:primaryTopic} mette in relazione un documento con
  qualcosa che \`e il suo argomento principale, \`e l'inversa di
  \xml{foaf:isPrimaryTopicOf}, il dominio \`e
  \xml{foaf:Document}, l'immagine \xml{owl:Thing};
\item \xml{foaf:knows} mette in relazione una persona con un'altra che egli
  conosce, il dominio \`e \xml{foaf:Person}, l'immagine
  \xml{foaf:Person};
\item \xml{foaf:made} mette in relazione un agente con qualcosa
  fatta da questa, \`e l'inversa di \xml{foaf:maker}, il dominio \`e
  \xml{foaf:Agent}, l'immagine \xml{owl:Thing};
\item \xml{foaf:maker} mette in relazione una cosa con l'agente che
  l'ha fatta, \`e l'inversa di \xml{foaf:made}, il dominio \`e
  \xml{owl:Thing}, l'immagine \xml{foaf:Agent};
\item \xml{foaf:mbox} mette in relazione un agente con la sua casella
  di posta, solitamente identificata con lo schema \emph{URI}
  \xml{mailto:}, il dominio \`e
  \xml{foaf:Agent}, l'immagine \xml{owl:Thing};
\item \xml{foaf:member} indica il membro di un gruppo, il dominio \`e
  \xml{foaf:Group}, l'immagine \xml{foaf:Agent};
\end{itemize}
\end{document}
