\documentclass[11pt,a4paper]{article}
\usepackage[latin1]{inputenc}
\usepackage[T1]{fontenc}
\usepackage[italian]{babel}

\pagestyle{empty}

\begin{document}
\begin{center}
\textbf{\LARGE RIASSUNTO}
\\[0.5cm]
\begin{tabular}{lp{8cm}}
\textbf{Titolo tesi:} & ``\textit{Analisi e sviluppo di un crawler per la creazione di una base di conoscenza semantica del personale universitario}''\\[0.25cm]
\textbf{Autore:} & Martina Stefano (\texttt{stefano.martina@gmail.com})\\
\textbf{Relatore:} & Barcucci Elena (\texttt{barcucci@dsi.unifi.it})\\
\textbf{Corelatore:} & Nesi Paolo (\texttt{nesi@dsi.unifi.it})\\
\end{tabular}
\\[1.5cm]
\end{center}
Il testo della tesi \`e il risultato del lavoro di \emph{stage} svolto
presso il laboratorio \emph{DISIT} del Dipartimento di Sistemi e
Informatica dell'Universit\`a degli Studi di Firenze.

Il lavoro effettuato si inserisce nel progetto \emph{OSIM} che ha
l'obiettivo di creare una base di conoscenza semantica riguardante il
personale universitario. In particolare \`e
stata sviluppata la parte dell'applicazione che riguarda il \emph{data
mining} delle informazioni riguardanti le persone, i corsi associati a
queste, e le competenze di ogni persona.

Il progetto fa uso delle
tecnologie \emph{RDF}, \emph{RDFS} e \emph{OWL} per l'organizzazione
della base di conoscenza semantica. Per quanto riguarda
l'applicazione, questa \`e sviluppata in \emph{Java} facendo uso delle
\emph{Servlet} e delle \emph{JSP} per fornire l'interfaccia \emph{web}
con l'utente.

Il testo \`e diviso in due parti: una parte consiste in una
sostanziosa lettura sullo stato dell'arte delle tecnologie usate,
dalle basi dell'\emph{XML} e delle tecnologie correlate sino ad una
visione eterogenea sulle tecnologie del web semantico;
l'altra parte concerne la descrizione del lavoro effettuato e un'analisi
dettagliata dell'applicazione nelle sue parti.

Il \emph{crawler} \`e costituito da due parti, una fase di estrazione
delle parole chiave ed una fase di estrazione di competenze composte
da pi\`u parole chiave. Entrambe le fasi vengono eseguite sulle pagine
del portale \emph{cercachi} del \emph{web} dell'universit\`a.

Per le parti di \emph{NLP} (\emph{Natural Language Processing}) l'applicazione si
appoggia al \emph{framework GATE} sviluppato dall'Universit\`a di
Sheffield.

Gli aspetti del lavoro svolto nell'arco della durata dello \emph{stage} sono stati
molteplici, vi sono state fasi di \emph{debug} del
codice preesistente, fasi di studio e ricerca in diversi campi  e
fasi di sviluppo di nuove funzionalit\`a.

Per quanto riguarda lo studio sono state considerate principalmente le
tecnologie legate a \emph{XML}, al web semantico, e al \emph{NLP}. Lo sviluppo si
\`e concentrato sul linguaggio \emph{Java}, sulle pagine \emph{JSP}, e
sulle \emph{pipe} di \emph{GATE}.

Lo \emph{stage} \`e stato svolto principalmente nei locali del
laboratorio \emph{DISIT} ed \`e stato coadiuvato dai ricercatori
presenti in un ambiente familiare e costruttivo di cooperazione.
\end{document}
