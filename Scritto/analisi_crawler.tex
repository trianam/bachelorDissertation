\documentclass[tesi.tex]{subfiles}
\begin{document}
\chapter{Analisi del crawler}
In questo capitolo viene analizzato nel dettaglio il lavoro svolto all'interno del
progetto \emph{OSIM} dove \`e stato svolto il lavoro, ossia lo studio
e la realizzazione di un servizio di crawling per l'individuazione e
l'estrazione delle competenze del personale universitario dal sito web
dell'ateneo.

\section{Il progetto \emph{OSIM}}
Il progetto \emph{OSIM}, \emph{Open Space Innovative Mind}, ha
l'obiettivo di creare una base di conoscenza dell'Universit\`a di
Firenze che contenga tutte le informazioni necessarie per identificare
le competenze in termini di ricercatori, docenti e gruppi.

In figura \ref{fig:osim} \`e possibile vedere uno schema della
struttura globale del progetto.
\image{img/schemaOsim.eps}{Schema generale del progetto \emph{OSIM}.}{fig:osim}
La parte di \emph{crawling} visibile a sinistra \`e quella che si
occupa di fare \emph{data ingestion} della struttura dell'ateneo in
termini di corsi, dipartimenti, facolt\`a, personale strutturato e
delle competenze attribuibili ad ognuno di questi. Il processo di \emph{crawling} estrae informazioni dalle pagine
web dell'universit\`a ed immagazzina questa conoscenza in un
\emph{database} semantico.

Questa base di dati semantica deve essere strutturata per poter essere
interrogata dagli utenti che desiderano cercare personale con
competenze specifiche.

Il lavoro svolto e presentato nel seguente testo \`e quello di
analizzare e sviluppare un \emph{crawler} delle competenze del
personale realizzato in  in \emph{Java} e che si interfaccia con
l'ontologia della base di dati semantica di \emph{OSIM}. Il
\emph{crawling} viene gestito da un pannello di controllo realizzato
con una pagina \emph{JSP} che si interfaccia con una \emph{Servlet}
\emph{Java}.

\section{Introduzione}
Il crawler \`e progettato in due passaggi. Durante il primo 
vengono estratte le parole chiave; successivamente un esperto di dominio deve effettuare
una selezione di tali parole, scegliendo se inserirle in un
gazetteer o in una black list. Il gazetteer scelto viene
usato nel secondo passaggio per individuare le competenze di ciascuna
persona, al termine di questa fase l'esperto di dominio puo'
organizzare tali competenze in una ontologia definendo relazioni di
tipo gerarchico. L'ontologia ottenuta pu\`o essere scaricata in
formato \emph{3-store} o \emph{rdf/xml}.
Analizziamo nel dettaglio le azioni descritte precedentemente:
\begin{enumerate}
  \item estrazione delle keyword;
  \item creazione del gazetteer;
  \item estrazione delle competenze;
  \item organizzazione delle competenze.
\end{enumerate}
I punti 1 e 3 sono automatici, i punti 2 e 4 necessitano
dell'intervento dell'esperto

\subsection{Estrazione delle keyword}\label{sez:estrazioneKeyword}
Il processo parte dalla pagina della lista del personale afferente ad
un certo dipartimento (l'interfaccia utente, e quindi anche il
crawler, sono divise per dipartimenti) e viene lanciato tramite il
pulsante \emph{launch keyword extraction task}. Appoggiandosi al
\emph{framework GATE} si effettua un primo passaggio di segnatura
della pagina usando un \emph{gazetteer} con la lista precompilata del
personale. In seguito tramite delle regole
\emph{JAPE} vengono ricavati per ciascun nominativo il link alla rispettiva
pagina. Ogni pagina viene poi processata

Per ogni persona in modo analogo si ricava il link alla pagina di
\emph{Penelope} che \`e quella che effettivamente ha le informazioni
utili. Quest'ultima pagina viene aggiunta ad un elenco di pagine da
processare, si ricavano i link alle pagine degli eventuali corsi, e si
aggiungono questi ultimi allo stesso elenco. Una volta ricavato
l'elenco di tutte le pagine utili del personale e dei corsi si usa
\emph{GATE} per estrarre da queste le frasi che contengono. Non \`e
presente l'associazione keyword-persona, visto che interessa solo 
avere un elenco di keyword. Queste frasi vengono salvate in 
\emph{file} temporanei.

\emph{GATE} permette di riconoscere la lingua, quindi ogni
frase viene concatenata ad un \emph{file} di frasi in lingua italiana o ad
uno in lingua inglese. Da questi due \emph{file} vengono
estratti i sostantivi sempre usando \emph{GATE} e vengono tradotti
usando le \emph{api} di \emph{Google translate}. Viene inserito un
record contenente una tripletta \emph{<keyword in italiano, keyword in
  inglese, stato>} in una tabella di un database \emph{Mysql}. Il
significato di \emph{stato} \`e quello di indicare nel passaggio
successivo se la keyword \`e da aggiungere al \emph{gazetteer}, alla
\emph{black list} o a nessun dei due.

\subsection{Creazione del gazetteer}
In questa fase l'esperto di dominio deve intervenire scegliendo le
\emph{keyword} che ritiene pi\`u rilevanti, inserendole in un
\emph{gazetteer}, e scartando quelle meno significative inserendole
eventualmente in una \emph{black list}. Tutte le operazioni sono
effettuate attraverso una interfaccia \emph{user-friendly} che
permette di creare facilmente il \emph{gazetteer}. Il procedimento \`e semplice e
intuitivo e non necessita di ulteriori delucidazioni.

\subsection{Estrazione delle competenze}
Questa fase viene avviata cliccando sul pulsante \emph{launch
  competence crawler}. Ripartendo dall'elenco del personale del
dipartimento vengono ricavati i \emph{link} alla pagina della persona,
quella alla pagina di \emph{penelope}, e quelli dei corsi, in modo
simile a quello visto nella sezione \ref{sez:estrazioneKeyword}. In
seguito viene popolata l'ontologia di dominio dell'ateneo indicando per ogni persona
l'afferenza al dipartimento e i corsi svolti. Dopodich\'e usando
\emph{GATE} vengono segnate le keyword scelte nel passaggio
precedente, e con delle regole \emph{JAPE} si individuano
le competenze composte da pi\`u \emph{keyword}. Il procedimento viene
effettuato sulla pagina della persona e su quelle dei corsi. Queste
competenze (anche solo le keyword non composte), individuate
nelle pagine della persona, vengono aggiunte all'ontologia delle competenze, ovviamente
con la relazione che le collega alla persona. Per lo \emph{storage} dell'ontologia viene
usato il \emph{framework sesame}.

\subsection{Organizzazione delle competenze}
L'ontologia delle competenze pu\`o essere organizzata da un esperto
come una tassonomia arricchita con la relazione semantica tra
competenze \emph{related}. Vengono mostrate due
colonne; sulla sinistra c'\`e l'elenco delle competenze estratte nel
passo precedente, queste possono essere trascinate una ad una nella
colonna di destra e organizzate in modo gerarchico ad albero, ogni
volta che viene spostata una competenza l'ontologia viene aggiornata
aggiungendo la relazione scelta.

\section{Analisi dettagliata}
Adesso verranno descritti nel dettaglio i diversi passaggi del
crawler. Come gi\`a detto ci sono due funzionalit\`a principali:
Il \emph{crawling delle keyword} e il \emph{crawling delle
  competenze}. 

In figura \ref{fig:listadipartimenti}
\image{img/listaDipartimenti.eps}{\emph{Screenshoot} della schermata iniziale.}{fig:listadipartimenti}
\`e possibile vedere la schermata iniziale che appare subito dopo aver
fatto il login. Questa pagina \`e costruita dalla \emph{JSP}
\java{index.jsp} e presenta una lista di dipartimenti, con delle
statistiche sulle varie fasi di \emph{crawling} effettuate. Per ogni
dipartimento sono presenti anche due \emph{link}, uno chiamato
``leggi'' e, se l'utente corrente ha i permessi per il
dipartimento in questione, un altro chiamato ``amministra''. Entrambi
i \emph{link} portano alla stessa pagina di amministrazione del
dipartimento. Il primo permette solo di compiere operazioni di lettura
delle \emph{keyword} e delle competenze. Il secondo fornisce i diritti
per amministrare tutto il dipartimento.

La pagina di amministrazione del dipartimento \`e creata dalla
\emph{JSP} \java{userPage.jsp}, e per il layout viene usato il plugin
\emph{jQueryUI Layout}\footnote{Vedere \cite{jqueryuilayout}.}. Tale
pagina presenta due \emph{tabs} denominate:
\begin{enumerate}
\item ``Ontology manager'';
\item ``Keys selection''.
\end{enumerate}

La prima \emph{tab} visibile in figura \ref{fig:pannelloamministrazione1}
\image{img/pannelloAmministrazione1.eps}{\emph{Screenshoot} del
  pannello di amministrazione, prima \emph{tab}.}{fig:pannelloamministrazione1}
presenta una serie di pulsanti in alto che servono per poter lanciare
le varie fasi e funzioni del \emph{crawler}, inoltre \`e divisa in
quattro parti. In alto a sinistra \`e presente una lista di competenze
organizzate in ordine alfabetico, queste sono il risultato della
seconda fase di \emph{crawling} di competenze. Tali competenze possono
essere trascinate nel \emph{quadrante} in alto a destra dove vi \`e la
possibilit\`a di organizzarle in una ontologia delle competenze
organizzata tramite \emph{SKOS}\footnote{Vedere
  \cite{skos} e sezione \ref{sec:ontologia}.} in modo gerarchico. Allo \emph{SKOS} \`e possibile anche
aggiungere elementi nuovi, non risultanti dal \emph{crawling}.

Il \emph{quadrante} in basso a sinistra costituisce un'area dove vengono visualizzati tutti i messaggi
lanciati dai processi del dipartimento in questione. Il
\emph{quadrante} in basso a destra \`e
costituito dalla lista di processi in coda, gi\`a eseguiti, e quello
correntemente in esecuzione. Dalla coda \`e possibile
eliminare i processi ancora da eseguire.

La seconda \emph{tab} del pannello di amministrazione, visibile in
figura \ref{fig:pannelloamministrazione2}
\image{img/pannelloAmministrazione2.eps}{\emph{Screenshoot} del pannello di amministrazione, seconda \emph{tab}.}{fig:pannelloamministrazione2}
\`e costituita da una tabella che rappresenta tutte le \emph{keyword}
estratte alla fine dell'esecuzione della prima fase di \emph{crawling}
per quel dipartimento.

Anche in questa \emph{tab} vi sono replicati i pulsanti per lanciare
le due fasi di \emph{crawling}, e la tabella \`e costituita da una
serie di righe dove per ognuna \`e presente il valore della
\emph{keyword} nel linguaggio corrente, il valore della \emph{keyword}
negli altri linguaggi (attualmente esistono due linguaggi), il numero di occorrenze
di tale \emph{keyword}, e lo stato attuale della \emph{keyword}: se
\`e stata scelta per far parte del \emph{gazetteer}, se
\`e in
\emph{black list}, o se non \`e ancora stato deciso niente su
questa. Il campo \emph{id} rappresenta un identificatore univoco della
\emph{keyword}, mentre il campo \emph{lang}
serve per poter cambiare lingua ad una \emph{keyword} che \`e stata
classificata erroneamente. \`E possibile ordinare la visualizzazione
delle \emph{keyword} per occorrenza sia in ordine crescente che
decrescente.

Il blocco dei bottoni, di cui \`e possibile vederne un ingrandimento
in figura \ref{fig:bottoni}
\image[5cm]{img/pulsanti.eps}{Dettaglio del blocco dei pulsanti}{fig:bottoni}
\`e costituito da una serie di pulsanti che, in ordine da sinistra a
destra, svolgono le seguenti funzioni:
\begin{description}
  \item[download dell'\emph{RDF Store}] permette di scaricare un file
    in uno dei due formati \emph{N-Triples} o
    \emph{RDF/XML}\footnote{Vedere la sezione
      \ref{sec:serializzazioni} per i dettagli sulle
      serializzazioni.} contenente tutto lo \emph{store} semantico. Viene scaricato sempre lo \emph{store} di tutti i
    dipartimenti, indipendentemente da dove viene richiesto;
  \item[lancia il \emph{crawling} delle \emph{keyword}] per il
    dipartimento corrente, vedere la sezione \ref{sec:crawlingkey};
  \item[lancia il \emph{crawling} delle competenze] per il
    dipartimento corrente, vedere la sezione \ref{sec:crawlingcomp};
  \item[lancia il processo di traduzione delle \emph{keyword}] che non
    sono state tradotte, per qualche eventuale problema, nel corso della
    fase di \emph{crawling} delle \emph{keyword}. Sotto al pulsante \`e
    presente un contatore che indica in tempo reale il numero di
    \emph{keyword} non tradotte;
  \item[lancia il processo di traduzione delle competenze] che non
    sono state tradotte, per qualche eventuale problema, nel corso della
    fase di \emph{crawling} delle competenze. Sotto al pulsante \`e
    presente un contatore che indica in tempo reale il numero di
    competenze non tradotte;
  \item[cancella la tabella delle \emph{keyword}] che non fanno parte
    del \emph{gazetteer};
  \item[cancella le \emph{keyword}] che compaiono nel \emph{gazetteer}
    delle \emph{stop words}. Questa funzione \`e utile quando si vuole
    aggiornare la lista di \emph{stop words} evitando di rilanciare il
    processo di estrazione delle \emph{keyword};
  \item[lancia il processo di pulizia dell'\emph{RDF}] che si occupa
    di eliminare eventuali triple ridondanti o altri casi particolari
    che si possono verificare nel corso dei processi di
    \emph{crawling} e traduzione.
\end{description}

\subsection{Crawling delle \emph{keyword}}\label{sec:crawlingkey}
Il \emph{crawling} delle \emph{keyword} viene lanciato dal pannello di
amministrazione di un dipartimento cliccando sul pulsante
corrispondente visibile in figura \ref{fig:pulsantecrawlkey}
\image[2cm]{img/keycrawl.eps}{Pulsante per lanciare il \emph{crawling} delle keyword.}{fig:pulsantecrawlkey}
che tramite una chiamata \emph{Javascript} alla
\emph{servlet} esegue il metodo \java{execute} del
gestore di questa operazione \java{OSIM\_RequestKBCommand}\footnote{Vedere la sezione \ref{sec:analisijsp} per i
  dettagli sul funzionamento di tali chiamate.}.
Tale metodo, che \`e il solito che gestisce la richiesta di
\emph{crawling} di competenze, aggiunge un nuovo processo alla coda
dei processi\footnote{Vedere la sezione \ref{sec:codaprocessi}.} e,
quando arriva il turno di tale processo, viene eseguito un thread
dell'oggetto \emph{runnable} descritto da
\java{CompetenceKeyCrawler}\footnote{Vedere sezione \ref{sec:classi}\label{note:vediclassi}.}
che \`e la stessa classe del \emph{thread} lanciato nella fase di
\emph{crawling} delle competenze.

Tale \emph{thread} verifica che l'operazione richiesta sia di
\emph{crawling} di \emph{keyword} e quindi istanzia la classe
\java{KeywordExtractionEngine}\footnote{Vedi nota
  \ref{note:vediclassi}.}, e lancia il suo metodo \java{extractKey}.

Questa fase comincia da una pagina del portale ``\emph{cercachi}'' con la
lista del personale del dipartimento\footnote{Ad esempio
  \link{http://www.unifi.it/cercachi/show.php?f=s\&codice=051400\&fonte=informatica}
  per il dipartimento di informatica e sistemi.} di cui vogliamo
prendere le \emph{keyword}. Tale pagina \`e possibile reperirla
all'interno della tabella \sql{departments} del
database\footnote{Vedere sezione \ref{sec:database}\label{note:vedidatabase}.}, dove ci sono
memorizzate tutte le pagine di partenza.

In figura \ref{fig:crawlingKey} \`e possibile vedere lo schema del
funzionamento della prima fase.
\begin{figure}
  \begin{center}
    \begin{tikzpicture}[auto]
      \node [db] (tabDip) {Tabella dipartimenti};
      \node [web, right = 1cm of tabDip] (listaPers) {Pagina con lista persone dipartimento};
      \node [gaz, above = 1cm of tabDip] (gazPers) {Gazetteer persone};
      \node [app, right = 2cm of gazPers] (gate1) {Estrazione \emph{link}};
      \node [weblist, right = 1cm of gate1] (pagine) {\emph{url} di tutte le pagine utili};
      \node [gate, below = 1cm of pagine] (gate) {GATE};
      \node [app, right = 1cm of pagine] (gate2) {Estrazione frasi};
      \node [filelist, below = 1cm of gate2] (files) {File di frasi};
      \node [app, below = 5cm of files] (gate3) {Riconoscim. lingua};
      \node [file, left = 1cm of gate3] (fileEn) {File frasi inglesi};
      \node [file, above = 1cm of fileEn] (fileIt) {File frasi italiane};
      \node [file, below = 1cm of fileEn] (fileBo) {File frasi sconosciute};
      \node [app, left = 1cm of fileEn] (gate4) {Estrazione sostantivi};
      \node [gaz, below = 1cm of gate4] (gazStop) {Stop words};
      \node [tab, left = 1cm of gate4] (keytab) {Tabella sostantivo lingua frequenza};
      \node [app, below = 4cm of keytab] (trad) {Traduzione};
      \node [db, above right = 0.25cm and 1.75cm of trad] (cache) {Cache};
      \node [webapp, below right = 0.25cm and 1cm of trad] (google) {Google Translate};
      \node [db, below = 1cm of trad] (keydbtab) {Tabella \emph{keyword}};

      \node [legenda, below = 1cm of fileBo.south east] (leg) {
        \begin{tikzpicture}
          \node[db,legendait] (uno) {};
          \node[legendalab, right = 0.1cm of uno] (unol) {Database};
          
          \node[gaz,legendait, below = 0.5cm of uno] (due) {};
          \node[legendalab, right = 0.1cm of due] {Gazetteer};
          
          \node[web,legendait, below = 0.5cm of due] (tre) {};
          \node[legendalab, right = 0.1cm of tre] {Pag. web};

          \node[tab,legendait, below = 0.5cm of tre] (qua) {};
          \node[legendalab, right = 0.1cm of qua] {Tabella\\in mem.};

          \node[app,legendait, right = 0.5cm of unol] (cin) {};
          \node[legendalab, right = 0.1cm of cin] {Processo};

          \node[file,legendait, below = 0.5cm of cin] (sei) {};
          \node[legendalab, right = 0.1cm of sei] {File};
          
          \node[webapp,legendait, below = 0.5cm of sei] (set) {};
          \node[legendalab, right = 0.1cm of set] {Servizio\\web};
        \end{tikzpicture}
      };

      \draw [freccia] (tabDip) -- (listaPers);
      \draw [freccia] (listaPers) -- (gate1);
      \draw [freccia] (gazPers) -- (gate1);
      \draw [freccia] (gate1) -- (pagine);
      \draw [freccia] (pagine) -- (gate2);
      \draw [freccia] (gate2) -- (files);
      \draw [freccia] (files) -- (gate3);
      \draw [freccia] (gate3) -- (fileIt);
      \draw [freccia] (gate3) -- (fileEn);
      \draw [freccia] (gate3) -- (fileBo);
      \draw [freccia] (fileIt) -- (gate4);
      \draw [freccia] (fileEn) -- (gate4);
      \draw [freccia] (gazStop) -- (gate4);
      \draw [freccia] (gate4) -- (keytab);
      \draw [freccia] (keytab) -- (trad);
      \draw [dfreccia] (cache) -- (trad);
      \draw [dfreccia] (google) -- (trad);
      \draw [freccia] (trad) -- (keydbtab);

      \draw [dfreccia] (gate) -- (gate1);
      \draw [dfreccia] (gate) -- (gate2);
      \draw [dfreccia] (gate) -- (gate3);
      \draw [dfreccia] (gate) -- (gate4);

    \end{tikzpicture}
  \end{center}
  \caption{Schema della fase di \emph{crawling} delle \emph{keyword}}
  \label{fig:crawlingKey}
\end{figure}
Lo schema indica il flusso logico delle operazioni che vengono svolte
dal processo. Il \emph{crawler} usa quattro \emph{pipe} di
\emph{GATE} diverse, che saranno spiegate in dettaglio nella sezione
\ref{sec:applicazionigate}.

In questa prima fase il crawler si preoccupa preventivamente di
reperire tutti i \emph{link} alle pagine utili salvandoli in un vettore, in
seguito usando \emph{GATE} individua tutte le frasi presenti su tutte
le pagine puntate dai \emph{link} e salva ogni frase in un file
temporaneo. Su questi \emph{file} viene usato nuovamente
\emph{GATE} per individuare la lingua di ogni \emph{file} e in
seguito le frasi vengono accodate in un singolo \emph{file} di frasi
in italiano, uno in inglese, e uno di lingua sconosciuta. Quello di
lingua sconosciuta viene in seguito ignorato.

Su questi due
\emph{file} viene usato \emph{GATE} per individuare
tutti i \emph{sostantivi} delle frasi che non compaiano nella lista
delle \emph{stop words}, e per ognuno di essi viene calcolato il
numero di occorrenze. Questi sostantivi, che rappresentano il
possibile elenco di \emph{keyword}, vengono salvati in una tabella
temporanea che ha le tre colonne:
\\\vbox{\begin{itemize}
\item keyword;
\item lingua;
\item occorrenze.
\end{itemize}}

Per ogni keyword di questa tabella viene calcolata la traduzione
nell'altra lingua, avvalendosi del servizio di \emph{Google Translate}.
Prima di accedere al servizio esterno, viene interrogata una cache
interna composta da righe nella forma:
\begin{itemize}
\item valore italiano;
\item valore inglese.
\end{itemize}
che viene aggiornata ogni volta che \`e necessario tradurre un valore
non presente.

Tutti i valori delle \emph{keyword}, le relative traduzioni, e le
occorrenze nelle pagine, vengono memorizzati nella tabella del
database relativa alle \emph{keyword} del dipartimento per il quale
\`e stata lanciata questa fase. Oltre ai valori detti viene
memorizzato per ogni \emph{keyword} un valore \emph{status}
inizializzato a zero. Questo status \`e il valore che l'esperto
modifica quando va a scegliere le \emph{keyword} da candidare per la
seconda fase, o quelle da mettere in \emph{black list}.

Vengono adesso analizzate nel dettaglio le varie parti di questa fase.

\subsubsection{Estrazione \emph{link}}
In questo punto si sfrutta \emph{GATE}, e in particolare le
annotazioni in base ai \emph{gazetteer}, per poter definire un sottoinsieme
    di persone su cui pu\`o essere effettuato il \emph{crawling}. Vengono
    ricavati tutti i \emph{link} alle pagine delle persone che sono nella \emph{lista
      del personale di un dipartimento} e che sono \emph{anche nel
      gazetteer}. Questi link di primo livello non sono utili allo scopo
    del crawler in quanto le pagine a cui puntano non contengono
    informazioni utili\footnote{Sono pagine nella forma:\\
      \link{http://www.unifi.it/cercachi/scheda.php?f=p\&codice=3718\&bol=AND\&cognome=nesi\\\&nome=paolo}.},
    quindi per ognuna di queste pagine vengono riutilizzate delle regole \emph{JAPE},
    per ricavare il link che punta alla pagina della persona sul portale
    \emph{Penelope}.

    Queste pagine del
    portale \emph{Penelope}\footnote{Ad esempio:\\
      \link{http://www.unifi.it/index.php?module=ofform\&mode=2\&cmd=1\&AA=2011\&fac=200006\\\&ord=N\&doc=3f2a3d2e36302b}.}
    sono utili in quanto sono presenti frasi che contengono sostantivi
    dai quali \`e possibile estrarre
    informazioni sulle competenze della persona, ad esempio il
    \emph{curriculum} della persona o gli interessi personali. Il loro
    \emph{url} viene memorizzato in un vettore da
    passare al punto successivo di estrazione delle frasi.

    Nelle pagine delle
    persone di \emph{Penelope} vi \`e una serie di \emph{link} alle pagine
    dei corsi\footnote{Ad esempio:\\
      \link{http://www.unifi.it/index.php?module=ofform\&mode=1\&cmd=3\&AA=2011\&fac=200006\\\&cds=B047\&pds=GEN\&afId=292585\&lan=0\&ord=N\&doc=3f2a3d2e36302b}.}
    (sempre sul portale \emph{Penelope}), che a loro volta
    contengono informazioni utili per il \emph{crawling} delle
    competenze. Quindi anche l'\emph{url} di queste pagine dei corsi
    viene aggiunto al solito vettore.

    L'output di questa parte \`e un vettore
    di \emph{url}.
\subsubsection{Estrazione frasi}
 In questa parte viene effettuato \emph{natural language
      processing} tramite \emph{GATE}, vengono analizzate le pagine indicate
    nel vettore di \emph{url} precedente e individuate le
    frasi presenti. Ogni frase estrapolata viene memorizzata in un
    singolo \emph{file} temporaneo.

    Alla fine si ha una serie di \emph{file} contenenti una
    singola frase, che vengono letti nella parte successiva.
\subsubsection{Riconoscimento lingua}
 Anche in questa parte viene usato \emph{GATE} per
    fare \emph{natural language processing}. Per
    ogni frase (quindi per ogni \emph{file} di input) viene
    individuata la lingua nel quale \`e scritta. Quindi a seconda di
    questa, viene accodata la frase in un \emph{file} di frasi in
    italiano, oppure in uno in inglese. Se la lingua di una frase non
    viene riconosciuta oppure se viene riconosciuta come una lingua
    diversa dall'italiano o dall'inglese, questa viene accodata in un file di lingua
    sconosciuta. Quest'ultimo viene ignorato ed ha funzione di \emph{debug}. Anche
    questi \emph{file} come quelli delle frasi sono temporanei e
    vengono cancellati dopo l'esecuzione del \emph{crawling} di
    \emph{keyword}.
\subsubsection{Estrazione sostantivi}
 In questa parte vengono usate due \emph{pipe}
    distinte di \emph{GATE} per individuare i sostantivi nelle
    frasi. Per le frasi in inglese viene usata la \emph{pipe} standard
    di \emph{GATE}, per quelle in italiano viene usata
    \emph{TreeTagger}.

    Queste \emph{pipe} si occupano anche di non annotare i sostantivi
    che compaiono in un certo \emph{gazetteer} di \emph{stop words}.

    Individuati ed estratti i sostantivi, questi vengono memorizzati
    in una tabella temporanea con colonne:
    \begin{itemize}
    \item keyword;
    \item lingua;
    \item occorrenze.
    \end{itemize}
    Dove \emph{keyword} \`e un singolo sostantivo estratto da una
    frase, la \emph{lingua} \`e nota sapendo da quale
    dei due \emph{file} (quello delle frasi italiane e quello delle
    inglesi) \`e stato estratto il sostantivo, e \emph{occorrenze} \`e
    il numero di occorrenze di un certo sostantivo all'interno di uno
    dei due \emph{file}.

    Per la precisione la tabella temporanea \`e un \java{Hashtab}
    \emph{Java} di due campi: un campo \java{key} che contiene una
    stringa nel formato \java{valore@lang} con \java{valore} che
    indica il valore
    del sostantivo e \java{lang} che indica la lingua (attualmente
    \java{it} o \java{en}), e un campo \java{freq} che contiene il
    numero di occorrenze.
\subsubsection{Traduzione}
 Questo ultima parte si occupa di tradurre i
    sostantivi trovati nel passaggio precedente e di unire le occorrenze
    di eventuali doppioni in modo da avere una \emph{keyword} unica
    per le stesse occorrenze in italiano e in inglese. Inoltre in questo
    modo \`e possibile rendere tutto il sistema multilingua, e un
    utente pu\`o effettuare \emph{query} sia in italiano che in
    inglese.

    La traduzione viene effettuata usando le \emph{api} del servizio
    \emph{Google Translate} integrato con una cache interna implementata
    con una tabella nel \emph{database} con due colonne:
    \begin{itemize}
    \item valore in italiano;
    \item valore in inglese.
    \end{itemize}
    La cache funziona nel seguente modo: quando viene richiesta una
    traduzione di una \emph{keyword} della tabella di \emph{input},
    viene prima interrogata la tabella di \emph{cache}, se esiste
    gi\`a una riga con il valore da tradurre cercato allora viene
    usato il corrispondente valore tradotto di tale riga della
    \emph{cache}; se non esiste una riga corrispondente viene invocato
    il servizio di \emph{Google} e viene aggiunta una riga alla
    tabella di \emph{cache} con i valori corrispondenti.

    Essendo necessario l'uso di un servizio esterno \`e previsto un
    meccanismo di posticipazione delle traduzioni. Nel caso in cui,per esempio, \emph{Google Translate}
    fosse momentaneamente non disponibile, la keyword viene comunque
    presa in considerazione, ma la sua traduzione viene impostata al
    valore \java{TO BE TRANSLATED}. Nel pannello di amministrazione
    \`e presente un pulsante con il relativo contatore di
    \emph{keyword} da tradurre che lancia un nuovo tentativo di
    traduzione di tutte le \emph{keyword} rimaste non
    tradotte. Ovviamente prima di lanciare la seconda fase di
    \emph{crawling} delle competenze \`e bene che non ci siano
    \emph{keyword} non tradotte.

    Le keyword con il relativo valore tradotto vengono salvate in una
    tabella di \emph{keyword}, specifica per ogni
    dipartimento\footnote{La tabella \`e descritta nella
      sezione \ref{sec:tabkeyword}.}.

    Questa tabella presenta le colonne:
    \begin{itemize}
    \item valore in italiano;
    \item valore in inglese;
    \item occorrenze,
    \item \emph{status}.
    \end{itemize}
    I primi due valori sono il valore originale della
    \emph{keyword} e il suo valore tradotto nell'altra lingua. Le
    occorrenze indicano quante volte la \emph{keyword} \`e stata
    trovata. \emph{Status} \`e il valore che indica l'azione scelta
    dall'esperto di dominio sulla \emph{keyword} (in \emph{gazetteer}
    o in \emph{black list}).

    La tabella delle \emph{keyword} non viene
    azzerata prima dell'esecuzione di questa fase, quindi i valori
    preesistenti rimangono memorizzati. Nel caso in cui vi \`e una
    richiesta di inserire una riga di una \emph{keyword} gi\`a
    esistente nella tabella, viene semplicemente aggiornato il valore
    delle occorrenze della \emph{keyword} e lasciato inalterato lo
    \emph{status}.

    Un'ultima considerazione \`e che esistono certe \emph{keyword} che
    pur avendo valori diversi in una lingua, hanno la stessa
    traduzione nell'altra, ad esempio i sostantivi inglesi ``program''
    e ``programme'' hanno la stessa traduzione italiana
    ``programma''. Questo problema viene risolto rendendo le
    \emph{keyword} in questa forma effettivamente indipendenti, ad
    esempio, i valori precedenti vengono memorizzati con le due righe
    distinte della tabella \ref{table:esempiokeywordsuguali}.
    \begin{table}
      \begin{center}
        \begin{tabular}{|c|c|c|}
          \hline
              {\bfseries en\_value} & {\bfseries it\_value} & {\bfseries \ldots} \\
              \hline
              ``program'' & ``programma'' & \ldots\\
              ``programme'' & ``programma'' & \ldots\\
              \hline
        \end{tabular}
      \end{center}
      \caption{Esempio di \emph{keyword} diverse con lo stesso valore}
      \label{table:esempiokeywordsuguali}
    \end{table}
    Queste \emph{keyword} vengono trattate come \emph{keyword} differenti
    nonostante abbiano un valore che \`e lo stesso in entrambe.

\subsection{Fase di selezione del gazetteer di \emph{keyword}}\label{sec:gazkeywords}
Finita la fase di \emph{crawling} delle \emph{keyword} \`e necessario
l'intervento di un esperto che decida quale deve essere il
\emph{gazetteer} delle \emph{keyword} da prendere in considerazione
per la seconda fase, ossia quelle che sono considerate significative
per il personale di un certo dipartimento. Anche
questa fase \`e divisa dipartimento per dipartimento e l'esperto
deve agire nel pannello di configurazione di un dipartimento nella
seconda \emph{tab} vista in figura \ref{fig:pannelloamministrazione2}
a pagina \pageref{fig:pannelloamministrazione2}.

Nella figura \ref{fig:creazionegaz}
\begin{figure}
  \begin{center}
    \begin{tikzpicture}[auto]
      \node [db] (tabKey) {Tabella \emph{keyword}};
      \node [web, right = 2cm of tabKey] (front) {Frontend pannello di amministazione};
      \node [man, below = 1.5cm of front] (pers) {Esperto};
      
      \draw [dfreccia] (tabKey) -- (front);
      \draw [dfreccia] (front) -- (pers);

    \end{tikzpicture}
  \end{center}
  \caption{Schema della fase di creazione del \emph{gazetteer}}
  \label{fig:creazionegaz}
\end{figure}
\`e possibile vedere uno schema di questa fase, dove l'esperto
pu\`o
cambiare lingua ad una \emph{keyword}, segnare una \emph{keyword} come
appartenente al \emph{gazetteer} di \emph{input} della fase di
\emph{crawling} di competenze, oppure segnarla come in \emph{black
  list}.

\subsection{Crawling delle competenze}\label{sec:crawlingcomp}
Quando \`e stato scelto un \emph{gazetteer} di \emph{keyword} consono,
\`e possibile lanciare la fase di \emph{crawling} di competenze
cliccando sul pulsante in figura \ref{fig:pulsantecrawlcomp}.
\image[2cm]{img/compcrawl.eps}{Pulsante per lanciare il \emph{crawling} delle competenze.}{fig:pulsantecrawlcomp}

Il funzionamento iniziale della chiamata \`e identico a quello del
\emph{crawling} delle \emph{keyword}. Viene fatta una chiamata
\emph{Javascript} come spiegato in sezione \ref{sec:analisijsp}, viene aggiunto un processo in
coda, e quando \`e il suo turno viene eseguito il \emph{thread}
\java{CompetenceKeyCrawler}. A differenza della fase precedente il
\emph{thread} riconosce che l'operazione \`e di \emph{crawling} di
competenze, perci\'o istanzia la classe
\java{CompetenceExtractionEngine} ed esegue il suo metodo \java{run}.

In figura \ref{fig:crawlingComp} \`e possibile vedere uno schema del
funzionamento di questa fase.
\begin{figure}
  \begin{center}
    \begin{tikzpicture}[auto]
      \node [db] (tabDip) {Tabella dipartimenti};
      \node [web, right = 1cm of tabDip] (listaPers) {Pagina lista persone dipartimento};
      \node [gaz, above = 1cm of tabDip] (gazPers) {Gazetteer possibili persone};
      \node [app, right = 2cm of gazPers] (estr) {Estrazione \emph{link}};
      \node [tab, right = 1cm of estr] (tabella) {Tabella\\nome - url};
      \node [round, below = 2cm of tabella] (pallino) {};
      \node [app, below = 1cm of pallino] (analisi) {Analisi persona};
      \node [app, below right = 0.5cm and 1cm of analisi.south] (trad) {\dimg Traduzione};
      \node [db, right = 0.75cm of analisi] (carriera) {Tabella carriera};
      \node [rdfclist, below = 2cm of analisi] (insegnamenti) {Istanze degli insegnamenti};
      \node [rdfc, left = 0.5cm of insegnamenti] (persona) {Istanza della persona};
      \node [rdfclist, right = 0.5cm of insegnamenti] (competenze) {Istanze delle competenze};
      \node [app, text width=3cm, below right = 2cm and 0.25 of insegnamenti.south] (serializza) {Serializzazione};
      \node [rdf, left = 1cm of serializza] (rdf) {RDF store};
      \node [choice, vuoto, below = 1cm of serializza] (rombo) {};

      \node [gate, below = 2cm of estr.east] (gate) {GATE};

      \node [point, right = 0.25cm of carriera] (p) {};

      \node [legenda, below = 1cm of tabDip.south east] (leg) {
        \begin{tikzpicture}
          \node[db,legendait] (uno) {};
          \node[legendalab, right = 0.1cm of uno] {Database};
          
          \node[rdf,legendait, below = 0.5cm of uno] (due) {};
          \node[legendalab, right = 0.1cm of due] {RDF store};
          
          \node[gaz,legendait, below = 0.5cm of due] (tre) {};
          \node[legendalab, right = 0.1cm of tre] {Gazetteer};
          
          \node[web,legendait, below = 0.5cm of tre] (qua) {};
          \node[legendalab, right = 0.1cm of qua] {Pag. web};

          \node[tab,legendait, below = 0.5cm of qua] (cin) {};
          \node[legendalab, right = 0.1cm of cin] {Tabella\\in mem.};

          \node[app,legendait, below = 0.5cm of cin] (sei) {};
          \node[legendalab, right = 0.1cm of sei] {Processo};

          \node[rdfc,legendait, below = 0.5cm of sei] (set) {};
          \node[legendalab, right = 0.1cm of set] {Classe\\RDF};
        \end{tikzpicture}
      };

      \draw [freccia] (tabDip) -- (listaPers);
      \draw [freccia] (listaPers) -- (estr);
      \draw [freccia] (gazPers) -- (estr);
      \draw [freccia] (estr) -- (tabella);
      \draw [freccia] (tabella) -- (pallino);
      \draw [freccia] (pallino) -- (analisi);
      \draw [freccia] (carriera) -- (analisi);

      \draw [freccia] (analisi) -- (persona);
      \draw [freccia] (analisi) -- (insegnamenti);
      \draw [freccia] (analisi) -- (trad);
      \draw [freccia] (trad) -- (competenze);

      \draw [freccia] (persona) -- (serializza);
      \draw [freccia] (insegnamenti) -- (serializza);
      \draw [freccia] (competenze) -- (serializza);
      \draw [freccia] (serializza) -- (rombo);
      \draw [freccia] (serializza) -- (rdf);

      \draw [nfreccia] (rombo) -| (p);
      \draw [freccia] (p) |- (pallino);

      \draw [dfreccia] (analisi) -- (gate);
      \draw [dfreccia] (estr) -- (gate);
    \end{tikzpicture}
  \end{center}
  \caption{Schema della fase di \emph{crawling} delle competenze}
  \label{fig:crawlingComp}
\end{figure}
I dettagli della traduzione sono stati omessi per motivi di
spazio, comunque sono completamente analoghi a quelli in
fase di \emph{crawling} di \emph{keyword}.

In questa fase vengono create le istanze delle classi \emph{RDFS} che riguardano le
persone, i corsi insegnati e le competenze. In particolare vengono create, o
modificate, le classi di tipo:
\begin{itemize}
\item \xml{foaf:Person} per la persona;
\item \xml{uni:Course} per gli insegnamenti;
\item \xml{uni:temporaryXXXStore} per le competenze, dove \xml{XXX} sta per
  il codice associato ad un certo dipartimento.
\end{itemize}
Nella sezione \ref{sec:ontologia} \`e possibile vedere nel dettaglio
lo schema \emph{RDFS} e le relazioni tra queste istanze.

Inizialmente \`e simile alla fase di estrazione delle \emph{keyword},
vengono estratti i \emph{link} relativi alle persone di un
dipartimento che compaiono nel \emph{gazetteer}. A differenza di
questa, viene esplorato solo il primo \emph{link} di ogni persona e
memorizzato in una tabella assieme al nome di quest'ultima.

Le fasi di analisi della persona e di serializzazione delle classi
vengono iterate per ogni riga di tale tabella, quindi per ogni
persona.

\subsubsection{Analisi della persona}
La fase di analisi della persona si occupa di creare le istanze delle classi
\emph{RDFS}, anche se non vengono memorizzate nell'\emph{RDF
  Store}. Nello specifico vengono creati degli oggetti \emph{Java} di
classe \java{UnifiOntologyInstance}; la memorizzazione nell'\emph{RDF
  Store}, assieme alle propriet\`a che mettono in relazione le diverse
istanze, avviene nella fase successiva di serializzazione.

 Partendo dall'\emph{url} della pagina e dal nome della
persona, viene richiamato il metodo \java{analyze} della classe
\java{PersonAnalyzer} che esegue le seguenti operazioni:
\begin{enumerate}
\item vengono recuperate le informazioni riguardanti la persona:
  \begin{itemize}
  \item l'hash del codice fiscale tramite la tabella \sql{carriera};
  \item l'indirizzo delle pagine della persona, della piattaforma
    \emph{cercachi}, e di \emph{Penelope};
  \item i link alle pagine degli eventuali corsi insegnati dalla
    persona;
  \end{itemize}
\item viene creata una istanza della persona;
\item Per ogni link ad un eventuale corso vengono recuperate le sue
  informazioni dalle pagine di questi:
  \begin{itemize}
  \item il nome del corso;
  \item il codice del corso;
  \end{itemize}
\item vengono create le istanze di tali corsi;
\item vengono cercate le competenze su ogni pagina del corso e create
  le loro istanze;
\item vengono cercate le competenze sulla pagina della persona e
  create le loro istanze.
\end{enumerate}

Il recupero dell'hash del codice fiscale serve per creare l'\emph{URI}
della persona, ed avviene facendo una \emph{query} al \emph{database}
usando il nome della stessa. Se la \emph{query} non torna il valore allora la
persona viene saltata.

Il recupero degli indirizzi delle pagine avviene sfruttando
\emph{GATE}, cos\`i come gli eventuali indirizzi delle pagine dei
corsi.

L'istanza della persona viene creata usando l'\emph{hash} del codice
fiscale come parte dell'\emph{URI}\footnote{Vedere la sezione
  \ref{sec:ontologia} per i dettagli sugli \emph{URI} e sulle
  propriet\`a.}, alla persona vengono assegnate le propriet\`a
riguardanti l'indirizzo delle sue pagine, e il dipartimento di
afferenza.

Viene usato \emph{GATE} sulla pagina di \emph{Penelope} per trovare
eventuali \emph{link} alle pagine dei corsi insegnati dalla
persona. Ognuna di queste pagine di corsi viene analizzata
singolarmente, viene usato \emph{GATE} per estrarre il nome del corso
e il suo codice da usare come parte dell'\emph{URI}. Viene creata una
istanza per tale corso assegnando le propriet\`a del suo nome e della
sua pagina.

Sempre sulla pagina del corso viene usato \emph{GATE} per trovare le
competenze, e per ognuna di esse viene tradotto il valore con un
sistema analogo a quello usato durante il \emph{crawling} delle
\emph{keyword}, e viene creata l'istanza della competenza. La
traduzione viene rieffettuata perch\'e possono essere estratte
competenze composte da pi\`u \emph{keyword}. Viene memorizzato nell'oggetto che
rappresenta l'istanza della competenza, anche il riferimento al corso
dove questa \`e stata trovata.

Il procedimento si ripete per tutti i corsi presenti, e infine viene
analizzata anche la pagina di \emph{Penelope} della
persona.

\subsubsection{Serializzazione}
Questa \`e la fase in cui le istanze create vengono effettivamente
memorizzate nell'\emph{RDF Store} e in cui vengono anche create le
relazioni tra queste.

Le operazioni effettuate sono:
\begin{enumerate}
\item controlla se le competenze trovate esistono gi\`a o no;
  questo serve per sapere se assegnare alla risorsa il tipo
  \xml{uni:temporaryXXXStore}\footnote{Vedi sezione
    \ref{sec:ontologia}.};
\item scrittura nell'\emph{RDF Store} dell'istanza della persona e
  di quelle delle competenze e dei corsi;
\item scrittura dell'associazione tra persona e competenza usando un
  \emph{bnode}.
\end{enumerate}

I dettagli della fase di serializzazione sono descritti meglio nella
sezione \ref{sec:ontologia}.

\subsection{Organizzazione dello \emph{SKOS}}
Questa fase, come quella di creazione del \emph{gazetteer}, prevede
l'intervento di un esperto. In questa fase vengono organizzate le
competenze, estratte nella fase precedente, in un albero gerarchico,
per descrivere la specificit\`a o la generalit\`a di una competenza
rispetto ad un'altra.

Durante la creazione dell'albero, vengono create, in maniera
trasparente all'esperto, le opportune relazioni semantiche nello \emph{SKOS}.
\end{document}
