\documentclass[tesi.tex]{subfiles}
\begin{document}
\chapter{XML e tecnologie correlate}
L'\emph{Extensible Markup Language} abbreviato con \emph{XML} \`e, come
dice il nome, un linguaggio di \emph{markup}, di \emph{marcatura}. Esso
deriva dallo \emph{Standard Generalized Markup Language} (\emph{SGML})
e nasce nel 1996 da un gruppo del \emph{W3C} sotto la guida di
\emph{Jon Bosak} della \emph{Sun Microsystems}.

Secondo le specifiche del
\emph{W3C}\footnote{\link{http://www.w3.org/TR/xml/} e nello specifico
  \link{http://www.w3.org/TR/xml/\#sec-origin-goals}.} gli obiettivi fondamentali
che hanno guidato lo sviluppo dell'XML sono:
\begin{enumerate}
  \item XML deve essere semplicemente usabile in internet;
  \item XML deve supportare un'ampia variet\`a di applicazioni;
  \item XML deve essere compatibile con SGML;
  \item deve essere facile scrivere programmi in grado di processare
    documenti XML;
  \item il numero delle funzionalit\`a opzionali di XML deve essere
    mantenuto al minimo, idealmente a zero;
  \item i documenti XML devono essere leggibili dall'uomo e
    ragionevolmente chiari;
  \item il design dell'XML deve essere preparato velocemente;
  \item il design dell'XML deve essere formale e conciso;
  \item i documenti XML devono essere facili da creare;
  \item la sinteticit\`a nel markup XML \`e di minima importanza.
\end{enumerate}

L'uso principale dell'XML \`e quello di creare documenti e scambiarli
tra diverse applicazioni usando una tecnologia che permette di creare
documenti strutturati con un livello scalabile di complessit\`a e con
un alto livello di flessibilit\`a. Infatti XML ha una sintassi
estremamente semplice e permette la rappresentazione strutturata
pressoch\'e di qualsiasi tipo di dato in quanto tale struttura viene
definita a piacere.

La codifica dei caratteri dei documenti XML \`e l'\emph{unicode}.

Oltre alle specifiche dei documenti XML esistono anche un certo numero
di tecnologie ausiliarie, anche se spesso con \emph{XML} ci si
riferisce a tutte quante in realt\`a esse non fanno parte delle
specifiche del nucleo propriamente detto \emph{XML}. Di tali
tecnologie noi analizzeremo solo quelle pi\`u importanti, ossia:
\begin{description}
  \item[Namespaces] l'uso dei namespace permette di definire dei
    \emph{vocabolari} che permettono di identificare e
    differenziare i tag tra di loro;
  \item[DTD] \`e un linguaggio di \emph{schema} che definisce
    praticamente una grammatica con cui validare un documento XML;
  \item[XML Schema] \`e il successore di \emph{DTD}, ed \`e pi\`u
    potente di questo, \`e in formato XML e fornisce un ricco
    sistema di tipi di dato;
  \item[XSLT] sta per \emph{Extensible Stilesheet Language
    Transformation} e permette di creare dei documenti di
    trasformazione che, insieme ad un \emph{parser}, permettono di
    elaborare un documento XML e trasformarlo in un altro documento
    (che pu\`o anche non essere XML);
  \item[XPath] \`e un linguaggio che permette di indirizzare parti
    di un documento XML. Grazie ad XPath \`e possibile
    fare operazioni come \emph{pescare} il contenuto di un certo tag
    o prendere tutti i figli di un certo tag;
  \item[XSL-FO] Sta per \emph{Extensible Stilesheet Language
    Formatting Objects} ed \`e un linguaggio di presentazione di documenti.
    Nell'idea originale un documento XML doveva essere trasformato
    tramite uno schema XSLT (con l'uso anche di XPath) in un
    documento XSL-FO che poteva essere
    renderizzato da un \emph{processore FO} in un output leggibile.
\end{description}

Di seguito analizziamo nel dettaglio l'xml e tali tecnologie.

\section{XML}
L'\emph{XML} � un \emph{metalinguaggio} di \emph{markup} che definisce
una sintassi per esprimere una struttura ad albero dove ogni nodo si
definisce \emph{tag}. La caratteristica principale � che � possibile
definire una struttura personalizzata a seconda dell'utilizzo che se
ne vuole fare, cosa che ne permette un uso molto eterogeneo: dalle
pagine web (con l'\emph{XHTML} che � una formalizzazione in \emph{XML}
dell'\emph{HTML}) allo scambio di dati o alla definizione di
linguaggi.

La codifica dei caratteri � \emph{unicode} valida in certi range,
inoltre � possibile inserire ogni carattere con la sua corrispondente
codifica decimale \xml{\&\#...;} o esadecimale
\xml{\&\#x...;}. Ci sono dei caratteri speciali che non possono essere
usati nel contenuto, ma solo nella struttura, per inserirli
nel contenuto � necessario usare le entit� XML illustrate nella
tabella \ref{tab:entita}.
\begin{table}
  \begin{center}
    \begin{tabular}{|c|c|}
      \hline Carattere & Codifica\\
      \hline \textless & \xml{\&lt;}\\
      \textgreater & \xml{\&gt;}\\
      \& & \xml{\&amp;}\\
      ` & \xml{\&apos;}\\
      `` & \xml{\&quot;}\\
      \hline
    \end{tabular}
    \caption{Entit� \emph{XML}}\label{tab:entita}
  \end{center}
\end{table}

Un codice \emph{XML} � costituito da un albero di tags con un singolo tag come radice.
\begin{code}
  \begin{xmlblock}
    <nome attributo1="valore1" attributo2="valore2" attributoX="valoreX">
      contenuto
    </nome>
  \end{xmlblock}
  \caption{Tag generico}\label{cod:singolotag}
\end{code}
Ogni singolo tag � costituito come nel codice \ref{cod:singolotag},
dove \xml{contenuto} pu� essere vuoto, un testo oppure una serie di
altri \emph{tags}, \xml{nome} e \xml{attributoX} non devono contenere
i caratteri: \\
\xml{!"\#\$\%\&'()*+,/;<=>?@[$\backslash$]\^`\{|\}~},\\
spazi, o cominciare per un numero o per \xml{-} o
\xml{.}. Ovviamente vi possono essere un numero qualsiasi di
attributi, anche nessuno. Nel caso in cui \xml{contenuto} sia
vuoto, � possibile usare una forma abbreviata come nel codice
\ref{cod:abbreviato}.
\begin{code}
  \begin{xmlblock}
    <nome attributo1="valore1" attributo2="valore2" attributoX="valoreX" />
  \end{xmlblock}
  \caption{Tag vuoto abbreviato}\label{cod:abbreviato}
\end{code}

Oltre alla struttura dei \emph{tags}, nello standard pu� essere
presente anche una riga preliminare di dichiarazione, definita come nel
codice \ref{cod:dichiarazione},
\begin{code}
  \begin{xmlblock}
    <?xml version="1.0" encoding="UTF-8" ?>
  \end{xmlblock}
  \caption{Dichiarazione}\label{cod:dichiarazione}
\end{code}
che deve essere necessariamente la prima e dove vi sono le
informazioni sulla versione usata dello standard e sulla codifica dei
caratteri.

Se un documento \emph{XML} rispetta le semplici regole appena definite
si dice che esso � \emph{ben formato}, oltre a questo, � possibile
definire uno schema che indica la struttura e i nomi possibili dei
tags come avviene, per esempio, nelle pagine web in \emph{xhtml} dove vi
� una riga preliminare (comunque sempre dopo l'eventuale
\emph{dichiarazione XML}) chiamata \emph{DTD} o \emph{Document Type
  Definition} tipo quella nel codice \ref{cod:dtd}.
\begin{code}
  \begin{xmlblock}
    <!DOCTYPE html PUBLIC "-//W3C//DTD XHTML 1.0 Transitional//EN" "http://www.w3.org/TR/xhtml1/DTD/xhtml1-transitional.dtd">
  \end{xmlblock}
  \caption{Esempio di DTD}\label{cod:dtd}
\end{code}
Se vi � la definizione dello schema allora il documento, oltre ad essere \emph{ben
  formato}, deve essere anche \emph{valido}, ovvero deve
rispettare lo schema definito. I concetti di \emph{schema} e di \emph{XML valido} verranno approfonditi meglio nella sezione
\ref{sec:schemaxml} dove verr\`a introdotto anche il successore del
\emph{DTD}: \emph{XML Schema}.

\subsection{XML Information Set (Infoset)}\label{sec:infoset}
Con il termine \emph{Infoset} ci si riferisce ad un \emph{data set}
astratto che serve per avere un set di definizioni da usare nelle
specifiche, dove si ha bisogno di accedere all'informazione in un
documento \emph{XML} ben formato. Pi\`u emplicemente serve per poterci
riferire in seguito, in questo stesso testo, a certi concetti di un documento
\emph{XML} come i figli di un elemento o il nome di un attributo.

Il \emph{set di informazioni} di un documento \emph{XML} consiste in
un certo numero di \emph{information items}, uno per ogni componente
del documento. Deve esistere almeno l'\emph{information item} del
\emph{documento}.

Esistono \emph{information items} per diversi concetti, ma qui ne
saranno enunciati solo alcuni con le propriet\`a pi\`u importanti,
quelli che saranno necessari allo sviluppo del
testo. Per una trattazione esaustiva, con la lista completa degli
\emph{information items}, si rimanda a
\link{http://www.w3.org/TR/xml-infoset}.

\subsubsection{Information item del documento}
C'\`e un solo \emph{information item} di documento, tutti gli altri
sono accessibili \emph{scendendo} nelle propriet\`a di questo
\emph{information item}. Le propriet\`a sono:
\begin{description}
  \item[figli] contiene una lista ordinata di \emph{information item} di elementi, quelli
    dei figli della radice del documento;
  \item[elemento documento], l'\emph{information item} di tipo elemento
    della radice del documento;
  \item[URI di base] l'uri di base del documento\footnote{Vedi sezione \ref{sec:urienamespaces}
    per una descrizione degli \emph{URI} e dei
    \emph{namespaces}\label{note:urienamespaces}.};
  \item[schema di codifica caratteri];
  \item[versione], la versione del \emph{XML} del documento.
\end{description}
\subsubsection{Information items degli elementi}
Ci sono \emph{information items} di tipo elemento per ogni elemento
del documento. Le sue propriet\`a sono:
\begin{description}
  \item[nome del namespace], se esiste, dell'elemento in questione. Se
    l'elemento non appartiene a nessun namespace allora la propriet\`a
    non ha valore\footnote{Vedi nota \ref{note:urienamespaces}.};
  \item[nome locale], il nome dell'elemento escluso l'eventuale
    prefisso del namespace\footnote{Vedi nota \ref{note:urienamespaces}.};
  \item[prefisso], il prefisso del namespace, se esistente,
    dell'elemento\footnote{Vedi nota \ref{note:urienamespaces}.};
  \item[figli] contiene una lista ordinata di \emph{information item}
    degli elementi figli di questo elemento;
  \item[attributi] contiene una lista non ordinata di \emph{information
    item} degli attributi di questo elemento, esclusi quelli di
    dichiarazione di namespace (\xml{xmlns="..."} e
    \xml{xmlns:nome="..."})\footnote{Vedi nota
      \ref{note:urienamespaces}.};
  \item[attributi namespace] contiene una lista non ordinata di \emph{information
    item} degli attributi di dichiarazione di namespace di questo
    elemento\footnote{Vedi nota \ref{note:urienamespaces}.};
  \item[URI di base], l'uri di base dell'elemento\footnote{Vedi nota
    \ref{note:urienamespaces}.};
  \item[parente], l'\emph{information item} dell'elemento o del
    documento che contiene l'\emph{information item} di questo
    elemento nella sua lista di \emph{figli}.
\end{description}

\subsubsection{Information items degli attributi}
Sono gli \emph{information item} degli attributi che si trovano dentro
gli elementi. Ogni attributo ne ha uno. Le propriet\`a sono:
\begin{description}
  \item[nome del namespace], se esiste, dell'attributo in questione. Se
    l'attributo non appartiene a nessun namespace allora la propriet\`a
    non ha valore\footnote{Vedi nota \ref{note:urienamespaces}.};
  \item[nome locale] il nome dell'attributo escluso l'eventuale
    prefisso del namespace\footnote{Vedi nota \ref{note:urienamespaces}.};
  \item[prefisso] il prefisso del namespace, se esistente,
    dell'elemento\footnote{Vedi nota \ref{note:urienamespaces}.};
  \item[valore normalizzato] il valore dell'attributo
    normalizzato\footnote{Vedere
      \link{http://www.w3.org/TR/REC-xml/\#AVNormalize} per la
      normalizzazione.};
  \item[specifico] un flag che indica se l'attributo \`e definito
    nello \emph{start tag} oppure \`e un elemento di default del
    \emph{DTD}\footnote{Vedi sezione \ref{sec:dtd}
    per una descrizione di \emph{DTD}\label{note:dtd}.};
  \item[tipo di attributo] indica il tipo di attributo indicato nel
    \emph{DTD}\footnote{Vedi nota \ref{note:dtd}.};
  \item[riferimento] se l'attributo \`e di un tipo riferimento, allora
    questa propriet\`a contiene la lista degli \emph{information
      items} degli elementi ai quali si riferisce;
  \item[elemento proprietario] l'\emph{information item} dell'elemento
    che contiene questo \emph{information item} di tipo attributo
    nella sua lista di \emph{attributi}.
\end{description}

\section{Uri e namespaces}\label{sec:urienamespaces}
Un \emph{URI} \`e un
identificatore che permette di indicare in modo univoco una certa
risorsa: come gli \emph{URL} nel web permettono di identificare
univocamente una pagina (o una sezione di una pagina), cos\`i gli
\emph{URI}, che sono una generalizzazione degli \emph{URL}, permettono
di identificare un concetto. Se un certo \emph{URI} compare in posti
diversi allora tutte le sue occorrenze rappresentano la stessa risorsa. La sintassi
degli \emph{URI}\footnote{Secondo l'\emph{internet standard} \emph{STD
    66}, vedere \cite{std66}.}  deve essere
nella seguente forma\footnote{Espressa nella forma \emph{ABNF}
  descritta nello standard \emph{STD 68}, vedere \cite{std68}.}:
\begin{abnfblock}
URI         = scheme ":" hier-part [ "?" query ] [ "#" fragment ]

hier-part   = "//" authority path-abempty
            / path-absolute
            / path-rootless
            / path-empty

authority   = [ userinfo "@" ] host [ ":" port ]
\end{abnfblock}
Dove \abnf{scheme} e  \abnf{path} sono obbligatori, ma il
\abnf{path} pu\`o essere anche vuoto. Quando
\abnf{authority} \`e presente allora il \abnf{path} deve
essere vuoto o cominciare per il carattere \abnf{/}; quando
\abnf{authority} non \`e presente allora \abnf{path} non
pu\`o cominciare con i caratteri \abnf{//}. Brevemente
accenniamo che nella rappresentazione descritta precedentemente tutti
i \abnf{path} sono composti da segmenti delimitati dal carattere
\abnf{/}, inoltre
\abnf{path-abempty} pu\`o essere vuoto o cominciare com \abnf{/},
\abnf{path-absolute} comincia con \abnf{/} ma non con \abnf{//},
\abnf{path-rootless} comincia con un segmento senza \abnf{/} iniziale,
\abnf{path-empty} \`e una sequenza vuota di caratteri. Per una formalizzazione
completa degli \emph{URI} si rimanda a \cite{std66}.

Gli \emph{URI}, oltre che nella suddetta forma, possono presentarsi anche in una
forma \emph{relativa}, a patto che esista un \emph{URI di base} con il
quale poter \emph{risolvere} l'\emph{URI} relativo in un \emph{URI
  target} nella forma vista precedentemente. Questi \emph{URI}
relativi sono nella forma:
\begin{abnfblock}
relative-ref  = relative-part [ "?" query ] [ "#" fragment ]

relative-part = "//" authority path-abempty
              / path-absolute
              / path-noscheme
              / path-empty
\end{abnfblock} 
dove \abnf{path-noscheme} \`e un segmento che non contiene nessun
\abnf{:} seguito da altri segmenti \emph{normali}, sempre separati da
\emph{/}.
  
Esiste un modo pratico per indicare gli \emph{URI} in una forma
maggiormente leggibile e facilmente modificabile: la sintassi in
questione prende il nome di \emph{qnames} (\emph{Qualified Names}) e
in questa sintassi un
\emph{URI} viene rappresentato nella forma
\begin{xmlblock}
prefix:identificatore
\end{xmlblock}
dove \xml{prefix} \`e usato per creare un mapping con un certo
\emph{namespace}, quindi da qualche parte \`e necessario indicare
questa corrispondenza \emph{prefix} - \emph{namespace}.

I \emph{namespaces} non fanno parte del \emph{core} di XML, ma sono
entrati cos\`i tanto nell'uso comune da essere di fatto una
funzionalit\`a di base. Essi servono per creare una sorta di vocabolario per poter
distinguere diverse entit\`a con nome uguale, ad esempio se in un
documento \emph{XML} voglio creare due tag \xml{<num>...</num>}
con significato semantico diverso (il primo \`e
l'identificativo di un cliente e il secondo il numero di dischi venduti
di un cantante) una soluzione pu\`o essere quella di creare due
namespace diversi ed assegnare ad ogni tag \emph{num} uno dei
due. Per fare ci\`o in \emph{XML} si usa lo speciale pseudo attributo
\xml{xmlns} nel modo seguente:
\begin{xmlblock}
<anchestor xmlns:cliente="http://www.example.org/ns/cliente"
xmlns:disco="http://www.example.org/ns/disco">...
\end{xmlblock}
adesso per ogni discendente di \xml{anchestor} esistono i due
\emph{namespaces} indicati ed \`e possibile riferirsi ad essi usando
la sintassi \emph{qname}
\begin{xmlblock}
<cliente:num>...</cliente:num>
...
<disco:num>...</disco:num>
\end{xmlblock}
Oltre che per gli elementi, \`e possibile applicare i namespaces anche
agli argomenti. \`E possibile inoltre indicare un \emph{namespace} di
default usando la forma
\xml{xmlns="http://www.example.org/ns/default"} (senza
specificare il prefisso), in questo modo tutti gli elementi (ed
attributi) senza un prefisso esplicito appartengono a tale
\emph{namespace}.

Vi \`e stato un lungo dibattito sulla questione se trattare i
\emph{namespace} come \emph{URI} e quindi \emph{risolverli} nel caso
in cui siano nella forma di \emph{URI relativi}, oppure se trattarli
puramente come delle stringhe e non risolverli. Il dibattito si \`e
concluso deprecando l'uso degli \emph{URI relativi} (oltre che di
quelli \emph{vuoti}) nei
\emph{namespaces}\footnote{Vedere a tal proposito
  \link{http://www.w3.org/TR/xml-names/\#iri-use}.}.

\section{Linguaggi di schema}\label{sec:schemaxml}
Esistono due linguaggi di schema associati ad XML: il
\emph{DTD} e l'\emph{XML Schema}. Entrambi svolgono la stessa funzione
di base, ossia creare uno schema che fornisce dei limiti alla
struttura logica che un documento XML esprime. La differenza tra i due
consiste in questo: il primo \`e nato con XML e perci\`o \`e supportato
in tutte le versioni ed usa una sintassi basata sulle espressioni
regolari; \emph{XML Schema}, invece, usa una sintassi XML, inoltre
\`e molto pi\`u potente di \emph{DTD} ed usa un ricco set di
dati\footnote{Che sar\`a utile anche in \emph{RDF}, vedere sezione \ref{sec:tipididato}.}.

Se un documento XML prevede anche uno schema, allora oltre ad essere
\emph{ben formato} deve essere anche \emph{valido}. Un documento \`e
\emph{valido} se rispetta le regole e le limitazioni imposte da uno
schema al quale appartiene.

\subsection{DTD}\label{sec:dtd}
Il \emph{Document Type Definition}, \emph{DTD}, consiste in una
grammatica per una certa classe di documenti. In un singolo documento
\`e possibile definire una \emph{document type declaration} che pu\`o
puntare ad un \emph{DTD} esterno, o contenere direttamente le
dichiarazioni che costituiscono un \emph{DTD}, nel codice
\ref{cod:dtd} abbiamo gi\`a visto un esempio di dichiarazione
\emph{DTD} che punta ad un \emph{DTD} esterno al documento.

\`E possibile vedere la definizione formale completa del \emph{DTD} con la
grammatica in notazione \emph{Extended Backus Naur Form}
all'indirizzo: \link{http://www.w3.org/TR/xml/\#sec-prolog-dtd}, comunque
qui accenniamo all'uso generale della dichiarazione del \emph{DTD} in
un documento \emph{XML}
e delle dichiarazioni che compongono il
\emph{DTD} stesso. Queste ultime dichiarazioni possono essere di
quattro tipi:
\begin{itemize}
  \item dichiarazione tipo di elemento;
  \item dichiarazione lista attributi;
  \item dichiarazione entit\`a;
  \item dichiarazione notazione.
\end{itemize}
Definiamo preliminarmente delle entit\`a che useremo nella
formalizzazione di tali dichiarazioni\footnote{In notazione Extended
  Backus Naur Form\label{nota:bnf}.} nel codice \ref{code:dtdcomuni}
\begin{code}
\begin{abnfblock}
NameStartChar   ::=   ":" | [A-Z] | "_" | [a-z] | [#xC0-#xD6] |
                      [#xD8-#xF6] | [#xF8-#x2FF] | [#x370-#x37D] | 
                      [#x37F-#x1FFF] | [#x200C-#x200D] |
                      [#x2070-#x218F] | [#x2C00-#x2FEF] |
                      [#x3001-#xD7FF] | [#xF900-#xFDCF] |
                      [#xFDF0-#xFFFD] | [#x10000-#xEFFFF]

NameChar        ::=   NameStartChar | "-" | "." | [0-9] | #xB7
                      | [#x0300-#x036F] | [#x203F-#x2040]

Name            ::=   NameStartChar (NameChar)*
Names           ::=   Name (#x20 Name)*
Nmtoken         ::=   (NameChar)+
Nmtokens        ::=   Nmtoken (#x20 Nmtoken)*

S               ::=   (#x20 | #x9 | #xD | #xA)+

ExternalID      ::=   'SYSTEM' S SystemLiteral
                    | 'PUBLIC' S PubidLiteral S SystemLiteral
SystemLiteral   ::=   ('"' [^"]* '"') | ("'" [^']* "'")
PubidLiteral    ::=   '"' PubidChar* '"' | "'" (PubidChar - "'")* "'"
PubidChar       ::=   #x20 | #xD | #xA | [a-zA-Z0-9] |[-'()+,./:=?;!*#@$_%]
\end{abnfblock}%$
\caption{Costrutti sintattici comuni}\label{code:dtdcomuni}
\end{code}
dove \abnf{Name} \`e un possibile tipo di nome costituito da uno dei
possibili caratteri di start \abnf{NameStartChar} seguito da un numero
a piacere (anche zero) di possibili caratteri
\abnf{NameChar}. \abnf{Names} \`e una lista di uno o pi\`u nomi
separati da spazi. \abnf{Nmtokens} \`e uguale a \abnf{Name} senza la
limitazione del carattere iniziale (deve comunque essere lungo almeno
un carattere). \abnf{Nmtokens} similmente a \abnf{Names} \`e una lista
di uno o pi\`u \abnf{Nmtoken} separati da spazi. \abnf{S} \`e un tipo
possibile di spazio separatore, comprese tabulazioni e ritorni a
capo. \abnf{ExternalID} rappresenta un \emph{URI} e pu\`o essere di
due tipi: usando l'identificatore \xml{SYSTEM} si indica una stringa
tra virgolette (che non contiene virgolette) o tra apici (che non
contiene apici) che va convertita in un \emph{URI} (la
posizione del \emph{DTD} esterno), usando invece \xml{PUBLIC} si
indicano due stringhe tra virgolette o apici, in quest'ultimo caso il
processore dell'\emph{XML} oltre a cercare lo schema nell'\emph{URI}
rappressentato dalla seconda stringa come nel caso di \xml{SYSTEM}, ha
ulteriori informazioni dalla prima stringa.

La dichiarazione del tipo di documento, all'interno del documento
\emph{XML} nel caso di un \emph{DTD} esterno ad esso, assume la
forma\footnote{Vedi nota \ref{nota:bnf}.} del codice \ref{code:dichiarazionedtd}
\begin{code}
\begin{abnfblock}
doctypedecl  ::=  '<!DOCTYPE' S Name (S ExternalID)? S? ('[' intSubset ']' S?)? '>'
intSubset    ::=  (markupdecl | DeclSep)*
markupdecl   ::=  elementdecl | AttlistDecl | EntityDecl | NotationDecl | PI | Comment
DeclSep      ::=  PEReference | S
\end{abnfblock}
\caption{Dichiarazione del \emph{doctype}}\label{code:dichiarazionedtd}
\end{code}
dove \abnf{Name} deve essere nella forma vista nel codice
\ref{code:dtdcomuni} e deve essere uguale al nodo radice del documento
\emph{XML}, \abnf{ExternalID} \`e un \emph{URI} tra virgolette nella
forma vista nel codice \ref{code:dtdcomuni} e \abnf{intSubset} indica
il subset del \emph{DTD} in questione da prendere in considerazione.
Un esempio di dichiarazione pu\`o essere quella dell'\emph{XHTML 1.1}:
\begin{abnfblock}
<!DOCTYPE html PUBLIC "-//W3C//DTD XHTML 1.1//EN"
  "http://www.w3.org/TR/xhtml11/DTD/xhtml11.dtd">
\end{abnfblock}
che indica che il nodo radice \`e \xml{html} e il tipo di
dichiarazione \`e \xml{PUBLIC} quindi fornisce due stringhe, una con
l'URL del file \emph{DTD} con lo schema e una con ulteriori
informazioni per chi processa questo file \emph{XHTML}. 

Vengono analizzate e formalizzate nel dettaglio le quattro possibili
dichiarazioni \emph{DTD}:

\subsubsection{Dichiarazioni di tipo elemento}
Servono per poter indicare la struttura di un certo elemento, \`e
possibile indicare per ogni singolo elemento quali figli pu\`o
contenere e in che numero e ordine usando una notazione in stile
\emph{espressioni regolari}, oppure se un certo elemento pu\`o
contenere del testo opzionalmente inframezzato da figli. Ogni
dichiarazione di questo tipo assume la forma\footnote{Vedi nota \ref{nota:bnf}.}:
\begin{abnfblock}
elementdecl   ::=   '<!ELEMENT' S Name S contentspec S? '>'
contentspec   ::=   'EMPTY' | 'ANY' | Mixed | children 
\end{abnfblock}
dove:
\begin{description}
  \item[S] \`e una possibile rappresentazione del carattere spazio
    come visto sopra nel codice \ref{code:dtdcomuni};
  \item[Name] \`e una sequenza di caratteri come visto sopra nel
    codice \ref{code:dtdcomuni};
  \item['EMPTY'] indica che il tag deve essere
    vuoto;
  \item['ANY'] che pu\`o contenere qualsiasi contenuto;
  \item[children] \`e una lista di tags che possono essere figli del tag
    in questione. Tale lista \`e espressa secondo la notazione delle
    \emph{espressioni regolari}, ossia pu\`o essere una scelta tra un set di
    valori separati da \xml{|}, una sequenza di valori separati da
    \xml{,} o una combinazione di queste racchiuse tra \xml{(...)} annotate
    con uno tra i caratteri \xml{+} per possibili occorrenze di una o
    pi\`u volte, \xml{*} per zero o pi\`u volte, \xml{?} per zero o una
    volta, se tale carattere non \`e specificato implica che deve
    esserci esattamente una occorrenza.

    Ad esempio la dichiarazione:
    \begin{xmlblock}
<!ELEMENT div1 (head, (p | list | note)*, div2*)>
    \end{xmlblock}
    indica che dentro ogni \xml{div1} ci deve essere un tag \xml{head}
    seguito da una serie lunga a piacere (anche vuota) di tags che
    possono essere scelti tra \xml{p}, \xml{list}, \xml{note} in
    qualsiasi ordine; seguita da una lista sempre lunga a piacere
    (anche vuota) ti tags di tipo \xml{div2}.
  \item[Mixed] indica che il tag pu\`o contenere solo del testo indicato
    con \xml{\#PCDATA}, oppure del testo inframezzato da possibili tags.
    Nel caso di contenuti misti non
    \`e possibile specificare l'ordine della sequenza dei figli, come
    nel caso precedente, ma solo la possibile scelta, inoltre si pu\`o
    solo usare l'operatore \xml{*} o nessun operatore.

    Ad esempio la dichiarazione:
    \begin{xmlblock}
<!ELEMENT p (#PCDATA|a|ul|b|i|em)*>
    \end{xmlblock}
    Indica che i tags di tipo \xml{p} possono contenere del testo
    inframezzato da uno dei possibili tags indicati, in qualsiasi
    ordine e numero, mentre la dichiarazione:
    \begin{xmlblock}
<!ELEMENT b (#PCDATA)>
    \end{xmlblock}
    indica che i tag di tipo \xml{b} possono contenere solo testo e
    non possono essere vuoti.
\end{description}

\subsubsection{Dichiarazioni di tipo \emph{lista di attributi}}
Queste dichiarazioni servono per poter definire il set di attributi
pertinenti ad un certo tipo di tag, porre dei limiti a tali attributi
e fornire dei valori di default per tali attributi. Queste
dichiarazioni sono nella forma\footnote{Vedi nota \ref{nota:bnf}.} del
codice \ref{code:dtdattribute}
\begin{code}
\begin{abnfblock}
AttlistDecl   ::=   '<!ATTLIST' S Name AttDef* S? '>'
AttDef        ::=    S Name S AttType S DefaultDecl

AttType       ::=    StringType | TokenizedType | EnumeratedType
StringType    ::=   'CDATA'
TokenizedType ::=   'ID'
                  | 'IDREF'
                  | 'IDREFS'
                  | 'ENTITY'
                  | 'ENTITIES'
                  | 'NMTOKEN'
                  | 'NMTOKENS'
EnumeratedType ::=    NotationType | Enumeration
NotationType   ::=   'NOTATION' S '(' S? Name (S? '|' S? Name)* S? ')'
Enumeration    ::=   '(' S? Nmtoken (S? '|' S? Nmtoken)* S? ')'

DefaultDecl   ::=   '#REQUIRED' | '#IMPLIED'
                  | (('#FIXED' S)? AttValue)
\end{abnfblock}
\caption{Dichiarazioni di tipo lista di
  attributi}\label{code:dtdattribute}
\end{code}
dove sono costituite da un nome del tag in questione su
cui si vogliono definire i limiti per gli attributi, e una serie di
triple chiamate nella definizione sopra \abnf{AttType} divise da spazi
che indicano tali limiti. Ogni tripla \`e
costituita da un nome dell'attributo per il quale stiamo definendo i
limiti, un tipo di attributo e un valore di default. Il tipo di
attributo, chiamato sopra \abnf{AttType}, pu\`o essere di tipo
\abnf{EnumeratedType} che consiste in una lista di scelte possibili
per il contenuto dell'attributo racchiuse tra parentesi \xml{(...)} e
separate da \xml{|}. Oppure \abnf{AttType} pu\`o essere anche di tipo
stringa indicato con \xml{CDATA} che implica che pu\`o essere scelta
qualsiasi stringa come contenuto di questo attributo oppure uno dei
seguenti tipo con i relativi limiti:
\begin{description}
  \item[ID] deve apparire un solo attributo di questo tipo per tag, il
    valore di tale attributo deve essere una produzione di \abnf{Name} visto
    sopra nel codice \ref{code:dtdcomuni};
    questo attributo identifica univocamente il tag, deve avere indicato come valore
    di default \xml{\#REQUIRED} o \xml{\#IMPLIED};
  \item[IDREF] il valore dell'attributo deve essere una produzione di
    \abnf{Name} visto nel codice \ref{code:dtdcomuni};
  \item[IDREFS] il valore dell'attributo pu\`o essere costituito da
    una lista di valori di tipo \abnf{Name}, cio� deve essere
    produzione di
    \abnf{Names} del codice \ref{code:dtdcomuni};
  \item[ENTITY] deve essere produzione di \abnf{Name} del codice
    \ref{code:dtdcomuni}, inoltre deve essere il nome di una entit\`a dichiarata in una
    dichiarazione di tipo entit\`a;
  \item[ENTITIES] deve essere produzione di \abnf{Names} del codice
    \ref{code:dtdcomuni}, inoltre deve essere una lista di nomi di
    entit\`a dichiarate;
  \item[NMTOKEN] il contenuto di questi attributi deve essere
    produzione di \abnf{Nmtoken} del codice \ref{code:dtdcomuni};
  \item[NMTOKENS] deve essere una lista di \abnf{Nmtoken} separati da
    spazi, ossia deve essere una produzione di \abnf{Nmtokens} del
    codice \ref{code:dtdcomuni}.
\end{description}

Il valore di default, chiamato \abnf{DefaultDecl} nel codice
\ref{code:dtdattribute}, pu\`o essere \xml{\#REQUIRED} che indica che
l'attributo deve essere inserito, \xml{\#IMPLIED} che non viene
fornito nessun valore di default, un valore che rappresenta il valore
di default o un valore preceduto da \xml{\#FIXED} che indica che il
valore di default fornito \`e l'unica scelta possibile per tale
attributo.

\subsubsection{Dichiarazione di entit\`a}
Le \emph{entit\`a} XML sono come quelle di default viste in tabella
\ref{tab:entita} di pagina \pageref{tab:entita}. Con questo tipo di
dichiarazione DTD \`e possibile definire nuove \emph{entit\`a}, in pratica trattasi di tipi di
\emph{placeholder} che vengono sostituiti
all'interno del documento con il testo corrispondente indicato da
queste dichiarazioni. Le dichiarazioni sono nella forma\footnote{Vedi nota \ref{nota:bnf}.} del
codice \ref{code:dtdentita}
\begin{code}
  \begin{abnfblock}
EntityDecl  ::=  GEDecl | PEDecl
GEDecl      ::=  '<!ENTITY' S Name S EntityDef S? '>'
PEDecl      ::=  '<!ENTITY' S '%' S Name S PEDef S? '>'
EntityDef   ::=  EntityValue | (ExternalID NDataDecl?)
PEDef       ::=  EntityValue | ExternalID 

NDataDecl   ::=  S 'NDATA' S Name 
  \end{abnfblock}
  \caption{Dichiarazioni di entit\`a}
  \label{code:dtdentita}
\end{code}
dove \abnf{Name} \`e il nome dell'entit\`a e il valore dell'entit\`a
pu\`o essere direttamente un valore racchiuso tra virgolette
(\abnf{EntityValue}) oppure un identificatore esterno
(\abnf{ExternalID} vedi codice \ref{code:dtdcomuni}) con una
dichiarazione opzionale del tipo di dato
dell'identificatore esterno(\abnf{NDataDecl}), in pratica un uri
esterno che pu\`o essere risolto.

\subsubsection{Dichiarazione di notazione}
Le dichiarazioni di notazione sono nella forma:
\begin{abnfblock}
NotationDecl  ::=  '<!NOTATION' S Name S (ExternalID | PublicID) S? '>'
PublicID      ::=  'PUBLIC' S PubidLiteral
\end{abnfblock}
e servono per definire un formato ad una risorsa che non \`e n\'e testo
n\'e XML. Le notazioni indicate possono essere usate in dichiarazioni di
entit\`a e liste di attributi.

\subsection{XML Schema}\label{sec:xmlschema}
\emph{XML Schema} svolge la stessa funzione di \emph{DTD}, \`e un
linguaggio che fornisce la possibilit\`a di creare
uno schema per definire la struttura di un documento \emph{XML}. \`E
possibile definire inoltre le relazioni tra gli \emph{elementi} e i loro
contenuti e gli \emph{attributi} e i loro valori, fornisce inoltre un
ricco set di \emph{tipi di dato}. \emph{XML Schema}, inoltre, usa una
sintassi basata su \emph{XML} e fornisce un \emph{namespace}
\xml{http://www.w3.org/2001/XMLSchema} per la
definizione degli elementi che compongono lo schema. \`E convenzione
chiamare tale namespace con il prefisso \xml{xsd} cos\`i come
l'estensione del file dello schema, oppure anche con il pi\`u conciso
\xml{xs}. Quindi nel primo tag \`e bene 
definire:
\begin{xmlblock}
xmlns:xs="http://www.w3.org/2001/XMLSchema"
\end{xmlblock}

Nel file \emph{XML} a cui deve essere assegnato lo \emph{schema}
bisogna usare il namespace
\xml{http://www.w3.org/2001/XMLSchema-instance} a cui
convenzionalmente si associa il prefisso xsi.
Il compito di assegnare lo schema ad un certo documento \`e del
\emph{processore} del documento \emph{XML}, come per il \emph{DTD},
proprio come quest'ultimo viene fornito un set di convenzioni per
fornire le informazioni al \emph{processore} per effettuare
l'associazione \emph{documento-schema}. Per questo esistono gli
attributi \xml{xsi:schemaLocation} e \xml{noNamespaceSchemaLocation},
nel primo \`e possibile indicare come valore di tale attributo una
serie di coppie \emph{namespace} - \emph{URI} dello schema, tali
valori sono tutti separati da spazi e ogni coppia rappresenta un
namespace e un percorso dove trovare uno schema da usare per tale
namespace. Se in un documento XML viene definito il primo
elemento come nel codice \ref{code:dichiarazionexmlschema}
\begin{code}
\begin{xmlblock}
<stylesheet xmlns="http://www.w3.org/1999/XSL/Transform"
            xmlns:html="http://www.w3.org/1999/xhtml"
            xmlns:xsi="http://www.w3.org/2001/XMLSchema-instance"
            xsi:schemaLocation="http://www.w3.org/1999/XSL/Transform
                                http://www.w3.org/1999/XSL/Transform.xsd
                                http://www.w3.org/1999/xhtml
                                http://www.w3.org/1999/xhtml.xsd">
\end{xmlblock}
\caption{Dichiarazione \emph{XML SCHEMA} dentro un documento}
\label{code:dichiarazionexmlschema}
\end{code}
per gli elementi associati al namespace di
default(\xml{http://www.w3.org/1999/XSL/Transform}) viene applicato
lo schema definito dal documento che si trova all'indirizzo
\xml{http://www.w3.org/1999/XSL/Transform.xsd}; per quelli associati
al namespace \xml{html} (\xml{http://www.w3.org/1999/xhtml}) si applica lo schema
in \xml{http://www.w3.org/1999/xhtml.xsd}. Esiste inoltre l'attributo
\xml{xsi:noNamespaceSchemaLocation} che, invece di avere una lista di
coppie di valori, ha una lista di valori singoli, senza che venga specificato il
\emph{namespace} degli elementi a cui viene applicato tale schema.

Uno schema \`e costituito da componenti che possono essere di diverso
tipo:
\begin{itemize}
  \item componenti definizione di tipo;
  \item componenti dichiarazione;
  \item componenti definizione di gruppi attributo;
  \item componenti gruppo modello e definizione di gruppi modello;
  \item componenti definizione \emph{identity-constraints};
  \item componenti annotazione.
\end{itemize}
Di seguito vengono analizzati nel dettaglio tali componenti e, inoltre,
viene dedicata una sezione anche alla descrizione del ricco \emph{set
  di dati} predefiniti che fornisce \emph{XML SCHEMA}.

I componenti sono definiti in termini delle sue propriet\`a, e per
ogni propriet\`a definendo la sua immagine. Si pu\`o immaginare uno
schema come un grafo orientato con radice dove questa radice \`e lo
schema stesso, ogni nodo \`e un componente o un letterale e ogni arco
\`e una propriet\`a. 

\subsubsection{Tipo di dato}\label{sec:datasetxmlschema}
In \emph{XML SCHEMA} un tipo di dato \`e costituito da una tripla
consistente di:
\begin{description}
  \item[spazio valori], un set di valori distinti che sono l'insieme
    dei valori per un certo tipo di dato. Ogni valore corrisponde ad
    uno o pi\`u letterali dello \emph{spazio lessicale}.

    Questo spazio pu\`o essere definito:
    \begin{itemize}
      \item in modo assiomatico;
      \item enumerandolo;
      \item restringendo lo \emph{spazio valori} di un tipo di
        dato gi\`a esistente, partendo da dei tipi di dato primitivi o da
        altri gi\`a \emph{derivati};
      \item come combinazione di valori da uno o pi\`u \emph{spazi
        valori} gi\`a esistenti;
    \end{itemize}

  \item[spazio lessicale], un set di \emph{letterali validi} per un
    certo tipo di dato. Spesso c'\`e una corrispondenza uno ad uno tra
    gli elementi dello \emph{spazio lessicale} e gli elementi dello
    \emph{spazio valori}, ma non sempre \`e cos\`i, a volte un singolo
    valore pu\`o essere rappresentato con diversi modi lessicali: ad esempio \xml{100} e \xml{1.0E2} sono due
    letterali diversi dello \emph{spazio lessicale} del tipo di dato
    \emph{float} che denotano entrambi lo stesso valore dello
    \emph{spazio valori};
  \item[facets], letteralmente sfaccettature, sono singoli aspetti che
    definiscono uno spazio, ogni \emph{facet} caratterizza uno
    \emph{spazio valori} su un'asse o dimensione
    indipendente. Esistono \emph{facets fondamentali} e \emph{facets
      non fondamentali}, i primi sono delle proriet\`a astratte che
    servono per caratterizzare semanticamente i valori di uno
    \emph{spazio valori} e queste sono tutte le seguenti
    autoesplicative:
    \begin{itemize}
      \item equal;
      \item ordered;
      \item bounded;
      \item cardinality;
      \item numeric.
    \end{itemize}
    I \emph{facets non fondamentali} sono propriet\`a opzionali che
    servono per restringere lo \emph{spazio valori}, questi sono:
    \begin{itemize}
      \item length;
      \item minLength;
      \item maxLength;
      \item pattern;
      \item enumeration;
      \item whiteSpace;
      \item maxInclusive;
      \item maxExclusive;
      \item minExclusive;
      \item minInclusive;
      \item totalDigits;
      \item fractionDigits.
    \end{itemize}
    Quest'ultimi \emph{facets non fondamentali} non si applicano
    indiscriminatamente a tutti i tipi di dato, ma ogni tipo primitivo
    e i suoi derivati ha una possibile scelta di \emph{facets} che
    possono essere usati.

    Per una trattazione esaustiva dei \emph{facets primitivi} e
    \emph{non fondamentali} si rimanda a
    \link{http://www.w3.org/TR/xmlschema-2/\#rf-fund-facets} e a
    \link{http://www.w3.org/TR/xmlschema-2/\#rf-facets}.
\end{description}

In figura \ref{fig:tipixmlschema}
\image{img/xmlschema-type-hierarchy.eps}{Gerarchia dei tipi built-in}{fig:tipixmlschema}
\`e possibile vedere tutta la vasta gamma dei tipi predefiniti
dell'\emph{XML-SCHEMA}. La \emph{derivazione per restrizione} consiste
nel derivare un tipo da un altro restringendo lo \emph{spazio dei
  valori} con l'uso di \emph{facets non fondamentali}. La \emph{derivazione
per lista} si ottiene creando uno spazio valori che consiste nelle
sequenze di lunghezza finita di valori di un altro tipo. Anche se non
\`e visibile nei tipi di dato built-in, esiste la \emph{derivazione per
unione} che consiste nel creare uno \emph{spazio dati} unendo gli
\emph{spazi dati} (e conseguentemente gli \emph{spazi lessicali}) di
altri tipi di dato.

\emph{XML SCHEMA} fornisce un \emph{namespace} di default per tutti i
tipi di dato built in e tutti i facets visti:
\begin{xmlblock}
http://www.w3.org/2001/XMLSchema
\end{xmlblock}
in modo che possano essere indirizzati aggiungendo il nome del tipo di
dato o del facet come \emph{frammento} dell'\emph{URI} nel modo
seguente:
\begin{xmlblock}
http://www.w3.org/2001/XMLSchema#int
http://www.w3.org/2001/XMLSchema#maxInclusive
\end{xmlblock}
Inoltre per i tipi di dato viene fornito anche il namespace:
\begin{xmlblock}
http://www.w3.org/2001/XMLSchema-datatypes
\end{xmlblock}

\subsubsection{Componenti definizione di tipo}
I \emph{componenti definizione di tipo} servono per definire i tipi di
dato come quelli visti nella sezione
\ref{sec:datasetxmlschema}. Esistono due tipi di definizioni di tipo:
tipo semplice e tipo complesso.

I componenti per la definizione dei tipi semplici sono una serie di restrizioni
sulle stringhe e informazioni sui valori che rappresentano applicabili
ai valori di un attributo XML o ai contenuti testuali di un
elemento. Ogni definizione di tipo semplice, sia di quelle predefinite
del \emph{dataset} dell'\emph{XML SCHEMA}\footnote{Vedere sezione
  \ref{sec:datasetxmlschema}.}, sia creata dall'utente, sono delle
restrizioni su delle particolari definizioni di tipo di base tramite i
facets. Ad esempio nel codice \ref{code:esempiotiposemp}
\begin{code}
  \begin{xmlblock}
<xs:simpleType name="civico">
  <xs:restriction base="xs:string">
    <xs:pattern value="[0-9]+(N|R)?(/[a-z]+)?"/>
  </xs:restriction>
</xs:simpleType>
  \end{xmlblock}
  \caption{Esempio di componente definizione di tipo semplice}
  \label{code:esempiotiposemp}
\end{code}
viene definito un nuovo tipo \xml{civico} restringendo lo \emph{spazio
  lessicale} (e quindi lo \emph{spazio valori}) di \xml{string} con un
pattern definito con una
\emph{espressione regolare}.

I \emph{tipi complessi} permettono elementi e attributi nel suo
contenuto. Sono definiti usando l'elemento
\begin{xmlblock}
xsi:complexType
\end{xmlblock}
e di solito contengono un set di definizioni di elementi, nomi di
elementi e dichiarazioni di attributi, ad esempio come nel codice
\ref{code:esempiotipocomp}.
\begin{code}
  \begin{xmlblock}
<xs:complexType name="Indirizzo" >
  <xs:sequence>
    <xs:element name="via" type="xs:string"/>
    <xs:element name="numero"   type="civico"/>
    <xs:element name="citta"   type="xs:string"/>
    <xs:element name="cap"  type="xs:decimal"/>
  </xs:sequence>
  <xs:attribute name="stato" type="xs:NMTOKEN" fixed="IT"/>
</xs:complexType>
  \end{xmlblock}
  \caption{Esempio di componente definizione di tipo complesso}
  \label{code:esempiotipocomp}
\end{code}
Si noti che in questo codice \`e stato usato come tipo del numero
civico quello definito prima nel codice \ref{code:esempiotiposemp}.

Per una trattazione completa dei componenti di definizione di tipo
semplice si rimanda a\\
\link{http://www.w3.org/TR/xmlschema-1/\#Simple\_Type\_Definitions} e a\\
\link{http://www.w3.org/TR/xmlschema-2/\#rf-defn},\\
 per i tipi complessi si rimanda invece a\\
\link{http://www.w3.org/TR/xmlschema-1/\#Complex\_Type\_Definitions}.

\subsubsection{Componenti dichiarazione}
Le dichiarazioni possono essere di \emph{attributi} o \emph{elementi},
le \emph{dichiarazioni di attributi} servono per fornire la
validazione degli attributi usando una definizione di tipo semplice,
inoltre permettono anche di specificare dei valori di default o fissi
per il valore dell'attributo. Ad esempio
\begin{xmlblock}
<xs:attribute name="eta" type="xs:positiveInteger" use="required"/>
\end{xmlblock}
indica che l'attributo \xml{eta} \`e obbligatorio e deve essere un
intero positivo.

Le dichiarazioni di \emph{elementi} servono per fornire la validazione
degli elementi usando una definizione di tipo, definire valori di
default o fissi per l'\emph{information item} di un
elemento\footnote{Vedere sezione \ref{sec:infoset}.}),
stabilire criteri di univocit\`a e relazioni tra i valori degli elementi
e attributi correlati. Ad esempio il codice
\ref{code:esempiodichiarazioneelem}
\begin{code}
  \begin{xmlblock}
<xs:element name="PurchaseOrder" type="PurchaseOrderType"/>

<xs:element name="gift">
 <xs:complexType>
  <xs:sequence>
   <xs:element name="birthday" type="xs:date"/>
   <xs:element ref="PurchaseOrder"/>
  </xs:sequence>
 </xs:complexType>
</xs:element>
  \end{xmlblock}
  \caption{Esempio di dichiarazione elemento}
  \label{code:esempiodichiarazioneelem}
\end{code}
definisce un elemento chiamato \xml{PurchareOrder} di tipo
\xml{PurchaseOrderType} e un altro elemento di nome \xml{gift} che
deve avere come figli un elemento \xml{birthday} di tipo \xml{date} e
un elemento di tipo \xml{PurchaseOrder} definito sopra.

Per una trattazione esaustiva sui componenti dichiarazione si rimanda
a\\
 \link{http://www.w3.org/TR/xmlschema-1/\#cAttribute\_Declarations} e a\\
 \link{http://www.w3.org/TR/xmlschema-1/\#cElement\_Declarations}.

\subsubsection{Componenti definizione di gruppi attributo}
\`E possibile definire dei gruppi di dichiarazioni di \emph{attributi}
per poi poterli riutilizzare all'interno di dichiarazioni di tipo
complesse, ad esempio il codice \ref{code:esempiogruppoattributo}
\begin{code}
  \begin{xmlblock}
<xs:attributeGroup name="myAttrGroup">
    <xs:attribute . . ./>
    . . .
</xs:attributeGroup>

<xs:complexType name="myelement">
    . . .
    <xs:attributeGroup ref="myAttrGroup"/>
</xs:complexType>
  \end{xmlblock}
  \caption{Esempio di definizione di gruppo di attributi}
  \label{code:esempiogruppoattributo}
\end{code}
raggruppa delle definizioni di attributo dentro il gruppo
\xml{myAttrGroup} e poi lo usa per definire il tipo \emph{myelement}.

Si rimanda a:\\
\link{http://www.w3.org/TR/xmlschema-1/\#cAttribute\_Group\_Definitions}\\
per una trattazione completa.

\subsubsection{Componenti gruppo modello e definizione di gruppi modello}
I \emph{gruppi modello} servono per poter definire nel dettaglio la
sequenza degli elementi figlio. Possono essere di tre tipi e possono
essere anche concatenati, per questo ai contenuti dei \emph{gruppi
  modello} ci si riferisce con il termine \emph{particelle} che
possono essere dichiarazioni di elemento o anche altri \emph{gruppi
  modello}. Questi tre tipi sono:
\begin{description}
  \item[sequence] corrisponde alle \emph{particelle} indicate
    nell'ordine indicato;
  \item[choice] corrisponde ad una delle particelle indicate;
  \item[all] corrisponde a tutte le \emph{particelle} indicate in
    qualsiasi ordine.
\end{description}
Abbiamo gi\`a visto un \emph{gruppo modello sequence} nel codice
\ref{code:esempiodichiarazioneelem}.

Inoltre \`e possibile definire dei \emph{gruppi modello} con
\xml{xs:group} in modo da poterli inserire per riferimento in altri
posti. Nel codice \ref{code:esempiogruppomodello}
\begin{code}
  \begin{xmlblock}
<xs:group name="myModelGroup">
 <xs:sequence>
  <xs:element ref="someThing"/>
  . . .
 </xs:sequence>
</xs:group>

<xs:complexType name="trivial">
 <xs:group ref="myModelGroup"/>
 <xs:attribute .../>
</xs:complexType>
  \end{xmlblock}
  \caption{Esempio di definizione di gruppo modello}
  \label{code:esempiogruppomodello}
\end{code}
\`e possibile vedere un esempio di creazione di un gruppo modello con
una \emph{sequence} chiamato \xml{myModelGroup}, e poi usato con
riferimento dentro la dichiarazione di un tipo di dato complesso
chiamato \xml{trivial}.

Per chiarezza, le \emph{particelle} che abbiamo citato prima (il
contenuto dei gruppi modello) oltre a poter essere delle dichiarazioni
di elemento o dei gruppi modello a sua volta, esse possono essere
anche delle \emph{wildcard}. Le \emph{wildcard} possono
\emph{matchare} \emph{information items} di elementi e attributi
dipendentemente dai loro nomi di namespace e indipendentemente dai
loro nomi locali\footnote{Vedere sezione \ref{sec:infoset}.}. 

Per una trattazione esaustiva dei \emph{gruppi modello} si rimanda a\\
\link{http://www.w3.org/TR/xmlschema-1/\#Model\_Groups},\\
mentre per la dichiarazione di \emph{gruppi modello}\\ 
\link{http://www.w3.org/TR/xmlschema-1/\#cModel\_Group\_Definitions}.

\subsubsection{Componenti definizione \emph{identity-constraints}}
Questi componenti forniscono la possibilit\`a di definire un sistema
di riferimenti e univocit\`a degli elementi o attributi di dove sono
definiti. Esistono tre tipi di questi componenti e tutti richiedono
che sia specificato un \xml{selector} e almeno un \xml{field}, oltre
che un \xml{name} con cui identificare il componente:
\begin{description}
  \item[unique] garantisce l'univocit\`a di tutte le tuple risultanti
    dalla valutazione dei campi \xml{field} rispetto al campo
    \xml{selector}. Nei campi \xml{field} vi \`e indicata una
    espressione \emph{XPath}\footnote{Vedere sezione \ref{sec:xpath}
      per le espressioni \emph{XPath}.} con cui effettuare la valutazione in
    questione;
  \item[key] oltre a garantire l'univocit\`a come per \emph{unique},
    assicura che le tuple risultanti dalla valutazione effettivamente
    siano presenti;
  \item[keyref] assicura una corrispondenza tra le tuple risultanti
    dalla valutazione e quelle identificate dal componente indicato
    nel campo \xml{refer}.
\end{description}

Per meglio comprendere prendiamo ad esempio il codice \emph{XML}
\ref{code:esempioidconst} e il relativo codice dello schema
\ref{code:esempioidconstschema},
\begin{code}
  \begin{xmlblock}
<dipartimento xmlns:xsi="http://www.w3.org/2001/XMLSchema-instance"
       xsi:noNamespaceSchemaLocation="schema.xsd">
  <docenti>
    <docente codiceCorso="386">
      <nome>Mario Rossi</nome>
      <ruolo valore="Professore"/>
    </docente>
    <docente codiceCorso="103">
      <nome>Paolo Verdi</nome>
    </docente>
    <assistente codiceCorso="386">
      <nome>Franco Neri</nome>
      <ruolo valore="Ricercatore"/>
    </assistente>
  </docenti>
  <corsi>
    <corso>
      <codice>386</codice>
      <nome>Informatica</nome>
      <cfu>12</cfu>
    </corso>
    <corso>
      <codice>103</codice>
      <nome>Analisi matematica</nome>
      <cfu>12</cfu>
    </corso>
  </corsi>
</dipartimento>
  \end{xmlblock}
  \caption{Esempio documento XML}
  \label{code:esempioidconst}
\end{code}
\begin{code}
  \begin{xmlblock}
<xs:schema xmlns:xs="http://www.w3.org/2001/XMLSchema">
  <xs:element name="dipartimento" type="DipartimentoType">
    <xs:keyref name="corsoKeyRef" refer="corsoKey">
      <xs:selector xpath="dipartimento/*"/>
      <xs:field xpath="@codiceCorso"/>
    </xs:keyref>
    <xs:key name="corsoKey">
      <xs:selector xpath=".//corso"/>
      <xs:field xpath="codice"/>
    </xs:key>
  </xs:element>
  <xs:complexType name="DipartimentoType">
    <xs:sequence>
      <xs:element name="docenti" type="DocentiType"/>
      <xs:element name="corsi" type="CorsiType"/>
    </xs:sequence>
  </xs:complexType>
  <xs:complexType name="DocentiType">
    <xs:choice maxOccurs="unbounded">
      <xs:element name="docente" type="DocenteType"/>
      <xs:element name="assistente" type="DocenteType"/>
    </xs:choice>
  </xs:complexType>
  <xs:complexType name="DocenteType">
    <xs:sequence>
      <xs:element name="nome" type="xs:string"/>
      <xs:element name="ruolo" type="RuoloType" minOccurs="0"/>
    </xs:sequence>
    <xs:attribute name="codiceCorso" type="xs:integer"/>
  </xs:complexType>
  <xs:complexType name="CorsiType">
    <xs:sequence>
      <xs:element name="corso" type="CorsoType"
                   maxOccurs="unbounded"/>
    </xs:sequence>
  </xs:complexType>
  <xs:complexType name="CorsoType">
    <xs:sequence>
      <xs:element name="codice" type="xs:integer"/>
      <xs:element name="nome" type="xs:string"/>
      <xs:element name="cfu" type="xs:integer"/>
    </xs:sequence>
  </xs:complexType>
  <xs:complexType name="RuoloType">
    <xs:attribute name="valore" type="xs:string"/>
  </xs:complexType>
</xs:schema>
  \end{xmlblock}
  \caption{Esempio di schema per il codice \ref{code:esempioidconst}}
  \label{code:esempioidconstschema}
\end{code}
oltre alla definizione dei componenti di definizione
\emph{identity-constraints} \`e possibile vedere anche i componenti di
dichiarazione degli elementi e degli attributi.

Come si pu\`o vedere dal codice, oltre alla definizione dello schema
degli elementi e attributi, viene usato un componente \xml{xs:key}
chiamato \xml{corsoKey} che definisce l'esistenza e l'univocit\`a di
elementi \xml{codice} figli di \xml{corso}. Viene inoltre dichiarata
una \xml{xs:keyref} di nome \xml{corsoKeyRef} con riferimento a
\xml{corsoKey} che garantisce che i valori degli attributi
\xml{codiceCorso} dei discendenti di \xml{dipartimento} siano
corrispondenti a quelli indicati nel componente \xml{corsoKey}.

Si ricorda che per comprendere le espressioni \emph{XPath} si rimanda
alla sezione \ref{sec:xpath}.

\subsubsection{Componenti annotazione}
Questi componenti servono per creare delle annotazioni adatte al
consumo umano o anche da parte di un programma. Queste annotazioni non
sono 
necessarie al funzionamento dello schema, ma sono utili per tenere traccia
delle revisioni o commentare delle parti del codice di uno
schema o anche per fornire ad un programma delle informazioni
su un componente, ad esempio come trattare un tipo di dato.

Queste annotazioni possono essere di tre tipi e sono tutti e tre
visibili nel codice di esempio \ref{code:esempioannotazioni}
\begin{code}
  \begin{xmlblock}
<xs:simpleType fn:note="special">
  <xs:annotation>
   <xs:documentation>A type for experts only</xs:documentation>
   <xs:appinfo>
    <fn:specialHandling>checkForPrimes</fn:specialHandling>
   </xs:appinfo>
  </xs:annotation>
</xs:simpleType>
  \end{xmlblock}
  \caption{Esempio di annotazioni}
  \label{code:esempioannotazioni}
\end{code}
tutte all'interno di una dichiarazione di un tipo di dato
semplice. \xml{xs:annotation} \`e il \emph{contenitore} delle
annotazioni e al suo interno pu\`o contenere gli altri due componenti:
\xml{xs:documentation} inteso per un consumo umano e \xml{xs:appinfo}
inteso per un consumo automatico.

\`E possibile anche fornire delle note usando un namespace diverso da
quello dello schema, sia all'interno di un componente annotazione, sia
per l'elemento che racchiude l'annotazione (come per \xml{fn:note}
nell'esempio).

\section{Famiglia XSL}\label{sec:xsl}
\emph{XSL} sta per \emph{Extensible Stylesheet Language} e, come
indica il nome, lo scopo della sua esistenza \`e quello di fornire un
metodo per creare dei fogli di stile per i documenti \emph{XML}. 
Con \emph{XSL} \`e possibile dividere il contenuto dei documenti
\emph{XML} dalla sua presentazione, il modo in cui vengono
visualizzati.

Per funzionare, \emph{XSL} ha bisogno di un \emph{processore XSL} che
porti a completamento il procedimento di creare una presentazione per
un documento \emph{XML} (o anche pi\`u documenti). Tale
\emph{processore} esegue due passaggi (tralasciando i
passaggi intermedi non utili alla comprensione), partendo dal
documento \emph{XML} iniziale effettua una \emph{trasformazione} in un altro
documento \emph{XML} formattato secondo un linguaggio chiamato
\emph{XSL-FO}, \emph{XSL Formatting Objects}, poi effettua una
\emph{formattazione} di tale documento \emph{XSL-FO} interpretandolo e
visualizzandolo, o stampandolo, o qualunque cosa a seconda del
dispositivo che effettua tale passaggio. Quest'ultimo passaggio in realt\`a
\`e composto da diversi sottopassaggi concatenati che non sono utili
per lo scopo di questo testo\footnote{Si veda
  \link{http://www.w3.org/TR/xsl/\#d0e209} per una introduzione a tali
  sottopassaggi.}.

Per poter fare la prima \emph{trasformazione} \`e necessario dover
fornire uno schema da seguire, per creare tale schema esiste un
linguaggio apposta: \emph{XSLT}, \emph{XSL Transformation}. I
documenti in formato \emph{XSLT} prendono il nome di
\emph{stylesheet}. 

Come spesso accade nel mondo dell'\emph{XML}, cose che sono create con un
certo scopo finiscono con l'essere utili anche in altri ambiti, cos\`i
l'\emph{XSLT} \`e talmente flessibile da poter essere usato per
trasformare un \emph{XML} in qualsiasi tipo di documento, non solo in
documenti \emph{XML-FO}, anche in documenti non \emph{XML}. Per questo
esistono dei \emph{processori XSLT} che, di fatto, implementano solo la
prima fase del \emph{processore XSL} visto prima, prendendo come input
il documento \emph{XML} da trasformare e lo stylesheet \emph{XSLT} e
fornendo in output il risultato della trasformazione. 

In figura \ref{fig:processorexsl}
\begin{figure}
  \begin{center}
    \begin{tikzpicture}[x=10cm, y=4cm]
      \node (xml) at (0,1) [file] {Documento XML};
      \node (xslt) at (0.25,0) [file] {Documento XSLT};
      \node (xslfo) at (0.5,1) [file] {Documento XSL-FO};
      \node (dispo) at (1,1) [disp] {Dispositivo};
      \node (trans) at (0.25,0.5) [app, text width = 2.5cm] {Trasformazione};
      \node (format) at (0.75,0.5) [app, text width = 2.5cm] {Formattazione};
      \draw [freccia] (xml) to (trans);
      \draw [freccia] (xslt) to (trans);
      \draw [freccia] (trans) to (xslfo);
      \draw [freccia] (xslfo) to (format);
      \draw [freccia] (format) to (dispo);
    \end{tikzpicture}
  \end{center}
  \caption{Schema del processore XSL}
  \label{fig:processorexsl}
\end{figure}
\`e possibile vedere uno schema del funzionamento di un
\emph{processore XSL}.
 
Per il suo funzionamento \emph{XSLT} usa un linguaggio per
identificare parti del documento \emph{XML} da trasformare. Questo
linguaggio prende il nome di \emph{XPath}. Nello specifico viene usato
per \emph{pescare} gli elementi o gli attributi, anche a blocchi dal
documento \emph{XML}, per poi poter decidere cosa farne e come devono
essere visualizzati nella trasformazione.

Di seguito vengono analizzati nel dettaglio \emph{XPath} e
\emph{XSLT}. \emph{XSL-FO} viene tralasciato in questo testo in quanto
si tratta di un vocabolario non molto usato e 
non \`e necessario per gli scopi di questo scritto.

\subsection{XPath}\label{sec:xpath}
Lo scopo principale di \emph{XPath} \`e quello di indirizzare parti di
documenti \emph{XML}. Pur essendo nato all'interno della famiglia
\emph{XSL}, e pi\`u precisamente all'interno di \emph{XSLT}, esso \`e
abbastanza versatile da poter essere usato anche in altri ambiti. Ad
esempio esistono librerie \emph{Java} che permettono di accedere a
documenti \emph{XML} tramite \emph{XPath}.

\emph{XPath} \`e formulato in modo da basarsi sulla \emph{struttura
  logica} di un \emph{XML} pi\`u che sulla struttura sintattica. Un
documento \emph{XML} viene interpretato come un \emph{albero} di
\emph{nodi} dove ogni nodo pu\`o essere di tipo \emph{elemento},
\emph{attributo}, \emph{testo}. Ognuno di questi nodi viene valutato
come un valore testuale, qualche tipo di nodo ha anche un nome.

Il modo in cui \emph{XPath} rappresenta il documento pu\`o essere
derivato dagli \emph{information items} forniti da \emph{Infoset}
(vedere sezione \ref{sec:infoset}). Per dettagli su questa derivazione
si rimanda all'indirizzo \link{http://www.w3.org/TR/xpath/\#infoset}.

\emph{XPath}, a differenza degli altri due linguaggi di \emph{XSL},
non usa una sintassi \emph{XML}, bens\`\i consiste in una
\emph{espressione} la cui valutazione ritorna degli indirizzi di punti
precisi del documento. Cosa fare con questi indirizzi dipende da chi
usa \emph{XPath}, ad esempio usato nelle \emph{XSLT} \`e possibile
prendere dei contenuti dagli attributi o elementi da poter portare nel
documento risultante della trasformazione.

\emph{XPath} supporta anche i \emph{namespaces}, quindi i nomi dei
nodi consistono in una \emph{parte locale} e un \emph{URI} del
\emph{namespace} eventualmente vuoto.

Pi\`u precisamente la valutazione dell'\emph{espressione} ritorna uno
dei seguenti possibili \emph{oggetti}:
\begin{description}
  \item[node-set] una collezione non ordinata di nodi senza duplicati;
  \item[booleano] vero o falso;
  \item[numero] un numero floating point;
  \item[stringa] una sequenza di caratteri.
\end{description}

La valutazione dell'espressione avviene relativamente ad un \emph{contesto}
che deve essere deciso da chi usa \emph{XPath} e richiama
l'espressione. Tale \emph{contesto} consiste in:
\begin{description}
  \item[\emph{context node}] un nodo che identifica il punto dal quale
    valutare una certa espressione;
  \item[\emph{context position} e \emph{Context size}] due interi
    positivi tali che la \emph{context position} \`e sempre minore o
    uguale della \emph{context size};
  \item[\emph{variable bindings}] un set che consiste in un
    \emph{mapping} da nomi di variabile a valori di variabile. I
    valori possono essere del tipo dei risultati
    dell'espressione visti prima, o anche di altri tipi;
  \item[\emph{function library}] un set che consiste in un
    \emph{mapping} da nomi di funzione a funzioni. Esiste una libreria
    di \emph{core} che tutte le implementazioni di \emph{XPath}
    implementano, inoltre gli utilizzatori di \emph{XPath} possono
    estendere tale set con altre funzioni personalizzate;
  \item[\emph{namespace declaration}] consiste in un mapping dai
    prefissi dei \emph{namespace} e i relativi \emph{URI}.
\end{description}

I \emph{variable bindings}, la \emph{function library} e i
\emph{namespace declaration} rimangono costanti per tutta
l'espressione. Il \emph{context node}, la \emph{context position} e la
\emph{context size} possono variare nella valutazione di
\emph{sottoespressioni} rispetto all'\emph{espressione contenitrice}.

Una espressione deve essere nella forma\footnote{In forma
  \emph{Extended Backus Naur Form}.} di \abnf{Expr} del seguente codice:%codice \ref{code:sintassixpath}
%\begin{code}
  \begin{abnfblock}
Expr   ::=   OrExpr

OrExpr         ::=   AndExpr
                   | OrExpr 'or' AndExpr
AndExpr        ::=   EqualityExpr
                   | AndExpr 'and' EqualityExpr
EqualityExpr   ::=   RelationalExpr
                   | EqualityExpr '=' RelationalExpr
                   | EqualityExpr '!=' RelationalExpr
RelationalExpr ::=   AdditiveExpr
                   | RelationalExpr '<' AdditiveExpr
                   | RelationalExpr '>' AdditiveExpr
                   | RelationalExpr '<=' AdditiveExpr
                   | RelationalExpr '>=' AdditiveExpr

AdditiveExpr       ::=   MultiplicativeExpr
                       | AdditiveExpr '+' MultiplicativeExpr
                       | AdditiveExpr '-' MultiplicativeExpr
MultiplicativeExpr ::=   UnaryExpr
                       | MultiplicativeExpr MultiplyOperator UnaryExpr
                       | MultiplicativeExpr 'div' UnaryExpr
                       | MultiplicativeExpr 'mod' UnaryExpr
UnaryExpr          ::=   UnionExpr
                       | '-' UnaryExpr

UnionExpr      ::=   PathExpr
                   | UnionExpr '|' PathExpr
PathExpr       ::=   LocationPath
                   | FilterExpr
                   | FilterExpr '/' RelativeLocationPath
                   | FilterExpr '//' RelativeLocationPath
FilterExpr     ::=   PrimaryExpr
                   | FilterExpr Predicate

LocationPath         ::= RelativeLocationPath
                       | AbsoluteLocationPath
AbsoluteLocationPath ::= '/'RelativeLocationPath?
                       | AbbreviatedAbsoluteLocationPath
RelativeLocationPath ::= Step
                       | RelativeLocationPath '/' Step
                       | AbbreviatedRelativeLocationPath
AbbreviatedAbsoluteLocationPath   ::=   '//'RelativeLocationPath
AbbreviatedRelativeLocationPath   ::=   RelativeLocationPath '//' Step

Step            ::=   AxisSpecifier NodeTest Predicate*
                    | AbbreviatedStep
AxisSpecifier   ::=   AxisName '::'
                    | AbbreviatedAxisSpecifier
AbbreviatedStep            ::=   '.'
                               | '..'
AbbreviatedAxisSpecifier   ::=   '@'?

NodeTest   ::=   NameTest
               | NodeType '(' ')'
               | 'processing-instruction' '(' Literal ')'
NameTest   ::=   '*'
               | NCName ':' '*'
               | QName

Predicate     ::=   '[' PredicateExpr ']'
PredicateExpr ::=   Expr

PrimaryExpr   ::=   VariableReference
                  | '(' Expr ')'
                  | Literal
                  | Number
                  | FunctionCall

VariableReference   ::=   '$' QName

Literal    ::=   '"' [^'']* '"'
               | "'" [^']* "'"
Number     ::=   Digits ('.' Digits?)?
               | '.' Digits
Digits     ::=   [0-9]+

FunctionCall ::=   FunctionName '(' ( Argument ( ',' Argument )* )?')'
Argument     ::=   Expr
FunctionName ::=   QName - NodeType

NodeType   ::=   'comment'
               | 'text'
               | 'processing-instruction'
               | 'node'

AxisName   ::=   'ancestor'
               | 'ancestor-or-self'
               | 'attribute'
               | 'child'
               | 'descendant'
               | 'descendant-or-self'
               | 'following'
               | 'following-sibling'
               | 'namespace'
               | 'parent'
               | 'preceding'
               | 'preceding-sibling'
               | 'self'

QName            ::=   PrefixedName
                     | UnprefixedName
PrefixedName     ::=    Prefix ':' LocalPart
UnprefixedName   ::=    LocalPart
Prefix           ::=   NCName
LocalPart        ::=   NCName

NCName   ::=   Name - (Char* ':' Char*)
  \end{abnfblock}%$
%  \caption{Sintassi delle espressioni \emph{XPath}}
%  \label{code:sintassixpath}
%\end{code}
dove \abnf{Name} \`e nella stessa forma dei nomi \emph{XML} visti
nel codice \ref{code:dtdcomuni} a pagina \pageref{code:dtdcomuni}.

\subsubsection{Location paths}
Le espressioni pi\`u importanti sono quelle che usano
\abnf{LocationPath} in quanto permettono di selezionare un set di nodi
relativo al \emph{context node} e di contenere ricorsivamente altre
espressioni. Ci sono due tipi di \emph{location
  path}: relativi e assoluti.

Quelli relativi consistono in una sequenza di uno o pi\`u \emph{steps}
separati da ``/''. Ogni \emph{step} seleziona un set di nodi relativo
ad un \emph{context node}, poi ogni nodo di tale set \`e usato come
\emph{context node} per lo \emph{step} successivo. 

Un \emph{path} assoluto consiste in un carattere ``/'' seguito da un
\emph{path} relativo. L'effetto di rendere un path assoluto \`e che il
\emph{context node} viene posto uguale alla radice del documento che
conteneva il \emph{context node} e poi viene valutato il path
relativo indicato.

Un singolo \emph{step} ha tre parti:
\begin{description}
  \item[un \emph{axis}] che specifica la relazione nell'albero tra il
    nodo selezionato e il \emph{context tree};
  \item[un \emph{note test}] che specifica il tipo o il nome del nodo selezionato;
  \item[zero o pi\`u \emph{predicati}] che usano espressioni
    arbitrarie per raffinare il set di nodi selezionato.
\end{description}

Gli \emph{axes} possono essere i seguenti:
\begin{description}
  \item[child] i figli del \emph{context node};
  \item[descendant] i figli del \emph{context node} e ricorsivamente tutti i
    discendenti;
  \item[parent] il genitore del \emph{context node}, se esiste;
  \item[anchestor] il genitore e ricorsivamente tutti gli antenati;
  \item[following-sibling] il fratello in ordine successivo del
    \emph{context node}. I nodi di tipo attributo o namespace hanno
    \emph{following-sibling} vuoto;
  \item[preceding-sibling] come il precedente ma per il fratello
    precedente;
  \item[following] tutti i nodi che sono successivi al \emph{context
    node} nell'ordine del documento, escludendo i discendenti e i nodi
    di tipo attributo e namespace;
  \item[preceding] come il precedente ma per i nodi precedenti;
  \item[attribute] contiene tutti gli attributi del \emph{context
    node}. Se il \emph{context node} non \`e un elemento, allora \`e
    vuoto. Esiste una abbreviazione per questo axis che corrisponde a
    \xml{@};
  \item[namespace] contiene tutti i namespace del \emph{context node}.
    Se il \emph{context node} non \`e un elemento, allora \`e vuoto;
  \item[self] contiene il \emph{context node} stesso;
  \item[descendant-or-self] contiene il \emph{context node} e i
    discendenti di questo;
  \item[ancestor-or-self] come il precedente per gli antenati.
\end{description}
Si pu\`o notare che \xml{anchestor}, \xml{descendant},
\xml{following}, \xml{preceding} e \xml{self} costituiscono una
partizione dell'insieme di
tutti i nodi di tipo elemento del documento di appartenenza del
\emph{context node}.

Passando ai \emph{node test} � bene premettere che ogni \emph{axis} ha il
suo \emph{tipo principale di nodo}, per \abnf{attribute} \`e
\emph{attributo}, per \abnf{namespace} \`e \emph{namespace}, e per
tutti gli altri \`e \emph{elemento}. Detto ci\`o, i \emph{node test}
servono per limitare il set dei nodi dell'axis. Ogni \emph{node test}
torna un valore di verit\`a per ogni nodo, se vero il nodo in
questione viene selezionato, altrimenti no.

Osservando la grammatica si nota che un \emph{node test} pu\`o essere
nelle forme:
\begin{itemize}
  \item \xml{QName} \`e vero se il tipo del nodo \`e il tipo
    principale di nodo per l'\emph{axis} in questione e se ha un
    nome esteso uguale al nome esteso specificato da \xml{QName}. Ad
    esempio l'espressione \xml{child::div} seleziona gli elementi
    \emph{div} figli del \emph{context node}, \xml{attribute::href}
    seleziona gli attributi \xml{href} del \emph{context node};
  \item \xml{NCName:*} \`e vero per ogni nodo del tipo principale che
    ha l'\emph{URI} del \emph{namespace} uguale all'espansione del
    prefisso \xml{NCName};
  \item \xml{*} \`e vero per ogni nodo di tipo uguale al tipo di nodo
    principale, ad esempio \xml{child::*} seleziona tutti i figli del
    \emph{context node};
  \item \xml{text()} \`e vero per ogni nodo \emph{testo}. \xml{child::text()}
    seleziona i nodi testo figli del \emph{context node};
  \item \xml{comment()} \`e vero per ogni nodo \emph{commento};
  \item \xml{processing-instruction()} \`e vero per ogni nodo
    \emph{processing instruction};
  \item \xml{processing-instruction(LITERAL)} \`e vero per ogni
    \emph{processing instruction} che ha nome uguale al valore
    \xml{LITERAL};
  \item \xml{node()} \`e vero per ogni nodo di ogni tipo.
\end{itemize}

Per quanto riguarda i \emph{predicati}, un \emph{axis} pu\`o essere
\emph{discendente} che seleziona nodi nell'ordine del documento, o
\emph{ascendente} se li seleziona nell'ordine
inverso. \emph{ancestor}, \emph{ancestor-or-self}, \emph{preceding} e
\emph{preceding-sibling} sono gli \emph{axis} ascendenti, gli altri
sono discendenti.

La \emph{proximity position} di un nodo membro di un \emph{node-set}
risultante di un \emph{axis} indica la posizione di tale nodo all'interno del
\emph{node-set}. Se l'\emph{axis} \`e discendente il \emph{node-set}
\`e ordinato nell'ordine del documento, viceversa, se \`e ascendente, \`e ordinato in
senso contrario.

Un \emph{predicato} filtra un \emph{node-set} risultante da un
\emph{axis} nel seguente modo: per ogni nodo del \emph{node-set}
originale valuta l'espressione \xml{PredicateExpr} usando come
\emph{context node} il nodo che sta valutando, come \emph{context
  position} la \emph{proximity position} del nodo che sta valutando e
come \emph{context size} il numero di nodi nel \emph{node-set}.
L'espressione \xml{PredicateExpr} \`e valutata determinando \xml{Expr} e
convertendo il risultato in un booleano. Se il risultato \`e un numero
viene convertito in \emph{vero} se \`e uguale alla \emph{context
  position}, in \emph{falso} altrimenti. Se \`e un \emph{node-set} \`e
\emph{vero} se non \`e vuoto, \emph{falso} altrimenti. Se \`e una
stringa \`e \emph{vero} se non \`e vuota, \emph{falso} altrimenti.

Esistono delle utili abbreviazioni per rendere l'espressione pi\`u
concisa e leggibile, queste abbreviazioni sono:
\begin{itemize}
  \item \xml{child::} pu\`o essere omesso, cosicch\'e l'espressione
    \xml{div/a} \`e equivalente a \xml{child::div/child::a};
  \item \xml{attribute::} pu\`o essere abbreviato in \xml{@}, ad
    esempio\\ 
    \xml{child::div/child::a[attribute::href="abc"]} diventa\\
    \xml{div/a[@href="abc"]};
  \item \xml{//} \`e l'abbreviazione di
    \xml{/descendant-or-self::node()/}. Ad esempio \xml{//a} \`e
    equivalente a \xml{/descendant-or-self::node()/child::a} (\`e
    stata usata anche l'abbreviazione di \xml{child::});
  \item \xml{.} \`e l'abbreviazione per \xml{self::node()};
  \item \xml{..} \`e l'abbreviazione per \xml{parent::node()}.
\end{itemize}

\subsubsection{Funzioni}
Come gi\`a detto \emph{XPath} fornisce un set di funzioni di
\emph{core} che possono essere usate. Nella grammatica indicata prima
\`e possibile vedere come usare le funzioni chiamate
\abnf{FunctionCall}, comunque sia queste sono costituite da un nome,
un valore di ritorno e un insieme di parametri.

Per vedere l'elenco completo di queste funzioni di \emph{core} si
rimanda a
\link{http://www.w3.org/TR/xpath/\#section-Node-Set-Functions}.

\subsection{XSLT}\label{sec:xslt}
Le trasformazioni \emph{XSLT} sono espresse in \emph{XML} e viene
usato un \emph{namespace} specifico per gli elementi che ha \emph{URI}
\link{http://www.w3.org/1999/XSL/Transform}. Solitamente si usa il
prefisso \xml{xsl} per tale \emph{namespace}.

Come gi\`a detto in sezione \ref{sec:xsl}, queste trasformazioni
servono per poter trasformare tramite un \emph{processore} un
documento \emph{XML} in un documento \emph{XSL-FO}. O perlomeno questo
era lo scopo iniziale. In realt\`a \emph{XSLT} permette di trasformare
un generico \emph{albero sorgente} in un generico \emph{albero
  destinazione}, anche non in formato \emph{XML}.

Le trasformazioni prendono il nome di \emph{stylesheet} perch\'e,
quando applicate per ottenere un documento \emph{XSL-FO}, funzionano
come un foglio di stile. Gli \emph{stylesheet} sono
composti da una serie di \emph{regole template}. Ogni \emph{regola
  template} \`e composta da due parti:
\begin{itemize}
  \item un \emph{pattern} da matchare con l'\emph{albero sorgente};
  \item un \emph{template} da istanziare come parte dell'\emph{albero
    destinazione}.
\end{itemize}

Un template pu\`o contenere elementi che specificano parti letterali
della struttura del
risultante, ma pu\`o anche contenere elementi del namespace
\emph{XSLT} che specificano istruzioni per creare frammenti
dell'albero risultante. Durante l'istanziazione ogni istruzione \`e
sostituita con il frammento corrispondente.

Quando un template \`e istanziato, avviene rispetto ad un \emph{nodo
  corrente} e ad una \emph{lista di nodi correnti}. Qualche istruzione
cambia questi due valori, durante l'istanziazione di queste regole la
\emph{lista dei nodi correnti} cambia in una nuova lista di nodi e
ogni nodo a turno diventa il \emph{nodo corrente}. Dopo
l'istanziazione il \emph{nodo corrente} e la \emph{lista dei nodi
  correnti} tornano ai valori precedenti.

Come gi\`a accennato, \emph{XSLT} fa uso di \emph{XPath} per
selezionare elementi durante il processamento.

\`E possibile definire uno \emph{stylesheet} \emph{XSLT} per un certo
documento \emph{XML} direttamente all'interno di quest'ultimo, usando
l'elemento \xml{xsl:stylesheet}. Oppure \`e possibile creare un
documento indipendente \emph{XML} contenente esclusivamente lo
\emph{stylesheet}, questo si ottiene usando il solito elemento
\xml{xsl:stylesheet} come radice del documento.

I nodi di \emph{input} della trasformazione vengono processati nel
seguente modo:
\begin{enumerate}
\item viene processata inizialmente una lista contenente solo la radice
  dell'albero sorgente;
\item una lista di nodi viene processata concatenando, nell'ordine della
  lista stessa, i nodi risultanti dal processamento di ogni nodo della
  lista;
\item un nodo \`e processato cercando i \emph{matching} con i
  \emph{pattern} definiti nelle regole, scegliere il migliore se non
  univoco e istanziare il relativo \emph{template} della regola
  usando il nodo come \emph{nodo corrente} e la lista dei nodi
  sorgenti come \emph{lista dei nodi correnti};
\item un template pu\`o contenere istruzioni per selezionare una lista
  ulteriore di nodi da processare. 
\end{enumerate}
\subsubsection{Patterns}
I patterns usano la grammatica indicata nel codice
\ref{code:patternxslt}
\begin{code}
  \begin{abnfblock}
Pattern             ::= LocationPathPattern 	
                      | Pattern '|' LocationPathPattern 	
LocationPathPattern ::= '/' RelativePathPattern? 	
                      | IdKeyPattern (('/' | '//') RelativePathPattern)? 	
                      | '//'? RelativePathPattern 	
IdKeyPattern        ::= 'id' '(' Literal ')' 	
                      | 'key' '(' Literal ',' Literal ')' 	
RelativePathPattern ::= StepPattern 	
                      | RelativePathPattern '/' StepPattern 	
                      | RelativePathPattern '//' StepPattern 	
StepPattern         ::= ChildOrAttributeAxisSpecifier NodeTest Predicate* 	
ChildOrAttributeAxisSpecifier ::= AbbreviatedAxisSpecifier 	
                                | ('child' | 'attribute') '::'
  \end{abnfblock}
  \caption{Grammatica di un pattern \emph{XSLT}}
  \label{code:patternxslt}
\end{code}
che si appoggia alla grammatica e ai tipi di dato di
\emph{XPath}\footnote{Vedere sezione \ref{sec:xpath}\label{node:xsltvedixpath}.} per
completarsi.

Un \emph{pattern} ha la funzione di trovare un \emph{match} con un nodo
del documento sorgente. Poich\'e il suo funzionamento \`e di fatto
quello delle espressioni \emph{XPath}, non viene approfondito
ulteriormente l'argomento.

\subsubsection{Template}
Un \emph{template} \`e una regola associata ad un \emph{pattern} e
viene definita con un elemento di tipo \xml{xsl:template} che possiede
un attributo \xml{match} che corrisponde al \emph{pattern}. Il
contenuto dell'elemento \xml{xsl:template} rappresenta l'albero di
output della trasformazione applicata da questa specifica regola.

Un esempio di definizione di regola \`e quello del codice
\ref{code:esempiotemplatexslt}
\begin{code}
  \begin{xmlblock}
<xsl:template match="in">
  <out/>
</xsl:template>
  \end{xmlblock}
  \caption{Esempio di definizione di regola \emph{XSLT}}
  \label{code:esempiotemplatexslt}
\end{code}
che, quando esiste un \emph{match} con un elemento \xml{in}, produce un
elemento \xml{out}.

Come detto prima un template pu\`o avere delle istruzioni per
processare una ulteriore lista di nodi, tale istruzione \`e
\xml{xsl:apply-template} con l'attributo opzionale \xml{select} che
deve essere una \abnf{Expr} di \emph{XPath}\footnote{Vedi nota
  \ref{node:xsltvedixpath}.}. Se \xml{select} non \`e presente vengono
aggiunti tutti i figli del \emph{nodo corrente}. Ad esempio nel codice
\ref{code:esempiotemplatexslt2}
\begin{code}
  \begin{xmlblock}
<xsl:template match="div">
  <blocco>
    <xsl:apply-template select="p"/>
  </blocco>
</xsl:template>
  \end{xmlblock}
  \caption{Esempio di definizione di regola \emph{XSLT} con applicazione di regola}
  \label{code:esempiotemplatexslt2}
\end{code}
quando avviene un \emph{match} con un nodo di \emph{input} \xml{div},
vengono aggiunti alla lista di processamento tutti i \xml{p} suoi
discendenti.

La definizione dei \xml{xsl:template} prevede di indicare un attributo
\xml{priority} con un valore numerico, che permette di identificare
quale regola scegliere in caso di conflitti durante il \emph{matching}
di un nodo di \emph{input}.

Le definizioni dei \emph{template} danno la possibilit\`a di definire
un attributo \xml{mode} che permette di creare una sorta di classe
assegnandoli un certo nome. Quando un'applicazione
\xml{xsl:apply-templates} indica anch'essa un attributo \xml{mode},
allora i \emph{matching} dei nodi verranno scelti solo tra i template
che presentano un attributo \xml{name} uguale.

Una definizione di \emph{template} permette anche di assegnare un nome
alla regola tramite l'attributo opzionale \xml{name}. Se un
\emph{template} ha un nome definito, \`e possibile applicarlo direttamente senza
\emph{matching} con l'istruzione \xml{xsl:call-template} e il suo
attributo \xml{name} che, ovviamente, deve corrispondere al nome del
template. Nel caso di chiamata diretta gli attributi \xml{match}, \xml{mode} e
\xml{priority} vengono ignorati.

All'interno dei \xml{xsl:template} \`e possibile indicare quale deve
essere l'albero risultante del \emph{templete}, non solo scrivendolo
direttamente nel contenuto della definizione come visto negli esempi
prima, ma \`e possibile anche usare a tal scopo due istruzioni particolari:
\xml{xsl:element} e \xml{xsl:attribute}.

Inolte l'\emph{XSLT} prevede anche delle istruzioni condizionali come
\xml{xsl:if} e \xml{xsl:choose} e altre funzionalit\`a che esulano
dagli scopi di questo testo. Per approfondire si rimanda a
\link{http://www.w3.org/TR/xslt}.
%\subsection{XSL-FO}
\end{document}
